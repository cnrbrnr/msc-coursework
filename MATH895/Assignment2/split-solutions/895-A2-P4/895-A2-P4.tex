\documentclass[10pt]{article}
\usepackage[margin=1.3cm]{geometry}

% Packages
\usepackage{amsmath, amsfonts, amssymb, amsthm}
\usepackage{bbm} 
\usepackage{dutchcal} % [dutchcal, calrsfs, pzzcal] calligraphic fonts
\usepackage{graphicx}
\usepackage[T1]{fontenc}
\usepackage[tracking]{microtype}

% Palatino for text goes well with Euler
\usepackage[sc,osf]{mathpazo}   % With old-style figures and real smallcaps.
\linespread{1.025}              % Palatino leads a little more leading

% Euler for math and numbers
\usepackage[euler-digits,small]{eulervm}

% Command initialization
\DeclareMathAlphabet{\pazocal}{OMS}{zplm}{m}{n}
\graphicspath{{./images/}}

% Custom Commands
\newcommand{\bs}[1]{\boldsymbol{#1}}
\newcommand{\E}{\mathbb{E}}
\newcommand{\var}[1]{\text{Var}\left(#1\right)}
\newcommand{\bp}[1]{\left({#1}\right)}
\newcommand{\mbb}[1]{\mathbb{#1}}
\newcommand{\1}[1]{\mathbbm{1}_{#1}}
\newcommand{\mc}[1]{\mathcal{#1}}
\newcommand{\nck}[2]{{#1\choose#2}}
\newcommand{\pc}[1]{\pazocal{#1}}
\newcommand{\ra}[1]{\renewcommand{\arraystretch}{#1}}
\newcommand*{\floor}[1]{\left\lfloor#1\right\rfloor}
\newcommand*{\ceil}[1]{\left\lceil#1\right\rceil}

\DeclareMathOperator{\Var}{Var}
\DeclareMathOperator{\Cov}{Cov}
\DeclareMathOperator{\diag}{diag}
\DeclareMathOperator{\as}{a.s.}
\DeclareMathOperator{\ale}{a.e.}
\DeclareMathOperator{\st}{s.t.}
\DeclareMathOperator{\io}{i.o.}
\DeclareMathOperator{\wip}{w.p.}
\DeclareMathOperator{\iid}{i.i.d.}
\DeclareMathOperator{\ifff}{if\;and\;only\;if}
\DeclareMathOperator{\inv}{inv}

\newtheorem{theorem}{Theorem}
\newtheorem{lemma}{Lemma}

\begin{document}
    \begin{center}
        {\bf\large{MATH 895: CORE COURSE IN PROBABILITY}}
        \smallskip
        \hrule
        \smallskip
        {\bf Assignment} 2\hfill {\bf Connor Braun} \hfill {\bf 2024-02-17}
    \end{center}
    \noindent{\bf Problem 4}\\[5pt]
    Let $(X_n)_{n\geq 1}$ be a sequence of $\iid$ random variables. Prove that $P(|X_n|>n\;\text{for infinitely many $n$})=0$ if and only if $\E(|X_1|)<\infty$.\\[5pt]
    {\bf Proof}\hspace{5pt} Both directions can be proven using the Borel-Cantelli lemma. Let $(\Omega,\mc{F},P)$ be the common probability space on which $X_n$ is defined for $n\geq 1$, and define $A_k:=\{\omega\in\Omega:k\leq |X_1|<k+1\}$ for $k\geq 1$. Then
    \begin{align*}
        \sum_{n=1}^\infty P(|X_n|>n)&=\sum_{n=1}^\infty \sum_{k=n}^\infty P(A_k)=\sum_{k=1}^\infty\sum_{n=1}^kP(A_k)=\sum_{k=1}^\infty kP(A_k)
    \end{align*}
    where the first equality holds since $A_i\cap A_j=\emptyset$ whenever $i\neq j$. Next, define a new random variable $Y:(\Omega,\mc{F},P)\rightarrow\mbb{N}$ as $Y(\omega)=\sum_{k=1}^\infty k\1{A_k}(\omega)$ so that for any $\omega\in\Omega$:
    \[Y(\omega)\leq |X_1(\omega)|<Y(\omega)+1\quad\Rightarrow\quad \E(Y)\leq \E(|X_1|)<\E(Y)+1\quad\Rightarrow\quad \E(|X_1|)-1 <\E(Y)\leq \E(|X_1|).\]
    However, by construction, we also have
    \[\E(Y)=\E\bp{\sum_{k=1}^\infty k\1{A_k}}=\sum_{k=1}^\infty k\E(\1{A_k})=\sum_{k=1}^\infty kP(A_k)=\sum_{n=1}^\infty P(|X_n|>n)\]
    where the second equality holds by the monotone convergence theorem. With this we obtain a bound on the series
    \begin{align*}
        \E(|X_1|)-1\leq \sum_{n=1}^\infty P(|X_n|>n) \leq \E(|X_1|).\tag{2}
    \end{align*}
    Now, suppose that $\E(|X_1|)<\infty$. By (3), $\sum_{n=1}^\infty P(|X_n|>n)<\infty$ also, and by the Borel-Cantelli lemma we thus have that $P(|X_n|>n\;\io)=0$. \\[5pt]
    For the converse, suppose instead that $P(|X_n|>n\;\io)=0$. Once more by (the contrapositive of) the Borel-Cantelli lemma, it must be that either $\sum_{n=1}^\infty P(|X_n|>n)<\infty$ or the events $(\{\omega\in\Omega:|X_n(\omega)|>n\})_{n\geq 1}$ are not mutually independent.
    However, by assumption the random variables $(X_n)_{n\geq 1}$ are $\iid$, so the events $(\{\omega\in\Omega:|X_n(\omega)|>n\})_{n\geq 1}$ are mutually independent, and we are forced to conclude that $\sum_{n=1}^\infty P(|X_n|>n)<\infty$. Then, by (2):
    \[\E(|X_1|)\leq 1+\sum_{n=1}^\infty P(|X_n|>n)\leq \infty\]
    so that $X_1$ has finite expectation, and we are done.\hfill{$\qed$}\\[5pt]
\end{document}