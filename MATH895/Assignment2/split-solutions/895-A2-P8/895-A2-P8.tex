\documentclass[10pt]{article}
\usepackage[margin=1.3cm]{geometry}

% Packages
\usepackage{amsmath, amsfonts, amssymb, amsthm}
\usepackage{bbm} 
\usepackage{dutchcal} % [dutchcal, calrsfs, pzzcal] calligraphic fonts
\usepackage{graphicx}
\usepackage[T1]{fontenc}
\usepackage[tracking]{microtype}

% Palatino for text goes well with Euler
\usepackage[sc,osf]{mathpazo}   % With old-style figures and real smallcaps.
\linespread{1.025}              % Palatino leads a little more leading

% Euler for math and numbers
\usepackage[euler-digits,small]{eulervm}

% Command initialization
\DeclareMathAlphabet{\pazocal}{OMS}{zplm}{m}{n}
\graphicspath{{./images/}}

% Custom Commands
\newcommand{\bs}[1]{\boldsymbol{#1}}
\newcommand{\E}{\mathbb{E}}
\newcommand{\var}[1]{\text{Var}\left(#1\right)}
\newcommand{\bp}[1]{\left({#1}\right)}
\newcommand{\mbb}[1]{\mathbb{#1}}
\newcommand{\1}[1]{\mathbbm{1}_{#1}}
\newcommand{\mc}[1]{\mathcal{#1}}
\newcommand{\nck}[2]{{#1\choose#2}}
\newcommand{\pc}[1]{\pazocal{#1}}
\newcommand{\ra}[1]{\renewcommand{\arraystretch}{#1}}
\newcommand*{\floor}[1]{\left\lfloor#1\right\rfloor}
\newcommand*{\ceil}[1]{\left\lceil#1\right\rceil}

\DeclareMathOperator{\Var}{Var}
\DeclareMathOperator{\Cov}{Cov}
\DeclareMathOperator{\diag}{diag}
\DeclareMathOperator{\as}{a.s.}
\DeclareMathOperator{\ale}{a.e.}
\DeclareMathOperator{\st}{s.t.}
\DeclareMathOperator{\io}{i.o.}
\DeclareMathOperator{\wip}{w.p.}
\DeclareMathOperator{\iid}{i.i.d.}
\DeclareMathOperator{\ifff}{if\;and\;only\;if}
\DeclareMathOperator{\inv}{inv}

\newtheorem{theorem}{Theorem}
\newtheorem{lemma}{Lemma}

\begin{document}
    \begin{center}
        {\bf\large{MATH 895: CORE COURSE IN PROBABILITY}}
        \smallskip
        \hrule
        \smallskip
        {\bf Assignment} 2\hfill {\bf Connor Braun} \hfill {\bf 2024-02-17}
    \end{center}
    \noindent{\bf Problem 8}\\[5pt]
    For $n\geq 1$, let $X_n$ be a uniform random variable on the interval $[-n,n]$. Let $F_n$ denote the distribution function of $X_n$. For every $x\in\mbb{R}$, find $\lim_{n\rightarrow\infty}F_n(x)$, if the limit exists.\\[5pt]
    {\bf Solution}\hspace{5pt} For $n\geq 1$, the uniform distribution function $F_n$ is given by
    \begin{align*}
        F_n(x)=\begin{cases}
            0,\quad&\text{if $x<-n$}\\
            \frac{x+n}{2n},\quad&\text{if $-n\leq x<n$}\\
            1,\quad&\text{if $x\geq 1$}.
        \end{cases}
    \end{align*}
    Now, fixing $x\in\mbb{R}$, we are interested in the limiting behavior of the sequence $\{F_n(x)\}_{n\geq 1}$. For this arbitrary point, $\exists N\in\mbb{N}$ so that $n\geq N$ implies that $|x|< n$, and further $F_n(x)=\tfrac{x+n}{2n}$. Of course, the limit of $(F_n(x))_{n\geq 1}$ and $(F_n(x))_{n\geq N}$ are the same, so we may compute
    \begin{align*}
        \lim_{n\rightarrow\infty} F_n(x)=\lim_{n\rightarrow\infty}\frac{x+n}{2n}=\frac{1}{2}
    \end{align*}
    which implies that $\lim_{n\rightarrow\infty}F_n=1/2=:F$ pointwise, a constant function over $\mbb{R}$. Interestingly, this limit is not a distribution function since $\lim_{x\rightarrow-\infty}F(x)=1/2\neq 0$, and $\lim_{x\rightarrow\infty}F(x)=1/2\neq 1$.\hfill{$\qed$}\\[5pt]
\end{document}