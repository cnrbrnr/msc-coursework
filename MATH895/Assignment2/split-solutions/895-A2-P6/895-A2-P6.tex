\documentclass[10pt]{article}
\usepackage[margin=1.3cm]{geometry}

% Packages
\usepackage{amsmath, amsfonts, amssymb, amsthm}
\usepackage{bbm} 
\usepackage{dutchcal} % [dutchcal, calrsfs, pzzcal] calligraphic fonts
\usepackage{graphicx}
\usepackage[T1]{fontenc}
\usepackage[tracking]{microtype}

% Palatino for text goes well with Euler
\usepackage[sc,osf]{mathpazo}   % With old-style figures and real smallcaps.
\linespread{1.025}              % Palatino leads a little more leading

% Euler for math and numbers
\usepackage[euler-digits,small]{eulervm}

% Command initialization
\DeclareMathAlphabet{\pazocal}{OMS}{zplm}{m}{n}
\graphicspath{{./images/}}

% Custom Commands
\newcommand{\bs}[1]{\boldsymbol{#1}}
\newcommand{\E}{\mathbb{E}}
\newcommand{\var}[1]{\text{Var}\left(#1\right)}
\newcommand{\bp}[1]{\left({#1}\right)}
\newcommand{\mbb}[1]{\mathbb{#1}}
\newcommand{\1}[1]{\mathbbm{1}_{#1}}
\newcommand{\mc}[1]{\mathcal{#1}}
\newcommand{\nck}[2]{{#1\choose#2}}
\newcommand{\pc}[1]{\pazocal{#1}}
\newcommand{\ra}[1]{\renewcommand{\arraystretch}{#1}}
\newcommand*{\floor}[1]{\left\lfloor#1\right\rfloor}
\newcommand*{\ceil}[1]{\left\lceil#1\right\rceil}

\DeclareMathOperator{\Var}{Var}
\DeclareMathOperator{\Cov}{Cov}
\DeclareMathOperator{\diag}{diag}
\DeclareMathOperator{\as}{a.s.}
\DeclareMathOperator{\ale}{a.e.}
\DeclareMathOperator{\st}{s.t.}
\DeclareMathOperator{\io}{i.o.}
\DeclareMathOperator{\wip}{w.p.}
\DeclareMathOperator{\iid}{i.i.d.}
\DeclareMathOperator{\ifff}{if\;and\;only\;if}
\DeclareMathOperator{\inv}{inv}

\newtheorem{theorem}{Theorem}
\newtheorem{lemma}{Lemma}

\begin{document}
    \begin{center}
        {\bf\large{MATH 895: CORE COURSE IN PROBABILITY}}
        \smallskip
        \hrule
        \smallskip
        {\bf Assignment} 2\hfill {\bf Connor Braun} \hfill {\bf 2024-02-17}
    \end{center}
    \noindent{\bf Problem 6}\\[5pt]
    We aim to show that the base-10 decimal expansion of a random number in $[0,1)$ satisfies the strong law of large numbers. For $x\in\mbb{R}$, let $\langle x\rangle:=x-\floor{x}$ denote the fractional part of $x$. Consider the transformation $T:[0,1)\rightarrow[0,1)$ given by $T(x)=\langle 10x\rangle$.\\[5pt]
    {\bf a)}\hspace{5pt} Let $\mc{B}$ be the Borel $\sigma$-field on $[0,1)$ and $P$ the Lebesgue measure on $\mc{B}$. Show that $P$ is $T$-invariant.\\[5pt]
    {\bf Proof}\hspace{5pt} Let $D=\{0,1,2,\dots,9\}$ and observe that for any $x\in[0,1)$, we may write $x=0.x_1x_2x_3\cdots$ with $x_i\in D$ for $i\geq 1$, a decimal expansion of $x$. From this perspective, $T$ can be viewed as a one-sided shift so that $T(x)=0.x_2x_3x_4\cdots$ and the preimage of $\{x\}$ under $T$ is given by
    \begin{align*}
        T^{-1}(\{x\})=\{0.x_0x_1x_2x_3\cdots:x_0\in D\}.
    \end{align*}
    Now, we seek to show that $\forall A\in\mc{B}$, $P(T^{-1}(A))=P(A)$. However, it suffices to check that this holds for elements of the generating class of $\mc{B}$. Fix one such set $(a,b)$, with $0\leq a,b<1$ and take $(a_i)_{i\geq 1}$, $(b_i)_{i\geq 1}$ with $a_i,b_i\in D$ for $i\geq 1$ so that $a=0.a_1a_2a_3\cdots,b=0.b_1b_2b_3\cdots$. We claim that
    \begin{align*}
        T^{-1}((a,b))=\bigsqcup_{k\in D}\left(\tfrac{a}{10}+\tfrac{k}{10},\tfrac{b}{10}+\tfrac{k}{10}\right).\tag{3}
    \end{align*}
    Supposing first that $\omega\in T^{-1}((a,b))$ with $\omega=0.\omega_1\omega_2\omega_3\cdots$, we have that $T(\omega)=0.\omega_2\omega_3\in(a,b)$, so
    \[a<0.\omega_2\omega_3\cdots<b\quad\Rightarrow\quad\tfrac{a}{10}<0.0\omega_2\omega_3\cdots <\tfrac{b}{10}\quad\Rightarrow\quad \tfrac{a}{10}+\tfrac{\omega_1}{10}<\omega<\tfrac{b}{10}+\tfrac{\omega_1}{10}\]
    so $\omega\in(a/10+k/10,b/10+k/10)$ for some $k\in D$. If instead $\omega\in (a/10+k/10,b/10+k/10)$ for some fixed $k\in D$, then we may write $\omega=0.k\omega_1\omega_2\cdots$ so that
    \begin{align*}
        \tfrac{a}{10}+\tfrac{k}{10}<\omega<\tfrac{b}{10}+\tfrac{k}{10}\quad&\Rightarrow\quad 0.ka_1a_2\cdots <0.k\omega_1\omega_2\cdots <0.kb_1b_2\cdots\\
        &\Rightarrow\quad 0.a_1a_2\cdots <0.\omega_1\omega_2\cdots < 0.b_1b_2\cdots\\
        &\Rightarrow\quad a<T(\omega)<b
    \end{align*}
    so indeed, $T(\omega)\in(a,b)$ and $\omega\in T^{-1}((a,b))$. This establishes (3), from which we may directly compute
    \begin{align*}
        P(T^{-1}((a,b)))=P\bp{\bigsqcup_{k=0}^9\bp{\tfrac{a}{10}+\tfrac{k}{10},\tfrac{b}{10}+\tfrac{k}{10}}}=\sum_{k=0}^9P\bp{\tfrac{a}{10}+\tfrac{k}{10},\tfrac{b}{10}+\tfrac{k}{10}}=\sum_{k=0}^9\bp{\tfrac{b}{10}-\tfrac{a}{10}}=b-a=P((a,b)).
    \end{align*}
    Thus, for any member $A$ of a generating class of $\mc{B}$, we have $P(T^{-1}(A))=P(A)$, so $P$ is $T$-invariant.\hfill{$\qed$}\\[5pt]
    {\bf b)} Write $x\in[0,1)$ in base ten as $x=\sum_{n=1}^\infty\tfrac{d_n}{10^n}$, where $d_n\in D$. Note that, almost surely, this representation for $x$ is unique, and so $d_n=d_n(x)$ is a $P-\ale$ defined function of $x$. Find some $f\in L^{1}([0,1),\mc{B},P)$ so that $f(T^{n-1}(x))=d_n(x)$ for every $n\geq 1$.\\[5pt]
    {\bf Solution}\hspace{5pt} For any $n\geq 1$, $x\in[0,1)$, we have $T^{n-1}(x)=\sum_{k=n}^\infty\tfrac{d_k(x)}{10^k}$. Setting $f:[0,1)\rightarrow D$ with $f(x)=\floor{10x}$, we have
    \[f(T^{n-1}(x))=\floor{10T^{n-1}(x)}=\floor{d_n(x)+\sum_{k=n+1}^\infty\tfrac{d_k(x)}{10^k}}=d_n(x)\]
    as desired. We need only verify that $f\in L^1([0,1),\mc{B},P)$. To see that $f$ is measurable, let us express it as
    \begin{align*}
        f(x)=\sum_{k=1}^9\1{\{10x\geq k\}}=\sum_{k=1}^9\1{\{x\geq k/10\}}
    \end{align*}
    which is a measurable function since for $k\in D$, $\{x\in[0,1):10x\geq k\}=\{x\in[0,1):x\geq k/10\}=[0.k,1)\in\mc{B}$. Now we compute
    \begin{align*}
        \int_0^1|f|dx=\int_0^1f(x)dx=\int_0^1\sum_{k=1}^9\1{\{x\geq k/10\}}dx=\sum_{k=1}^9\int_{0.k}^1dx=\frac{\sum_{k=1}^9k}{10}=4.5<\infty
    \end{align*}
    and thus $f\in L^1([0,1),\mc{B},P)$, the set of measurable and integrable functions on $([0,1),\mc{B},P)$.\hfill{$\qed$}\\[5pt]
    {\bf c)}\hspace{5pt} Assuming that $T$ is ergodic (which can be proved using Fourier series) find a constant $\alpha$ so that 
    \begin{align*}
        lim_{N\rightarrow\infty}\frac{1}{N}\sum_{n=1}^Nd_n=\alpha\qquad\as
    \end{align*}
    {\bf Solution}\hspace{5pt} Let $x\in[0,1)$. Since $d_n(x)=f(T^{n-1}(x))$ for $n\geq 1$ with $f\in L^1([0,1),\mc{B},P)$ and $T$ $P$-preserving, by the Birkhoff ergodic theorem $\exists \alpha\in L^1([0,1),\mc{B}_{inv(T)},P|_{\inv(T)})$ so that
    \[\frac{1}{N}\sum_{n=1}^Nd_n\overunderset{\as\text{ and in $L^1$}}{N\rightarrow\infty}{\xrightarrow{\hspace*{40pt}}}\alpha\]
    where $\mc{B}_{\inv(T)}$ is the $\sigma$-field of sets $A\in\mc{B}$ satisfying $T^{-1}(A)=A$ and $P|_{\inv(T)}$ is the restriction of $P$ to this $\sigma$-field. However, since $T$ is ergodic, $\alpha$ is constant $\as$ and furthermore
    \[\alpha=\int_0^1f(x)dx=4.5\]
    almost surely. Thus, we have found a constant $\alpha$ so that for any $x\in[0,1)$, the digits of the decimal expansion of $x$ satisfy the strong law of large numbers:
    \[\lim_{N\rightarrow\infty}\sum_{n=1}^Nd_n(x)=\alpha\qquad\as\]
    and we are finished.\hfill{$\qed$}\\[5pt]
\end{document}