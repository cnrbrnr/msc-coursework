\documentclass[10pt]{article}
\usepackage[margin=1.3cm]{geometry}

% Packages
\usepackage{amsmath, amsfonts, amssymb, amsthm}
\usepackage{bbm} 
\usepackage{dutchcal} % [dutchcal, calrsfs, pzzcal] calligraphic fonts
\usepackage{graphicx}
\usepackage[T1]{fontenc}
\usepackage[tracking]{microtype}

% Palatino for text goes well with Euler
\usepackage[sc,osf]{mathpazo}   % With old-style figures and real smallcaps.
\linespread{1.025}              % Palatino leads a little more leading

% Euler for math and numbers
\usepackage[euler-digits,small]{eulervm}

% Command initialization
\DeclareMathAlphabet{\pazocal}{OMS}{zplm}{m}{n}
\graphicspath{{./images/}}

% Custom Commands
\newcommand{\bs}[1]{\boldsymbol{#1}}
\newcommand{\E}{\mathbb{E}}
\newcommand{\var}[1]{\text{Var}\left(#1\right)}
\newcommand{\bp}[1]{\left({#1}\right)}
\newcommand{\mbb}[1]{\mathbb{#1}}
\newcommand{\1}[1]{\mathbbm{1}_{#1}}
\newcommand{\mc}[1]{\mathcal{#1}}
\newcommand{\nck}[2]{{#1\choose#2}}
\newcommand{\pc}[1]{\pazocal{#1}}
\newcommand{\ra}[1]{\renewcommand{\arraystretch}{#1}}
\newcommand*{\floor}[1]{\left\lfloor#1\right\rfloor}
\newcommand*{\ceil}[1]{\left\lceil#1\right\rceil}

\DeclareMathOperator{\Var}{Var}
\DeclareMathOperator{\Cov}{Cov}
\DeclareMathOperator{\diag}{diag}
\DeclareMathOperator{\as}{a.s.}
\DeclareMathOperator{\ale}{a.e.}
\DeclareMathOperator{\st}{s.t.}
\DeclareMathOperator{\io}{i.o.}
\DeclareMathOperator{\wip}{w.p.}
\DeclareMathOperator{\iid}{i.i.d.}
\DeclareMathOperator{\ifff}{if\;and\;only\;if}
\DeclareMathOperator{\inv}{inv}

\newtheorem{theorem}{Theorem}
\newtheorem{lemma}{Lemma}

\begin{document}
    \begin{center}
        {\bf\large{MATH 895: CORE COURSE IN PROBABILITY}}
        \smallskip
        \hrule
        \smallskip
        {\bf Assignment} 2\hfill {\bf Connor Braun} \hfill {\bf 2024-02-17}
    \end{center}
    \noindent{\bf Problem 2}\\[5pt]
    Prove the following lemma, used in a proof of the Borel-Cantelli lemma.\\[5pt]
    {\bf Lemma}\hspace{5pt} Let $(p_n)_{n\geq 1}$ be a sequence of real numbers with $0\leq p_n<1$ for $n\geq 1$. Then
    \[\prod_{n=1}^\infty(1-p_n)=0\quad\ifff\quad\sum_{n=1}^\infty p_n=\infty.\]
    {\bf Proof}\hspace{5pt} To begin, suppose $\prod_{n=1}^\infty(1-p_n)=0$. Additionally, suppose that $p_n\rightarrow 0$ as $n\rightarrow \infty$, for otherwise the consequent follows trivially. We have
    \begin{align*}
        -\infty=\log\bp{\lim_{N\rightarrow\infty}\prod_{n=1}^N(1-p_n)}&=\lim_{N\rightarrow\infty}\log\bp{\prod_{n=1}^N(1-p_n)}\tag{continuity of logarithms}\\
        &=\lim_{N\rightarrow\infty}\sum_{n=1}^N\log(1-p_n)=\sum_{n=1}^\infty\log(1-p_n)
    \end{align*}
    which further implies that 
    \[\sum_{n=1}^\infty\log\bp{\frac{1}{1-p_n}}=\infty.\]
    Using the well-known upper bound $\log(x)\leq x-1$, where $x\geq 1$, established by
    \[\log(x)=\int_1^x\frac{1}{t}dt\leq \int_1^xdt=x-1\]
    we have that
    \begin{align*}
        \infty=\sum_{n=1}^\infty\log\bp{\frac{1}{1-p_n}}\leq\sum_{n=1}^\infty\frac{1}{1-p_n}-1=\sum_{n=1}^\infty\frac{p_n}{1-p_n}.
    \end{align*}
    Now, since $p_n\rightarrow 0$ as $n\rightarrow\infty$, $\exists K\in\mbb{N}:$ $n\geq K$ implies that $p_n<\tfrac{1}{2}$, and further that $1-p_n\geq \tfrac{1}{2}$. By the above, this yields
    \begin{align*}
        \infty\leq\sum_{n=1}^\infty\frac{p_n}{1-p_n}=\sum_{n<K}\frac{p_n}{1-p_n}+\sum_{n\geq K}\frac{p_n}{1-p_n}\leq \sum_{n<K}\frac{p_n}{1-p_n}+2\sum_{n=K}^\infty p_n\quad\Rightarrow\quad \infty \leq 2\sum_{n=K}^\infty p_n \leq 2\sum_{n=1}^\infty p_n
    \end{align*}
    which, of course, implies that $\sum_{n=1}^\infty p_n=\infty$ as desired. For the converse, suppose that $\sum_{n=1}^\infty p_n=\infty$. Then, taking the Taylor expansion of $\log(1-x)$ with $0\leq x<1$ about $1$, we obtain
    \begin{align*}
        \log(1-x)&=\log(1)+(1-x-1)-\frac{1}{2!}(1-x-1)^2+\frac{2}{3!}(1-x-1)^3-\dots\\
        &=-x-\frac{x^2}{2}-\frac{x^3}{3}-\dots\\
        &=-\sum_{k=1}^\infty\frac{x^k}{k}\\
        &\leq -x.\tag{1}
    \end{align*}
    Using this along with the hypothesis, we have
    \begin{align*}
        \sum_{n=1}^\infty p_n=\infty\quad\Rightarrow\quad -\infty=-\sum_{n=1}^\infty p_n&\geq \sum_{n=1}^\infty\log(1-p_n)\tag{by (1)}\\
        &=\log\bp{\prod_{n=1}^\infty(1-p_n)}
    \end{align*}
    where in the last equality we implicitly used the continuity of the natural logarithm. Similarly taking the continuity of $e^x$ for granted:
    \begin{align*}
        0\leq \prod_{n=1}^\infty(1-p_n)\leq \lim_{N\rightarrow\infty}e^{-\sum_{n=1}^Np_n}=0
    \end{align*}
    whereby the squeeze theorem we conclude that $\prod_{n=1}^\infty(1-p_n)=0$, and the lemma is thus proven.\hfill{$\qed$}\\[5pt]
\end{document}