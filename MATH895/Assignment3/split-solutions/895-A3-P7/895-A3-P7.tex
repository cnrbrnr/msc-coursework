\documentclass[10pt]{article}
\usepackage[margin=1.3cm]{geometry}

% Packages
\usepackage{amsmath, amsfonts, amssymb, amsthm}
\usepackage{bbm} 
\usepackage{dutchcal} % [dutchcal, calrsfs, pzzcal] calligraphic fonts
\usepackage{graphicx}
\usepackage[T1]{fontenc}
\usepackage[tracking]{microtype}

% Palatino for text goes well with Euler
\usepackage[sc,osf]{mathpazo}   % With old-style figures and real smallcaps.
\linespread{1.025}              % Palatino leads a little more leading

% Euler for math and numbers
\usepackage[euler-digits,small]{eulervm}

% Command initialization
\DeclareMathAlphabet{\pazocal}{OMS}{zplm}{m}{n}
\graphicspath{{./images/}}

% Custom Commands
\newcommand{\bs}[1]{\boldsymbol{#1}}
\newcommand{\E}{\mathbb{E}}
\newcommand{\var}[1]{\text{Var}\left(#1\right)}
\newcommand{\bp}[1]{\left({#1}\right)}
\newcommand{\mbb}[1]{\mathbb{#1}}
\newcommand{\1}[1]{\mathbbm{1}_{#1}}
\newcommand{\mc}[1]{\mathcal{#1}}
\newcommand{\nck}[2]{{#1\choose#2}}
\newcommand{\pc}[1]{\pazocal{#1}}
\newcommand{\ra}[1]{\renewcommand{\arraystretch}{#1}}
\newcommand*{\floor}[1]{\left\lfloor#1\right\rfloor}
\newcommand*{\ceil}[1]{\left\lceil#1\right\rceil}

\DeclareMathOperator{\Var}{Var}
\DeclareMathOperator{\Cov}{Cov}
\DeclareMathOperator{\diag}{diag}
\DeclareMathOperator{\as}{a.s.}
\DeclareMathOperator{\ale}{a.e.}
\DeclareMathOperator{\st}{s.t.}
\DeclareMathOperator{\io}{i.o.}
\DeclareMathOperator{\wip}{w.p.}
\DeclareMathOperator{\iid}{i.i.d.}
\DeclareMathOperator{\ifff}{if\;and\;only\;if}
\DeclareMathOperator{\inv}{inv}

\newtheorem{theorem}{Theorem}
\newtheorem{lemma}{Lemma}

\begin{document}
    \begin{center}
        {\bf\large{MATH 895: CORE COURSE IN PROBABILITY}}
        \smallskip
        \hrule
        \smallskip
        {\bf Assignment} 3\hfill {\bf Connor Braun} \hfill {\bf 2024-03-22}
    \end{center}
    \noindent{\bf Problem 7}\\[5pt]
    Let $a_1,a_2,a_3\dots$ be a sequence of non-negative real numbers such that $\sum_{i=1}^\infty a_i=1$. Let $P$ be the measure on $\mc{B}(\mbb{R})$ defined as $\sum_{i=1}^\infty a_i\delta_i$. Find a sequence of probability measures $(P_n)_{n\geq 1}$, each having density
    with respect to the Lebesgue measure on $\mbb{R}$, such that $P_n\Longrightarrow P$ as $n\rightarrow\infty$\\[5pt]
    {\bf Solution}\hspace{5pt} For $n\geq 1$, define the function $f_n:\mbb{R}\rightarrow[0,1]$ by
    \begin{align*}
        f_n(x)=\sum_{k=1}^n\frac{a_k}{\sum_{j=1}^na_j}\frac{n}{2}\1{\left[k-\tfrac{1}{n},k+\tfrac{1}{n}\right]}(x).
    \end{align*}
    Next, define a sequence of functions $\{F_n\}_{n\geq 1}$ with
    \begin{align*}
        F_n(x)=\int_{-\infty}^xf_n(t)dt=\int_{-\infty}^x\sum_{k=1}^n\frac{a_k}{\sum_{j=1}^na_j}\frac{n}{2}\1{\left[k-\tfrac{1}{n},k+\tfrac{1}{n}\right]}(t)dt.
    \end{align*}
    We will show that $F_n$ is a distribution function, implying that $f_n$ is the corresponding density for $n\geq 1$. First, for $n\geq 1$ we have
    \begin{align*}
        \lim_{x\rightarrow\infty}F_n(x)=\int_{-\infty}^\infty\sum_{k=1}^n\frac{a_k}{\sum_{j=1}^na_j}\frac{n}{2}\1{\left[k-\tfrac{1}{n},k+\tfrac{1}{n}\right]}(t)dt=\sum_{k=1}^n\frac{a_k}{\sum_{j=1}^na_j}\frac{n}{2}\int_{k-\tfrac{1}{n}}^{k+\tfrac{1}{n}}dt=\sum_{k=1}^n\frac{a_k}{\sum_{j=1}^na_j}\frac{n}{2}\frac{2}{n}=1
    \end{align*}
    and
    \begin{align*}
        \lim_{x\rightarrow-\infty}F_n(x)=\lim_{x\rightarrow-\infty}\int_{-\infty}^xf_n(t)dt=0.
    \end{align*}
    Further, $f_n$ is non-negative on $\mbb{R}$, so $F_n$ is non-decreasing on $\mbb{R}$. To establish the right-continuity of $F_n$, take some $\alpha\in\mbb{R}$ and $\{x_n\}_{n\geq 1}$ with $x_n\searrow\alpha$ as $n\rightarrow\infty$. Then $(-\infty,x_n]\supset(-\infty,x_{n+1}]$ for all $n\geq 1$, so $\lim_{n\rightarrow\infty}(-\infty,x_n]=\cap_{n\geq 1}(-\infty,x_n]=(-\infty,x]$, and thus $\lim_{n\rightarrow\infty}\1{(-\infty,x_n]}=\1{(-\infty,x]}$. Fixing $n\geq 1$ for the moment, we find
    \begin{align*}
        \lim_{k\rightarrow\infty}F_n(x_k)=\lim_{k\rightarrow\infty}\int_{-\infty}^{x_k}f_n(t)dt&=\lim_{k\rightarrow\infty}\int_{-\infty}^\infty f_n(t)\1{(-\infty,x_k]}(t)dt\\
        &=\int_{-\infty}^\infty \lim_{k\rightarrow\infty}f_n(t)\1{(-\infty,x_k]}(t)dt\tag{by dominated convergence}\\
        &=\int_{-\infty}^\infty f_n(t)\1{(-\infty,x]}(t)dt\\
        &=\int_{-\infty}^xf_n(t)dt\\
        &=F_n(x)
    \end{align*}
    so indeed, $F_n$ is right-continuous. The dominated convergence theorem is applicable here, since $\forall k\geq 1$, $|f_n(t)\1{(-\infty,x_k]}(t)|\leq f_n(t)\in L^1(\mbb{R})$. \\[5pt]
    With this finding, we conclude that $\forall n\geq 1$, $F_n$ is a distribution corresponding to some random variable (call it $X_n$)
    admitting a density $f_n$. We can also compute the distribution function corresponding to $P$, which we shall denote $F$. For $x\in\mbb{R}$, we have
    \begin{align*}
        F(x)=\int_{-\infty}^xP(dx)=\int_{-\infty}^x\sum_{i=1}^\infty a_i\delta_i(dx)=\sum_{i\in\mbb{N}:i\leq x}a_i
    \end{align*}
    with points of discontinuity at all $i\in\mbb{N}$. We proceed by establishing the antecedent of the Helly-Bray theorem. For this, take $x\in\mbb{R}\setminus\mbb{N}$. If $x<1$, then $F_n(x)=0$ $\forall n\geq 1$ and $F(x)=0$, and we have trivial convergence of $F_n(x)$ to $F(x)$ as $n\rightarrow\infty$ at this point.
    Instead, let us take $x$ positive so that $m<x<m+1$ for some $m\in\mbb{N}$. Then, take $N\in\mbb{N}$ sufficiently large so that $x\notin\left[m-\tfrac{1}{n},m+\tfrac{1}{n}\right]\cup\left[(m+1)-\tfrac{1}{n},(m+1)+\tfrac{1}{n}\right]$ and $n>x$ whenever $n\geq N$. Taking $n\geq N$, we find
    \begin{align*}
        F_n(x)&=\int_{-\infty}^x\sum_{k=1}^n\frac{a_k}{\sum_{j=1}^n a_j}\frac{n}{2}\1{\left[k-\tfrac{1}{n},k+\tfrac{1}{n}\right]}(t)dt\\
        &=\sum_{k=1}^n\frac{a_k}{\sum_{j=1}^na_j}\frac{n}{2}\int_{k-\tfrac{1}{n}}^{k+\tfrac{1}{n}}\1{(-\infty,x]}(t)dt\\
        &=\sum_{k=1}^m\frac{a_k}{\sum_{j=1}^n a_j}\frac{n}{2}\frac{2}{n}\tag{3}\\
        &=\frac{\sum_{k=1}^ma_k}{\sum_{j=1}^na_j}
    \end{align*}
    passing this expression to the limit:
    \begin{align*}
        \lim_{n\rightarrow\infty}F_n(x)=\lim_{n\rightarrow\infty}\frac{\sum_{k=1}^ma_k}{\sum_{j=1}^na_j}=\frac{\sum_{k=1}^ma_k}{\sum_{j=1}^\infty a_j}=\sum_{k\in\mbb{N}:k<x}a_k=F(x)
    \end{align*}
    so we have that $F_n\rightarrow F$ pointwise on $\mbb{R}\setminus\mbb{N}$ (where $\mbb{N}$ are the set of discontinuities of $F$) as $n\rightarrow\infty$. To see why (3) holds, observe that when our conditions on $x$ and $n$ hold we have:
    \begin{align*}
        \int_{k-\tfrac{1}{n}}^{k+\tfrac{1}{n}}\1{(-\infty,x]}(t)dt=\begin{cases}
            \frac{2}{n}\quad&\text{if $\left[k-\tfrac{1}{n},k+\tfrac{1}{n}\right]\subset(-\infty,x]\;\Leftrightarrow\;k<x$}\\
            0\quad&\text{if $k\geq x$.}
        \end{cases}
    \end{align*}
    Of course, taking $n\geq N$ in our computations is of no consequence since we are interested in the limiting behavior of $F_n$. With all of this, and by the Helly-Bray theorem, we have that
    $P_n\Longrightarrow P$ as $n\rightarrow\infty$.\hfill{$\qed$}\\[5pt]
\end{document}
