\documentclass[10pt]{article}
\usepackage[margin=1.3cm]{geometry}

% Packages
\usepackage{amsmath, amsfonts, amssymb, amsthm}
\usepackage{bbm} 
\usepackage{dutchcal} % [dutchcal, calrsfs, pzzcal] calligraphic fonts
\usepackage{graphicx}
\usepackage[T1]{fontenc}
\usepackage[tracking]{microtype}

% Palatino for text goes well with Euler
\usepackage[sc,osf]{mathpazo}   % With old-style figures and real smallcaps.
\linespread{1.025}              % Palatino leads a little more leading

% Euler for math and numbers
\usepackage[euler-digits,small]{eulervm}

% Command initialization
\DeclareMathAlphabet{\pazocal}{OMS}{zplm}{m}{n}
\graphicspath{{./images/}}

% Custom Commands
\newcommand{\bs}[1]{\boldsymbol{#1}}
\newcommand{\E}{\mathbb{E}}
\newcommand{\var}[1]{\text{Var}\left(#1\right)}
\newcommand{\bp}[1]{\left({#1}\right)}
\newcommand{\mbb}[1]{\mathbb{#1}}
\newcommand{\1}[1]{\mathbbm{1}_{#1}}
\newcommand{\mc}[1]{\mathcal{#1}}
\newcommand{\nck}[2]{{#1\choose#2}}
\newcommand{\pc}[1]{\pazocal{#1}}
\newcommand{\ra}[1]{\renewcommand{\arraystretch}{#1}}
\newcommand*{\floor}[1]{\left\lfloor#1\right\rfloor}
\newcommand*{\ceil}[1]{\left\lceil#1\right\rceil}

\DeclareMathOperator{\Var}{Var}
\DeclareMathOperator{\Cov}{Cov}
\DeclareMathOperator{\diag}{diag}
\DeclareMathOperator{\as}{a.s.}
\DeclareMathOperator{\ale}{a.e.}
\DeclareMathOperator{\st}{s.t.}
\DeclareMathOperator{\io}{i.o.}
\DeclareMathOperator{\wip}{w.p.}
\DeclareMathOperator{\iid}{i.i.d.}
\DeclareMathOperator{\ifff}{if\;and\;only\;if}
\DeclareMathOperator{\inv}{inv}

\newtheorem{theorem}{Theorem}
\newtheorem{lemma}{Lemma}

\begin{document}
    \begin{center}
        {\bf\large{MATH 895: CORE COURSE IN PROBABILITY}}
        \smallskip
        \hrule
        \smallskip
        {\bf Assignment} 3\hfill {\bf Connor Braun} \hfill {\bf 2024-03-22}
    \end{center}
    {\bf Problem 2}\\[5pt]
    Let the real-valued random variable $X$ have density $f$. Let $\alpha,\beta\in\mbb{R}$ with $\alpha\neq 0$. Suppose that $Y=\alpha X+\beta$. Find the density of $Y$.\\[5pt]
    {\bf Solution}\hspace{5pt} Let $\lambda$ be the Lebesgue measure on $\mc{B}(\mbb{R})$. Then we proceed by finding the Radon-Nikodym derivative of $\mc{L}(Y)$. For this, take $A\in\mc{B}(\mbb{R})$. Then,
    \begin{align*}
        \mc{L}(Y)(A)=P(Y^{-1}(A))=P((\alpha X+\beta)^{-1}(A))=P\bp{X^{-1}\bp{\frac{A-\beta}{\alpha}}}=\mc{L}(X)\bp{\frac{A-\beta}{\alpha}}
    \end{align*}
    where we define $\tfrac{A-\beta}{\alpha}=\{x\in\mbb{R}:\alpha x+\beta\in A\}$ and used $\{\alpha X+\beta\in A\}=\{X\in\tfrac{A-\beta}{\alpha}\}$ to justify the second-to-last equality. With this, we have
    \begin{align*}
        \mc{L}(Y)(A)=\mc{L}(X)\bp{\frac{A-\beta}{\alpha}}&=\int_{\frac{A-\beta}{\alpha}}f(x)\lambda(dx)\\
        &=\int_A\frac{1}{\alpha}f\bp{\frac{t-\beta}{\alpha}}\lambda(dt)
    \end{align*}
    having applied the change of variables $t=\alpha x+\beta$ so that $x\in\tfrac{A-\beta}{\alpha}$ implies that $t\in A$. Of course, if $A$ were to be such that $\lambda(A)=0$, then by the above expression $\mc{L}(Y)(A)=0$ too, so $L(Y)\ll\lambda$ and $\tfrac{d\mc{L}(Y)}{d\lambda}(x)=\frac{1}{\alpha}f\bp{\tfrac{x-\beta}{\alpha}}$ is
    the density of $Y$.\hfill{$\qed$}\\[5pt]
\end{document}
