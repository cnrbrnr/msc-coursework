\documentclass[10pt]{article}
\usepackage[margin=1.3cm]{geometry}

% Packages
\usepackage{amsmath, amsfonts, amssymb, amsthm}
\usepackage{bbm} 
\usepackage{dutchcal} % [dutchcal, calrsfs, pzzcal] calligraphic fonts
\usepackage{graphicx}
\usepackage[T1]{fontenc}
\usepackage[tracking]{microtype}

% Palatino for text goes well with Euler
\usepackage[sc,osf]{mathpazo}   % With old-style figures and real smallcaps.
\linespread{1.025}              % Palatino leads a little more leading

% Euler for math and numbers
\usepackage[euler-digits,small]{eulervm}

% Command initialization
\DeclareMathAlphabet{\pazocal}{OMS}{zplm}{m}{n}
\graphicspath{{./images/}}

% Custom Commands
\newcommand{\bs}[1]{\boldsymbol{#1}}
\newcommand{\E}{\mathbb{E}}
\newcommand{\var}[1]{\text{Var}\left(#1\right)}
\newcommand{\bp}[1]{\left({#1}\right)}
\newcommand{\mbb}[1]{\mathbb{#1}}
\newcommand{\1}[1]{\mathbbm{1}_{#1}}
\newcommand{\mc}[1]{\mathcal{#1}}
\newcommand{\nck}[2]{{#1\choose#2}}
\newcommand{\pc}[1]{\pazocal{#1}}
\newcommand{\ra}[1]{\renewcommand{\arraystretch}{#1}}
\newcommand*{\floor}[1]{\left\lfloor#1\right\rfloor}
\newcommand*{\ceil}[1]{\left\lceil#1\right\rceil}

\DeclareMathOperator{\Var}{Var}
\DeclareMathOperator{\Cov}{Cov}
\DeclareMathOperator{\diag}{diag}
\DeclareMathOperator{\as}{a.s.}
\DeclareMathOperator{\ale}{a.e.}
\DeclareMathOperator{\st}{s.t.}
\DeclareMathOperator{\io}{i.o.}
\DeclareMathOperator{\wip}{w.p.}
\DeclareMathOperator{\iid}{i.i.d.}
\DeclareMathOperator{\ifff}{if\;and\;only\;if}
\DeclareMathOperator{\inv}{inv}

\newtheorem{theorem}{Theorem}
\newtheorem{lemma}{Lemma}

\begin{document}
    \begin{center}
        {\bf\large{MATH 895: CORE COURSE IN PROBABILITY}}
        \smallskip
        \hrule
        \smallskip
        {\bf Assignment} 3\hfill {\bf Connor Braun} \hfill {\bf 2024-03-22}
    \end{center}
    \noindent{\bf Problem 3}\\[5pt]
    Let $X$ be a real-valued random variable whose distribution function $F$ is continuous. Find the distribution function of $Y=F(X)$.\\[5pt]
    {\bf Solution}\hspace{5pt} With $x<0$, we have $P(Y\leq x)=P(F(X)\leq x)=0$, since $F(t)\geq 0$ $\forall t\in\mbb{R}$. Similarly, when $1\leq x$, $P(Y\leq x)=P(F(X)\leq x)=1$.
    More interesting is the case where $0\leq x<1$. For this, let $\mc{T}_x=\{t\in\mbb{R}:F(t)\leq x\}$ and $s_x=\sup\mc{T}_x$. Take $(t_n)_{n\geq 1}\subset\mc{T}_x$ with $t_n\rightarrow s_x$ as $n\rightarrow\infty$. Then
    \[F(t_n)\leq x\quad\forall n\geq 1\quad\Rightarrow\quad\lim_{n\rightarrow\infty}F(t_n)=F(s_x)\leq x\tag{1}\]
    since $F$ is continuous. That is, $s_x\in\mc{T}_x$. Further, if $t<s_x$, then $F(t)\leq F(s_x)\leq x$ since $F$ is monotonically increasing, so $t\in\mc{T}_x$. Thus, we have
    $\mc{T}_x=(-\infty,s_x]$. This allows us to write
    \[\{\omega\in\Omega:F(X(\omega))\leq x\}=\{\omega\in\Omega:X(\omega)\in\mc{T}_x\}=\{\omega\in\Omega:X(\omega)\in(-\infty,s_x]\}.\tag{2}\]
    Now set $\{x_n\}_{n\geq 1}$ so that $x_n\searrow s_x$ and $\{x_n\}_{n\geq 1}\subset\mbb{R}\setminus\mc{T}_x$. We have
    \[F(x_n)\geq x\quad\forall n\geq 1\quad\Rightarrow\quad\lim_{n\rightarrow\infty}F(x_n)=F(s_x)\geq x\]
    since $F$ continuous. With this and (1), we obtain $F(s_x)=x$ and so
    \begin{align*}
        P(Y\leq x)=P(F(X)\leq x)&=P(X\leq s_x)\tag{by (2)}\\
        &=F(s_x)\\
        &=x.
    \end{align*}
    With this, we may write the distribution function of $Y$: 
    \begin{align*}
        P(Y\leq x)=\begin{cases}
            0\quad&\text{if $x<0$}\\
            x\quad&\text{if $0\leq x<1$}\\
            1\quad&\text{if $1\leq x$}
        \end{cases}
    \end{align*}
    which is precisely the distribution of a standard uniform random variable.\hfill{$\qed$}\\[5pt]
\end{document}
