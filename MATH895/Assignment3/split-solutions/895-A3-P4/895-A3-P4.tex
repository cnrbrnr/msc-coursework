\documentclass[10pt]{article}
\usepackage[margin=1.3cm]{geometry}

% Packages
\usepackage{amsmath, amsfonts, amssymb, amsthm}
\usepackage{bbm} 
\usepackage{dutchcal} % [dutchcal, calrsfs, pzzcal] calligraphic fonts
\usepackage{graphicx}
\usepackage[T1]{fontenc}
\usepackage[tracking]{microtype}

% Palatino for text goes well with Euler
\usepackage[sc,osf]{mathpazo}   % With old-style figures and real smallcaps.
\linespread{1.025}              % Palatino leads a little more leading

% Euler for math and numbers
\usepackage[euler-digits,small]{eulervm}

% Command initialization
\DeclareMathAlphabet{\pazocal}{OMS}{zplm}{m}{n}
\graphicspath{{./images/}}

% Custom Commands
\newcommand{\bs}[1]{\boldsymbol{#1}}
\newcommand{\E}{\mathbb{E}}
\newcommand{\var}[1]{\text{Var}\left(#1\right)}
\newcommand{\bp}[1]{\left({#1}\right)}
\newcommand{\mbb}[1]{\mathbb{#1}}
\newcommand{\1}[1]{\mathbbm{1}_{#1}}
\newcommand{\mc}[1]{\mathcal{#1}}
\newcommand{\nck}[2]{{#1\choose#2}}
\newcommand{\pc}[1]{\pazocal{#1}}
\newcommand{\ra}[1]{\renewcommand{\arraystretch}{#1}}
\newcommand*{\floor}[1]{\left\lfloor#1\right\rfloor}
\newcommand*{\ceil}[1]{\left\lceil#1\right\rceil}

\DeclareMathOperator{\Var}{Var}
\DeclareMathOperator{\Cov}{Cov}
\DeclareMathOperator{\diag}{diag}
\DeclareMathOperator{\as}{a.s.}
\DeclareMathOperator{\ale}{a.e.}
\DeclareMathOperator{\st}{s.t.}
\DeclareMathOperator{\io}{i.o.}
\DeclareMathOperator{\wip}{w.p.}
\DeclareMathOperator{\iid}{i.i.d.}
\DeclareMathOperator{\ifff}{if\;and\;only\;if}
\DeclareMathOperator{\inv}{inv}

\newtheorem{theorem}{Theorem}
\newtheorem{lemma}{Lemma}

\begin{document}
    \begin{center}
        {\bf\large{MATH 895: CORE COURSE IN PROBABILITY}}
        \smallskip
        \hrule
        \smallskip
        {\bf Assignment} 3\hfill {\bf Connor Braun} \hfill {\bf 2024-03-22}
    \end{center}
    \noindent{\bf Problem 4}\\[5pt]
    Let $\alpha,\beta>0$. Recall that a real-valued random variable $X$ with density
    \begin{align*}
        f_{\alpha,\beta}(x)=\begin{cases}
            \frac{\beta^\alpha}{\Gamma(\alpha)}x^{\alpha-1}e^{\beta x}\quad&\text{if $x\geq 0$,}\\
            0\quad&\text{if $x<0$.}
        \end{cases}
    \end{align*}
    is called a gamma random variable with shape $\alpha$ and rate $\beta$. Let $\alpha_1,\alpha_2,\beta>0$ and suppose that $X_1,X_2$ are independent gamma random variables,
    where $X_i$ has shape $\alpha_i$ and rate $\beta$ for $i=1,2$. Show that $X_1+X_2$ is a gamma random variable with rate $\alpha_1+\alpha_2$ and rate $\beta$.\\[5pt]
    {\bf Proof}\hspace{5pt} We may compute the density of $\mc{L}(X_1+X_2)=\mc{L}(X_1)\ast\mc{L}(X_2)$ directly to verify the claim. Let $h$ be the density of $\mc{L}(X_1+X_2)$. Take $x\in\mbb{R}$
    \begin{align*}
        h(x)=\int_{\mbb{R}}f_{\alpha_1,\beta}(x-y)f_{\alpha_2,\beta}(y)dy
        &=\int_0^x\frac{\beta^{\alpha_1+\alpha_2}}{\Gamma(\alpha_1)\Gamma(\alpha_2)}(x-y)^{\alpha_1-1}e^{-\beta(x-y)}y^{\alpha_2-1}e^{-\beta y}dy
    \end{align*}
    where we restrict the region of integration since $f_{\alpha_1,\beta}(x-y)f_{\alpha_2,\beta}(y)\geq 0\;\Leftrightarrow\;x\geq y\geq 0$, so we further require $x\geq 0$. Continuing,
    \begin{align*}
        h(x)&=\int_0^x\frac{\beta^{\alpha_1+\alpha_2}}{\Gamma(\alpha_1)\Gamma(\alpha_2)}(x-y)^{\alpha_1-1}e^{-\beta(x-y)}y^{\alpha_2-1}e^{-\beta y}dy
        =\frac{\beta^{\alpha_1+\alpha_2}}{\Gamma(\alpha_1)\Gamma(\alpha_2)}e^{-\beta x}\int_0^x(x-y)^{\alpha_1-1}y^{\alpha_2-1}dy\\
    \end{align*}
    and now focusing on the computation of the integral:
    \begin{align*}
        \int_0^x(x-y)^{\alpha_1-1}y^{\alpha_2-1}dy&=\int_0^1(x-tx)^{\alpha_1-1}(tx)^{\alpha_2-1}xdt\tag{substituting $y=tx$}\\
        &=x^{\alpha_1+\alpha_2-1}\int_0^1(1-t)^{\alpha_1-1}t^{\alpha_2-1}dt\\
        &=x^{\alpha_1+\alpha_2-1}B(\alpha_1,\alpha_2)
    \end{align*}
    where for $z,w>0$ we define $B(z,w):=\int_0^1(1-t)^{z-1}t^{w-1}dt$ the beta function which satisfies the identity [1]
    \[B(z,w)=\frac{\Gamma(z)\Gamma(w)}{\Gamma(z+w)}\] 
    so that our density function can now be written
    \begin{align*}
        h(x)=\frac{\beta^{\alpha_1+\alpha_2}}{\Gamma(\alpha_1)\Gamma(\alpha_2)}\frac{\Gamma(\alpha_1)\Gamma(\alpha_2)}{\Gamma(\alpha_1+\alpha_2)}x^{\alpha_1+\alpha_2-1}e^{-\beta x}=\frac{\beta^{\alpha_1+\alpha_2}}{\Gamma(\alpha_1+\alpha_2)}x^{\alpha_1+\alpha_2-1}e^{-\beta x}=f_{\alpha_1+\alpha_2,\beta}(x).
    \end{align*}
    Of course, when $x<0$, the original integral vanishes and $h(x)=0$. Thus, the density of $\mc{L}(X_1+X_2)$ is the gamma density with shape $\alpha_1+\alpha_2$ and rate $\beta$, as claimed.\hfill{$\qed$}\\[5pt]
\end{document}
