\documentclass[10pt]{article}
\usepackage[margin=1.3cm]{geometry}

% Packages
\usepackage{amsmath, amsfonts, amssymb, amsthm}
\usepackage{bbm} 
\usepackage{dutchcal} % [dutchcal, calrsfs, pzzcal] calligraphic fonts
\usepackage{graphicx}
\usepackage[T1]{fontenc}
\usepackage[tracking]{microtype}

% Palatino for text goes well with Euler
\usepackage[sc,osf]{mathpazo}   % With old-style figures and real smallcaps.
\linespread{1.025}              % Palatino leads a little more leading

% Euler for math and numbers
\usepackage[euler-digits,small]{eulervm}

% Command initialization
\DeclareMathAlphabet{\pazocal}{OMS}{zplm}{m}{n}
\graphicspath{{./images/}}

% Custom Commands
\newcommand{\bs}[1]{\boldsymbol{#1}}
\newcommand{\E}{\mathbb{E}}
\newcommand{\var}[1]{\text{Var}\left(#1\right)}
\newcommand{\bp}[1]{\left({#1}\right)}
\newcommand{\mbb}[1]{\mathbb{#1}}
\newcommand{\1}[1]{\mathbbm{1}_{#1}}
\newcommand{\mc}[1]{\mathcal{#1}}
\newcommand{\nck}[2]{{#1\choose#2}}
\newcommand{\pc}[1]{\pazocal{#1}}
\newcommand{\ra}[1]{\renewcommand{\arraystretch}{#1}}
\newcommand*{\floor}[1]{\left\lfloor#1\right\rfloor}
\newcommand*{\ceil}[1]{\left\lceil#1\right\rceil}

\DeclareMathOperator{\Var}{Var}
\DeclareMathOperator{\Cov}{Cov}
\DeclareMathOperator{\diag}{diag}
\DeclareMathOperator{\as}{a.s.}
\DeclareMathOperator{\ale}{a.e.}
\DeclareMathOperator{\st}{s.t.}
\DeclareMathOperator{\io}{i.o.}
\DeclareMathOperator{\wip}{w.p.}
\DeclareMathOperator{\iid}{i.i.d.}
\DeclareMathOperator{\ifff}{if\;and\;only\;if}
\DeclareMathOperator{\inv}{inv}

\newtheorem{theorem}{Theorem}
\newtheorem{lemma}{Lemma}

\begin{document}
    \begin{center}
        {\bf\large{MATH 895: CORE COURSE IN PROBABILITY}}
        \smallskip
        \hrule
        \smallskip
        {\bf Assignment} 3\hfill {\bf Connor Braun} \hfill {\bf 2024-03-22}
    \end{center}
    \noindent{\bf Problem 6}\\[5pt]
    For each $n\geq 1$, let $P_n$ be a probability measure on $\mc{B}(\mbb{R})$ with density $f_n(x)=(1+\sin(2\pi nx))\1{[0,1]}(x)$, $x\in\mbb{R}$. Let $P$ be a probability measure on $\mc{B}(\mbb{R})$ with density $\1{[0,1]}(x)$, $x\in\mbb{R}$. Prove or disprove: $P_n\Longrightarrow P$ as $n\rightarrow\infty$.\\[5pt]
    The statement is true.\\[5pt]
    {\bf Proof}\hspace{5pt} Pointwise convergence of characteristic functions is equivalent to weak convergence of the corresponding measures. Thus, we will show that the sequence of characteristic functions $\{\varphi_n\}_{n\geq 1}$ corresponding to the laws $\{P_n\}_{n\geq 1}$ converge pointwise on $\mbb{R}$ to $\varphi$, the characteristic function of $P$.\\[5pt]
    For $n\geq 1$, we find that
    \begin{align*}
        \varphi_n(t)=\int_\mbb{R}e^{itx}P_n(dx)=\int_\mbb{R}e^{itx}f_n(x)dx&=\int_0^1e^{itx}(1+\sin(2\pi nx))dx\\
        &=\int_0^1e^{itx}+(\cos(tx)+i\sin(tx))\sin(2\pi nx)dx\\
        &=\int_\mbb{R}e^{itx}\1{[0,1]}(x)dx+\int_0^1\cos(tx)\sin(2\pi nx)dx+i\int_0^1\sin(tx)\sin(2\pi nx)dx.
    \end{align*}
    The first term here is just $\varphi(t)$, so we need only compute the second two integrals. Using trigonometric product-sum identities [1], we compute
    \begin{align*}
        \varphi_n(t)-\varphi(t)&=\int_0^1\cos(tx)\sin(2\pi nx)dx+i\int_0^1\sin(tx)\sin(2\pi nx)dx\\
        &=\frac{1}{2}\int_0^1\sin((2\pi n+t)x)+\sin((2\pi n-t)x)dx+\frac{i}{2}\int_0^1\cos((2\pi n-t)x)-\cos((2\pi n+t)x)dx\\
        &=\frac{-\cos((2\pi n+t)x)-i\sin((2\pi n+t)x)}{4\pi n+2t}\bigg|_0^1+\frac{-\cos((2\pi n-t)x)+i\sin((2\pi n -t)x)}{4\pi n-2t}\bigg|_0^1\\
        &=\frac{-\cos(2\pi n+t)-i\sin(2\pi n+t)+1}{4\pi n+2t}+\frac{-\cos(2\pi n-t)+i\sin(2\pi n -t)+1}{4\pi n-2t}\\
        &=\frac{-\cos(t)-i\sin(t)+1}{4\pi n+2t}+\frac{-\cos(-t)+i\sin(-t)+1}{4\pi n-2t}
    \end{align*}
    since $\sin$ and $\cos$ are $2\pi$-periodic. Thus, for any $t\in\mbb{R}$ we have
    \begin{align*}
        \lim_{n\rightarrow\infty}\varphi_n(t)-\varphi(t)=\lim_{n\rightarrow\infty}\bp{\frac{-\cos(t)-i\sin(t)+1}{4\pi n+2t}+\frac{-\cos(-t)+i\sin(-t)+1}{4\pi n-2t}}=0
    \end{align*}
    so indeed, $\varphi_n\rightarrow\varphi$ pointwise on $\mbb{R}$ as $n\rightarrow\infty$, and we conclude that $P_n\Longrightarrow P$ as $n\rightarrow\infty$, as claimed.\hfill{$\qed$\\[5pt]}
\end{document}
