\documentclass[10pt]{article}
\usepackage[margin=1.3cm]{geometry}

% Packages
\usepackage{amsmath, amsfonts, amssymb, amsthm}
\usepackage{bbm} 
\usepackage{dutchcal} % [dutchcal, calrsfs, pzzcal] calligraphic fonts
\usepackage{graphicx}
\usepackage[T1]{fontenc}
\usepackage[tracking]{microtype}

% Palatino for text goes well with Euler
\usepackage[sc,osf]{mathpazo}   % With old-style figures and real smallcaps.
\linespread{1.025}              % Palatino leads a little more leading

% Euler for math and numbers
\usepackage[euler-digits,small]{eulervm}

% Command initialization
\DeclareMathAlphabet{\pazocal}{OMS}{zplm}{m}{n}
\graphicspath{{./images/}}

% Custom Commands
\newcommand{\bs}[1]{\boldsymbol{#1}}
\newcommand{\E}{\mathbb{E}}
\newcommand{\var}[1]{\text{Var}\left(#1\right)}
\newcommand{\bp}[1]{\left({#1}\right)}
\newcommand{\mbb}[1]{\mathbb{#1}}
\newcommand{\1}[1]{\mathbbm{1}_{#1}}
\newcommand{\mc}[1]{\mathcal{#1}}
\newcommand{\nck}[2]{{#1\choose#2}}
\newcommand{\pc}[1]{\pazocal{#1}}
\newcommand{\ra}[1]{\renewcommand{\arraystretch}{#1}}
\newcommand*{\floor}[1]{\left\lfloor#1\right\rfloor}
\newcommand*{\ceil}[1]{\left\lceil#1\right\rceil}

\DeclareMathOperator{\Var}{Var}
\DeclareMathOperator{\Cov}{Cov}
\DeclareMathOperator{\diag}{diag}
\DeclareMathOperator{\as}{a.s.}
\DeclareMathOperator{\ale}{a.e.}
\DeclareMathOperator{\st}{s.t.}
\DeclareMathOperator{\io}{i.o.}
\DeclareMathOperator{\wip}{w.p.}
\DeclareMathOperator{\iid}{i.i.d.}
\DeclareMathOperator{\ifff}{if\;and\;only\;if}
\DeclareMathOperator{\inv}{inv}

\newtheorem{theorem}{Theorem}
\newtheorem{lemma}{Lemma}

\begin{document}
    \begin{center}
        {\bf\large{MATH 895: CORE COURSE IN PROBABILITY}}
        \smallskip
        \hrule
        \smallskip
        {\bf Assignment} 3\hfill {\bf Connor Braun} \hfill {\bf 2024-03-22}
    \end{center}
    \noindent{\bf Problem 9}\\[5pt]
    Let $\varphi_Z(t)=\E{(e^{itZ})}$ denote the characteristic function of the random variable $Z$. Prove the following.\\[5pt]
    {\bf Theorem}\hspace{5pt} Let $X$, $Y$ be two independent, real-valued random variables. Then for every $t\in\mbb{R}$, we have $\varphi_{X+Y}(t)=\varphi_X(t)\varphi_Y(t)$.\\[5pt]
    {\bf Proof}\hspace{5pt} Let $t\in\mbb{R}$. Then
    \begin{align*}
        \varphi_{X+Y}(t)=\E{(e^{it(X+Y)})}=\E{(e^{itX}e^{itY})}=\E{(e^{itX})}\E{(e^{itY})}=\varphi_X(t)\varphi_Y(t)
    \end{align*}
    where the penultimate equality holds due to the independence of $X$ and $Y$.\hfill{$\qed$}\\[5pt]
\end{document}
