\documentclass[10pt]{article}
\usepackage[margin=1.3cm]{geometry}

% Packages
\usepackage{amsmath, amsfonts, amssymb, amsthm}
\usepackage{bbm} 
\usepackage{dutchcal} % [dutchcal, calrsfs, pzzcal] calligraphic fonts
\usepackage{graphicx}
\usepackage[T1]{fontenc}
\usepackage[tracking]{microtype}

% Palatino for text goes well with Euler
\usepackage[sc,osf]{mathpazo}   % With old-style figures and real smallcaps.
\linespread{1.025}              % Palatino leads a little more leading

% Euler for math and numbers
\usepackage[euler-digits,small]{eulervm}

% Command initialization
\DeclareMathAlphabet{\pazocal}{OMS}{zplm}{m}{n}
\graphicspath{{./images/}}

% Custom Commands
\newcommand{\bs}[1]{\boldsymbol{#1}}
\newcommand{\E}{\mathbb{E}}
\newcommand{\var}[1]{\text{Var}\left(#1\right)}
\newcommand{\bp}[1]{\left({#1}\right)}
\newcommand{\mbb}[1]{\mathbb{#1}}
\newcommand{\1}[1]{\mathbbm{1}_{#1}}
\newcommand{\mc}[1]{\mathcal{#1}}
\newcommand{\nck}[2]{{#1\choose#2}}
\newcommand{\pc}[1]{\pazocal{#1}}
\newcommand{\ra}[1]{\renewcommand{\arraystretch}{#1}}
\newcommand*{\floor}[1]{\left\lfloor#1\right\rfloor}
\newcommand*{\ceil}[1]{\left\lceil#1\right\rceil}

\DeclareMathOperator{\Var}{Var}
\DeclareMathOperator{\Cov}{Cov}
\DeclareMathOperator{\diag}{diag}

\newtheorem{theorem}{Theorem}
\newtheorem{lemma}{Lemma}

\begin{document}

    \begin{center}
        {\bf\large{MATH 895: CORE COURSE IN PROBABILITY}}
        \smallskip
        \hrule
        \smallskip
        {\bf Assignment} 1\hfill {\bf Connor Braun} \hfill {\bf 2024-01-25}
    \end{center}
    \begin{center}
        \begin{minipage}{\dimexpr\paperwidth-10cm}
            Some solutions presented here were, at least in part, the product of collaboration with my fellow students and
            Professor Cellarosi himself. To be more precise, problems 1, 3, 5, and 7 incorporate ideas presented to me during discussion with Osman Bicer (1b)
            Anthony Pasion (problem 7) and Timothy Liu (problems 3 and 5). Professor Cellarosi contributed significantly to solutions for problems 2, 3 and 5.
            Problems 7-10 were completed with occasional reference to [1].
        \end{minipage}
    \end{center}
    \vspace{5pt}
    \noindent{\bf Problem 1}\\[5pt]
    {\bf a)} Let $(\Omega, \mc{F},P)$ be a probability space, and let $A_1,\dots, A_n\in\mc{F}$ be $n\geq 1$ events.
    Prove the so-called {\it inclusion-exclusion formula}
    \[P\left(\bigcup_{i=1}^nA_i\right)=\sum_{k=1}^n(-1)^{k+1}\sum_{1\leq i_1<\dots i_k\leq n}P(A_{i_1}\cap\cdots\cap A_{i_k}).\tag{1}\]
    {\bf Proof}\hspace{5pt} We proceed by induction. First, consider the basis where we fix $n=1$. Then, trivially,
    \[P\left(\bigcup_{i=1}^n A_i\right)=P(A_1)=\sum_{k=1}^1(-1)^{k+1}P(A_1)=\sum_{k=1}^n(-1)^{k+1}\sum_{1\leq i_1<\dots <i_k\leq n}P(A_{i_1}\cap\cdots\cap A_{i_k}).\]
    Now supposing that (1) holds for any subcollection of $\mc{F}$ of size $n\geq 1$, we aim to show that for $A_1,\dots,A_{n+1}\in\mc{F}$
    \[P\left(\bigcup_{i=1}^{n+1}A_i\right)=\sum_{k=1}^{n+1}(-1)^{k+1}\sum_{1\leq i_1<\dots i_k\leq n+1}P(A_{i_1}\cap\cdots\cap A_{i_k}).\tag{2}\]
    For notational compactness, take $A=\cup_{i=1}^nA_i$ and $B=A_{n+1}$, each an element of $\mc{F}$. Then, we can rewrite the probability of interest
    \begin{align*}
        P\left(\cup_{i=1}^{n+1}A_i\right)=P(A\cup B)=P(A)+P(B\setminus(A\cap B))&=P(A)+P(B\setminus (A\cap B))+P(A\cap B)-P(A\cap B)\\
        &=P(A)+P((B\setminus(A\cap B))\cup (A\cap B))-P(A\cap B)\\
        &=P(A)+P(B)-P(A\cap B)
    \end{align*}
    yielding an expression amenable to the inductive hypothesis:
    \begin{align*}
        P\bp{\bigcup_{i=1}^{n+1}A_i}&=P\bp{\bigcup_{i=1}^nA_i}+P(A_{n+1})-P\bp{\bigcup_{i=1}^n(A_i\cap A_{n+1})}\\
        &=P(A_{n+1})+\sum_{k=1}^n(-1)^{k+1}\sum_{1\leq i_1<\dots <i_k\leq n}P(A_{i_1}\cap\cdots A_{i_k})
        \quad+\sum_{k=1}^n(-1)^{k+2}\sum_{1\leq i_1<\dots<i_k\leq n}P(A_{i_1}\cap\dots A_{i_k}\cap A_{n+1})\tag{3}
    \end{align*}
    since both $\cup_{i=1}^nA_i$ and $\cup_{i=1}^n(A_i\cap A_{n+1})$ are $n$-unions over elements of $\mc{F}$. Massaging this expression, we find that
    \begin{align*}
        P\bp{\bigcup_{i=1}^{n+1}A_i}&=\sum_{k=1}^{n+1}P(A_k)+\sum_{k=2}^n(-1)^{k+1}\sum_{1\leq i_1<\dots<i_k\leq n}P(A_{i_1}\cap\cdots\cap A_{i_k})+\sum_{k=2}^{n+1}(-1)^{k+1}\sum_{1\leq i_1<\dots<i_{k-1}\leq n}P(A_{i_1}\cap\cdots\cap A_{i_{k-1}}\cap A_{n+1})
    \end{align*}
    where the second term contains all intersections of $2$ or more elements of $\{A_i\}_{i=1}^n$ (i.e., those not including $A_{n+1}$) and the third term contains all intersections of two or more elements of $\{A_i\}_{i=1}^{n+1}$, at least once of which is $A_{n+1}$.
    Combining these, we get the appropriate sum of probabilities over all intersections of two or more elements of $\{A_i\}_{i=1}^{n+1}$:
    \begin{align*}
        P\bp{\bigcup_{i=1}^{n+1}A_i}&=\sum_{k=1}^{n+1}P(A_{k})+\sum_{k=2}^{n+1}(-1)^{k+1}\sum_{1\leq i_1<\dots<i_k\leq n+1}P(A_{i_1}\cap\cdots\cap A_{i_k})\\
        &=\sum_{k=1}^{n+1}(-1)^{k+1}\sum_{1\leq i_1<\dots i_k\leq n+1}P(A_{i_1}\cap\cdots\cap A_{i_k}).
    \end{align*} 
    whereby the basis, inductive step, and principle of induction we have that (1) holds for any $n\geq 1$.\hfill{$\qed$}\\[5pt]
    {\bf b)} Let $n\geq 3$ be given, and consider $A_1,\dots,A_n$ in some probability space $(\Omega,\mc{F},P)$. We say that an event {\it occurs} if it has nonzero probability.
    Assume that
    \begin{itemize}
        \item at least one of the events occurs,
        \item no more than three events occur,
        \item the probability of at least two events occuring is $\frac{1}{2}$,
        \item $P(A_i)=p$, $P(A_i\cap A_j)=q$ and $P(A_i\cap A_j\cap A_k)=r$ for $1\leq i<j<k\leq n$.
    \end{itemize}
    Show that both $p\geq \frac{3}{2n}$ and $q\leq \frac{4}{n}$.\\[5pt]
    {\bf Proof}\hspace{5pt}\\[5pt]
    Since at least one of the events occur, we have (restricting the probability space to the union of the $\{A_i\}_{i=1}^n$)
    \[P\bp{\bigcup_{i=1}^nA_i}=1\]
    where from the inclusion-exclusion principle, we obtain
    \begin{align*}
        1=\sum_{i=1}^nP(A_i)-\sum_{1\leq i<j\leq n}P(A_i\cap A_j)+\sum_{1\leq i<j<k\leq n}P(A_i\cap A_j\cap A_k)=np-\nck{n}{2}q+\nck{n}{3}r.
    \end{align*}
    Now, letting $\mathcal{I}=\{(i,j): 1\leq i<j\leq n\}$, the probability that at least two events occur is $\frac{1}{2}$ so we can again apply the inclusion-exclusion principle to obtain
    \begin{align*}
        \frac{1}{2}&=P\bp{\bigcup_{1\leq i<j\leq n}(A_i\cap A_j)}\\
        &=\sum_{1\leq i<j\leq n}P(A_i\cap A_j)-\sum_{\substack{(i,j),(k,\ell)\in\mathcal{I} \\ (i,j)\neq (k,\ell)}}P(A_i\cap A_j\cap A_k\cap A_\ell)+\sum_{\substack{(i,j),(k,\ell),(s,t)\in\mathcal{I} \\ (i,j)\neq(k,\ell)\neq(s,t)}}P(A_i\cap A_j\cap A_k\cap A_\ell\cap A_s\cap A_t).
    \end{align*}
    Some care is needed to justify this latest expression, in particular the fact that we can truncate the expansion after the third sum. For this, simply observe that $\nck{2}{2}=1<\nck{3}{2}=3$, so any set of $\geq 4$ combinations of two (out of three) events cannot be pairwise distinct. Equivalently,
    taking $\geq 4$ pairwise distinct elements of $\mathcal{I}$ must specify at least four distinct events such that the probability of their intersection is zero.\\[5pt]
    Through a similar line of reasoning, observe that for any $(i,j),(k,\ell)\in\mathcal{I}$ with $(i,j)\neq(k,\ell)$, at least three of $A_i, A_j,A_k,A_\ell$ are pairwise distinct. Of course, we can restrict our attention to the case when exactly three are pairwise distinct,
    since the probability that greater than three events occur is zero. Now, fixing three sets $A_{i_1},A_{i_2},A_{i_3}\in\{A_i\}_{i=1}^n$, there are $\nck{3}{2}$ distinct elements of $\mathcal{I}$ specifying a pair of them. Then, there are $\nck{\nck{3}{2}}{2}=\nck{3}{2}=3$ ways of selecting two of these three elements, each
    of which specify the same intersection $A_{i_1}\cap A_{i_2}\cap A_{i_3}$.\\[5pt]
    Finally, for any $(i,j),(k,\ell),(s,t)\in\mathcal{I}$ with $(i,j)\neq(k,\ell)\neq (s,t)$, at least three of $A_i,A_j,A_k,A_\ell,A_s,A_t$ are pairwise distinct and, again,
    we restrict our consideration for the case when exactly three are pairwise distinct. With the same arbitrary $A_{i_1},A_{i_2},A_{i_3}$ as before, there is now $\nck{\nck{3}{2}}{3}=\nck{3}{3}=1$ way of selecting three of the three available pairwise distinct elements of $\mathcal{I}$ specifying a pair of them.
    Importantly, there were $\nck{n}{3}$ ways we could have fixed $A_{i_1},A_{i_2}$ and $A_{i_3}$.\\[5pt]
    With this cumbersome counting aside, the above plainly reduces to
    \begin{align*}
        \frac{1}{2}=\nck{n}{2}q-3\nck{n}{3}r+\nck{n}{3}r=\nck{n}{2}q-2\nck{n}{3}r\leq \nck{n}{2}q-\nck{n}{3}r
    \end{align*}
    which we can use to obtain the desired inequalities. Specifically, we will need the facts
    \[\frac{1}{2}\leq \nck{n}{2}q-\nck{n}{3}r\quad\text{and}\quad\nck{n}{3}r=\frac{1}{2}\nck{n}{2}q-\frac{1}{4}\]
    the latter of which is easily derived from the above result. Using the first, we can get
    \begin{align*}
        1=np-\nck{n}{2}q+\nck{n}{3}r\quad\Rightarrow\quad 1+\frac{1}{2}\leq np\quad\Rightarrow\quad p\geq \frac{3}{2n}
    \end{align*}
    as desired. We also know that $p\leq 1$, so that
    \begin{align*}
        1=np-\nck{n}{2}q+\nck{n}{3}r&\leq n-\nck{n}{2}q+\nck{n}{3}r+\frac{1}{2}\nck{n}{2}q-\frac{1}{4}\\
        &=n-\frac{1}{2}\nck{n}{2}q-\frac{1}{4}.
    \end{align*}
    Isolating for $q$ gives us
    \begin{align*}
        1\leq n-\frac{1}{2}\nck{n}{2}q-\frac{1}{4}\quad\Rightarrow\quad \nck{n}{2}q\leq 2n-\frac{5}{2}\quad\Rightarrow\quad &q\leq \frac{(4n-5)(n-2)!2!}{2n!}\\
        \Rightarrow\quad &q\leq \frac{4n-5}{n^2-n}=\frac{4n}{n^2}\frac{(1-\frac{5}{4n})}{(1-\frac{1}{n})}\leq \frac{4}{n}
    \end{align*}
    with the final inequality holding since $(1-\frac{5}{4n})\leq (1-\frac{1}{n})$. Thus we have shown that both $p\geq \frac{3}{2n}$ and $q\leq \frac{4}{n}$, as desired.\hfill{$\qed$}\\[5pt]
\end{document}