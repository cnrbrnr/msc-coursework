\documentclass[10pt]{article}
\usepackage[margin=1.3cm]{geometry}

% Packages
\usepackage{amsmath, amsfonts, amssymb, amsthm}
\usepackage{bbm} 
\usepackage{dutchcal} % [dutchcal, calrsfs, pzzcal] calligraphic fonts
\usepackage{graphicx}
\usepackage[T1]{fontenc}
\usepackage[tracking]{microtype}

% Palatino for text goes well with Euler
\usepackage[sc,osf]{mathpazo}   % With old-style figures and real smallcaps.
\linespread{1.025}              % Palatino leads a little more leading

% Euler for math and numbers
\usepackage[euler-digits,small]{eulervm}

% Command initialization
\DeclareMathAlphabet{\pazocal}{OMS}{zplm}{m}{n}
\graphicspath{{./images/}}

% Custom Commands
\newcommand{\bs}[1]{\boldsymbol{#1}}
\newcommand{\E}{\mathbb{E}}
\newcommand{\var}[1]{\text{Var}\left(#1\right)}
\newcommand{\bp}[1]{\left({#1}\right)}
\newcommand{\mbb}[1]{\mathbb{#1}}
\newcommand{\1}[1]{\mathbbm{1}_{#1}}
\newcommand{\mc}[1]{\mathcal{#1}}
\newcommand{\nck}[2]{{#1\choose#2}}
\newcommand{\pc}[1]{\pazocal{#1}}
\newcommand{\ra}[1]{\renewcommand{\arraystretch}{#1}}
\newcommand*{\floor}[1]{\left\lfloor#1\right\rfloor}
\newcommand*{\ceil}[1]{\left\lceil#1\right\rceil}

\DeclareMathOperator{\Var}{Var}
\DeclareMathOperator{\Cov}{Cov}
\DeclareMathOperator{\diag}{diag}

\newtheorem{theorem}{Theorem}
\newtheorem{lemma}{Lemma}

\begin{document}

    \begin{center}
        {\bf\large{MATH 895: CORE COURSE IN PROBABILITY}}
        \smallskip
        \hrule
        \smallskip
        {\bf Assignment} 1\hfill {\bf Connor Braun} \hfill {\bf 2024-01-25}
    \end{center}
    \noindent\begin{center}
        {\bf Prelude to Problems 7 through 10}
    \end{center}
    In 1928, A.N. Kolmogorov published the following theorem providing necessary and sufficient conditions for the Weak Law of Large Numbers.
    \begin{theorem}[Kolmogorov, 1928]
        Let $(X_k)_{k\geq 1}$ be a sequence of mutually independent, zero mean, real-valued random variables. Let $X_{k,n}=X_k\1{|X_k|<n}$. Then
        \[\frac{1}{n}\sum_{k=1}^nX_k\overunderset{P}{n\rightarrow\infty}{\longrightarrow} 0\]
        if and only if the following three conditions are all satisfied:
        \begin{enumerate}
            \item $\sum_{k=1}^nP(|X_k|\geq n)\rightarrow 0\quad\text{as}\quad n\rightarrow\infty$
            \item $\frac{1}{n}\sum_{k=1}^n\E(X_{k,n})\rightarrow 0\quad\text{as}\quad n\rightarrow\infty$
            \item $\frac{1}{n^2}\sum_{k=1}^n\Var(X_{k,n})\rightarrow 0\quad\text{as}\quad n\rightarrow\infty$.
        \end{enumerate}
    \end{theorem}
    \noindent In the same paper, Kolmogorov included, without a proof, a sharpened version of the above theorem where condition $3$ is replaced by
    \begin{enumerate}
        \setcounter{enumi}{3}
        \item $\frac{1}{n^2}\sum_{k=1}^n\E(X_{n,k}^2)\rightarrow 0\quad\text{as}\quad n\rightarrow\infty$.
    \end{enumerate}
    The improved 'theorem' was 'proved' by B.V. Gnedenko and A.N. Kolmogorov in 1949. However, the result is false tnd their proof contained a smitake. The goal of the following four problems is to show, assuming Kolmogorov's corrent theorem, that conditions 1,2,4 
    are not necessary for the Weak Law of Large Numbers to hold.\\[5pt]
    As a specific counterexample, consider $(X_k)_{k\geq 1}$ mutually independent random variables such that
    \begin{align*}
        &P(X_1=0)=1\\
        &P\bp{X_k=(-1)^kk^{5/2}}=\frac{1}{k^2},\quad P\bp{X_k=(-1)^{k+1}\frac{k^{1/2}}{1-k^{-2}}}=1-\frac{1}{k^2}\quad\text{for}\quad k\geq 2.\tag{16}
    \end{align*} 
    {\bf Problem 7}\\[5pt]
    Show that $(X_k)_{k\geq 1}$ as defined in (16) satisfies condition 1 of Kolmogorov's theorem.\\[5pt]
    {\bf Proof}\hspace{5pt} First, observe that if $k<n^{2/5}$ then $k^{5/2}<n$ and so
    \[0<\left|(-1)^{k+1}\frac{k^{1/2}}{1-k^{-2}}\right|=\frac{k^{5/2}}{k^2-1}<k^{5/2}=\left|(-1)^kk^{5/2}\right|<n.\tag{17}\]
    That is, $|X_k|>n$ with probability $0$ if $k\leq n^{2/5}$ since both of the values it can take are less than $n$ (and, of course, $X_1$ is always less than $n$). If instead $n^{2/5}\leq k\leq n$, then $n\leq k^{5/2}=|(-1)^kk^{5/2}|$, but still we have
    \begin{align*}
        \left|(-1)^{k+1}\frac{k^{1/2}}{1-k^{-2}}\right|=\frac{k^{1/2}}{1-k^{-2}}=\frac{k^{5/2}}{k^2-1}< k^{1/2}<k\leq n\tag{18}
    \end{align*}
    so that $|X_k|\geq n$ only when $X_k=(-1)^kk^{5/2}$, which occurs with probability $\tfrac{1}{k^2}$. Thus, we have
    \begin{align*}
        \sum_{k=1}^nP(|X_k|\geq n)=\sum_{k=\ceil{n^{2/5}}}^n\frac{1}{k^2}<\sum_{k=\ceil{n^{2/5}}}^\infty\frac{1}{k^2}=\sum_{k=1}^\infty\frac{1}{k^2}-\sum_{k=1}^{\ceil{n^{2/5}}-1}\frac{1}{k^2}\overset{n\rightarrow\infty}{\longrightarrow}\sum_{k=1}^\infty\frac{1}{k^2}-\sum_{k=1}^\infty\frac{1}{k^2}=0
    \end{align*}
    admissable, since $\sum_{k=1}^\infty\tfrac{1}{k^2}<\infty$.\hfill{$\qed$}\\[5pt]
\end{document}