\documentclass[11pt, letterpaper]{article}
\usepackage[margin=1.5cm]{geometry}
\pagestyle{plain}

\usepackage{amsmath, amsfonts, amssymb, amsthm}
\usepackage{bbm}
\usepackage{calrsfs}
\usepackage[shortlabels]{enumitem}
\usepackage[makeroom]{cancel}
\usepackage{graphicx}
\usepackage{xcolor}
\usepackage{array, booktabs, ragged2e}

\graphicspath{{./images/}}
\DeclareMathAlphabet{\pazocal}{OMS}{zplm}{m}{n}

\newcommand{\bs}[1]{\boldsymbol{#1}}
\newcommand{\mbb}[1]{\mathbb{#1}}
\newcommand{\bbm}[1]{\mathbbm{#1}}
\newcommand{\mc}[1]{\mathcal{#1}}
\newcommand{\pc}[1]{\pazocal{#1}}
\newcommand{\ra}[1]{\renewcommand{\arraystretch}{#1}}

\title{\bf Analysis I: Assignment III}
\author{\bf Connor Braun}
\date{}

\begin{document}
    \maketitle
    \noindent{\bf Problem 1} Let $(X,\mc{A},\mu)$ be a measure space and $A_k,$ $k\geq 1$ be a countable sequence of measurable sets.
    Prove that
    \[\mu\left(\bigcup_{k=1}^\infty A_k\right)=\lim_{n\rightarrow\infty}\mu\left(\bigcup_{k=1}^nA_k\right).\]
    {\bf Proof} For $n\geq 1$, define $B_n=\cup_{k=1}^nA_k$ and notice that 
    \[B_n=\cup_{k=1}^nA_k\subseteq\cup_{k=1}^{n+1}A_k=B_{n+1}\quad\text{so}\quad B_n\subseteq B_{n+1}\]
    and further,
    \[\cup_{j=1}^n B_j=\cup_{j=1}^n\cup_{k=1}^j A_k=\cup_{k=1}^n A_k=B_n.\tag{1}\]
    Define another sequence of sets $C_n=B_n\setminus B_{n-1}$ for $n\geq 2$ and $C_1=B_1$. Let $i,j\in\mbb{N}$ with $i\neq j$. Without
    loss of generality take $i<j$, and notice that
    \[C_j=B_j\setminus B_{j-1},\quad\text{and}\quad C_i\subseteq B_i\subseteq B_{j-1}\quad\text{so}\quad C_j\cap C_i=\emptyset.\]
    Further, we have
    \begin{align*}
        \cup_{j=1}^nC_j&=C_1\cup C_2\cup\dots\cup C_n\\
        &=B_1\cup(B_2\setminus B_1)\cup\dots\cup(B_{n-1}\setminus B_{n-2})\cup(B_n\setminus B_{n-1})\\
        &=\cup_{j=1}^nB_j\\
        &=B_n\tag{2}
    \end{align*}
    where the last equality uses (1). Armed with this we derive the result:
    \begin{align*}
        \mu\left(\bigcup_{k=1}^n A_k\right)&=\mu\left(\lim_{n\rightarrow\infty}\bigcup_{k=1}^nA_k\right)\\
        &=\mu\left(\lim_{n\rightarrow\infty}B_n\right)\\
        &=\mu\left(\lim_{n\rightarrow\infty}\bigcup_{k=1}^nC_k\right)\tag{by (2)}\\
        &=\lim_{n\rightarrow\infty}\sum_{k=1}^n\mu(C_k)\tag{countable additivity}\\
        &=\lim_{n\rightarrow\infty}\mu\left(\bigcup_{k=1}^n C_j\right)\tag{finite additivity}\\
        &=\lim_{n\rightarrow\infty}\mu\left(B_n\right)\tag{by (2)}\\
        &=\lim_{n\rightarrow\infty}\mu\left(\bigcup_{k=1}^nA_k\right)\tag*{$\qed$}
    \end{align*}
    {\bf Problem 2} Show that the function $f:[0,1]\rightarrow\mbb{R}$ given by $f(x)=1$ if $x$ irrational and zero otherwise,
    is a measurable function where $[0,1]$ is endowed with Lebesgue measure and $\mbb{R}$ with the usual topology.\\[10pt]
    {\bf Proof} Let $([0,1],\mc{A},\mu)$ be the set $[0,1]$ equipped with the Lebesgue measure $\mu:\mc{A}\rightarrow[0,\infty]$ and $\sigma$-algebra $\mc{A}$ generated by sets of the form $(a,b)$, where $0\leq a<b\leq 1$.
    To prove this we make use of the following lemma:
    \begin{center}
        \begin{minipage}[c]{0.85\linewidth}
            {\bf Lemma 2.1} Let $(\mbb{X},\mc{B},\lambda)$ be a measure space and for $E\subseteq \mbb{X}$, $\mathbbm{1}_E:\mbb{X}\rightarrow\mbb{R}$ the characteristic function of $E$ defined by
            \[\mathbbm{1}_E(x)=\begin{cases}
                1\quad\text{if $x\in \mbb{X}$}\\
                0\quad\text{otherwise}
            \end{cases}.\]
            Then $\mathbbm{1}_E$ is measurable if and only if $E$ is.\\[10pt]
            {\bf Proof} Let $\alpha\in\mbb{R}$. Then there are three cases to consider when evaluating the set $\{x\in \mbb{X}:\mathbbm{1}_E>\alpha\}$:
            \begin{align*}
                \begin{array}{cl}
                    \{x\in\mbb{X}:\bbm{1}_E(x)>\alpha\}=\emptyset& \quad\text{if $\alpha\geq 1$}\\
                    \{x\in\mbb{X}:\bbm{1}_E>\alpha\}=E&\quad\text{if $\alpha\in[0,1)$}\\
                    \{x\in\mbb{X}:\bbm{1}_E(x)>\alpha\}=\mbb{X}&\quad\text{if $\alpha<0$}
                \end{array}
            \end{align*}
            where, since $\mc{B}$ is a $\sigma$-algebra, $\mbb{X}\in\mc{B}$ and $\mbb{X}\setminus\mbb{X}=\emptyset\in\mc{B}$, so $\bbm{1}_E$ is measurable if and only if $E\in\mc{B}$.\hfill{$\qed$}
        \end{minipage}
    \end{center}\vspace{10pt}
    The function of interest $f$ is just the characteristic function on the set $[0,1]\setminus\mbb{Q}_{[0,1]}$, where $\mbb{Q}_{[0,1]}=\mbb{Q}\cap[0,1]$, so we need only ascertain
    the measurability of $[0,1]\setminus\mbb{Q}_{[0,1]}$. For notational convenience, let $\mbb{P}=[0,1]\setminus\mbb{Q}_{[0,1]}$. To establish the measurability of $\mbb{P}$ we need a bit more background.\\[10pt]
    \begin{center}
        \begin{minipage}[c]{0.85\linewidth}
            {\bf Lemma 2.2} Let $\mc{B}$ be a $\sigma$-algebra on some space $\mbb{X}$ and $\{B_n\}_{n=1}^\infty\subseteq\mc{B}$. Then $\cap_{n=1}^\infty B_n\in\mc{B}$.\\[10pt]
            {\bf Proof} This follows from the the definition of $\sigma$-algebras and an application of DeMorgan's law:
            \begin{align*}
                \cap_{n=1}^\infty B_n&=\mbb{X}\setminus\left(\cup_{n=1}^\infty (\mbb{X}\setminus B_n)\right)
            \end{align*}
            where $\mbb{X}\setminus B_n\in\mc{B}$ for all $n\in\mbb{N}$ since $\mc{B}$ is closed under complementation, so $\cup_{n=1}^\infty(\mbb{X}\setminus B_n)\in\mc{B}$ since $\mc{B}$ is closed under countable union, and finally $\mbb{X}\setminus(\cup_{n=1}^\infty(\mbb{X}\setminus B_n))$,
            once more by closure of $\mc{B}$ under complementation.\hfill{$\qed$}
        \end{minipage}
    \end{center}\vspace{10pt}
    \begin{center}
        \begin{minipage}[c]{0.85\linewidth}
            {\bf Lemma 2.3} Let $a<b$ and $\mc{B}$ be the $\sigma$-algebra on $[a,b]$ generated by the relative topology of $\mbb{R}$ on $[a,b]$. Then $\forall x\in[a,b]$, $\{x\}\in\mc{B}$.\\[10pt]
            {\bf Proof} Let $x\in(a,b)$ and $\varepsilon=\min\{|x-a|,|x-b|\}$. Consider the sequence of sets $\{A_n\}_{n=1}^\infty$ where $A_n=(x-\varepsilon/n,x+\varepsilon/n)\in\mc{B}$ for $n\geq 1$. Since $x\in A_n$ for all $n$, we have
            \[x\in\cap_{n=1}^\infty A_n.\]
            Now consider $y\in[a,b]$ with $y\neq x$. Then $\exists\delta>0$ so that $|x-y|<\delta$. But by the Archimedean property of $\mbb{R}$, $\exists M\in\mbb{Z}_{>0}$ so that $\delta>\varepsilon/M$. Then, provided $n\geq M$, 
            \[y\notin A_n=(x-\varepsilon/n,x+\varepsilon/n).\]
            Since $y$ was arbitrary, we conclude that $\{x\}=\cap_{n=1}^\infty A_n$, so $\{x\}\in\mc{B}$ by lemma 2.2. For $x=a$ or $x=b$ we repeat this argument, taking $\varepsilon=|x-b|$ and $A_n=[x,x+\varepsilon/n)$ if $x=a$, or else $\varepsilon=|x-a|$ and $A_n=(x-\varepsilon/n,x]$ if $x=b$.\hfill{$\qed$}
        \end{minipage}
    \end{center}\vspace{10pt}
    \begin{center}
        \begin{minipage}[c]{0.85\linewidth}
            {\bf Definition 2.4} Let $\mc{I}$ be the set of all intervals in $\bar{\mbb{R}}$ ($\bar{\mbb{R}}=\mbb{R}\cup\{-\infty,\infty\}$) so that for $I\in\mc{I}$, $\exists a,b\in\bar{\mbb{R}}$ with $a\leq b$ so that $I\in\{(a,b),[a,b),(a,b],[a,b]\}$. We define $\ell:\mc{I}\rightarrow[0,\infty]$ to be the length function, given by
            \[\ell(I)=b-a.\]
            In particular, if $I=(a,a)=\emptyset$ or $I=[a,a]=\{a\}$, we have $\ell(I)=a-a=0$. Furthermore, if $\mc{B}\subseteq 2^{\bar{\mbb{R}}}$ is a $\sigma$-algebra and $\lambda:\mc{B}\rightarrow[0,\infty]$ is the Lebesgue measure, then $\forall I\in\mc{I}$ with $I\in\mc{B}$, $\lambda(I)=\ell(I)$. 
        \end{minipage}
    \end{center}\vspace{10pt}
    With these we are prepared to solve the problem at hand. Since $\mbb{Q}_{[0,1]}$ is countable, we can write $\mbb{Q}_{[0,1]}=\{q_k\}_{k=1}^\infty$. 
    But $\forall q\in\mbb{Q}_{[0,1]}$, $\{q\}\in\mc{A}$ (by lemma 2.3), so
    \[\mbb{Q}_{[0,1]}=\cup_{n=1}^\infty\{q_n\}\in\mc{A}\]
    by closure of $\mc{A}$ under countable union. But then
    \[\mbb{P}=[0,1]\setminus\mbb{Q}_{[0,1]}\in\mc{A}\]
    since $\mc{A}$ is closed under complementation. But this says that $\mbb{P}$ is measurable, so by lemma 2.1, $f$ is measurable.\hfill{$\qed$}\\[10pt]
    {\bf Problem 3} A measure space $(X,\mc{A},\mu)$ is called {\it complete} if every subset of a null set (that is, a set of measure zero) is measurable.
    For a complete measure space $(X,\mc{A},\mu)$, let $A\subseteq B$ be measurable sets such that $\mu(A)=\mu(B)$. Show that any set $C$ with $A\subseteq C\subseteq B$
    is also measurable and $\mu(A)=\mu(C)$.\\[10pt]
    {\bf Proof} Suppose $(X,\mc{A},\mu)$ is a complete measure space with $A,B\in\mc{A}$ such that $A\subseteq B$ and $\mu(A)=\mu(B)$. Then let $C$ be such that $A\subseteq C\subseteq B$. First, we have that
    \[\mu(B)=\mu((B\setminus A)\cup A)=\mu(B\setminus A)+\mu(A)\tag{since $(B\setminus A)\cap A=\emptyset$}\] 
    which makes use of the finite additivity of $\mu$:
    \begin{center}
        \begin{minipage}[c]{0.85\linewidth}
            {\bf Lemma 3.1} Let $\mc{B}$ be a $\sigma$-algebra and $\lambda:\mc{B}\rightarrow[0,\infty]$ a measure on $\mc{B}$. Then, if $U,V\in\mc{B}$ with $U\cap V=\emptyset$, we have
            \[\lambda(U\cup V)=\lambda(U)+\lambda(V).\]
            {\bf Proof} This follows directly from countable additivity. Let $\{U_k\}_{k=1}^\infty\subseteq\mc{B}$ be a sequence of measurable sets, where $U_1=U, U_2=V$ and $U_k=\emptyset$ for all $k\geq 3$.
            Then $U_i\cap U_j=\emptyset$ for all $i\neq j$, so
            \[\lambda(U\cup V)=\lambda\left(\bigcup_{k=1}^\infty U_k\right)=\sum_{k=1}^\infty\lambda(U_k)=\lambda(U)+\lambda(V).\tag*{$\qed$}\]
            \end{minipage}
    \end{center}\vspace{10pt}
    Resuming the above, we have
    \[\mu(B)=\mu(B\setminus A)+\mu(A)\quad\Rightarrow\quad \mu(B\setminus A)=0.\tag{since $\mu(A)=\mu(B)$}\]
    But $C\subseteq B\Rightarrow C\setminus A\subseteq B\setminus A$, so by the completeness of $X$ we have $C\setminus A\in\mc{A}$. Further, since $A\subseteq C$, $C=(C\setminus A)\cup A\in\mc{A}$ due to the closure of $\mc{A}$ under finite unions.\\[10pt]
    We next show that $\mu(A)=\mu(C)$. Using similar reasoning as to show $\mu(B\setminus A)=0$:
    \begin{align*}
        &\mu(B)=\mu((B\setminus C)\cup(C\setminus A)\cup A)=\mu(B\setminus C)+\mu(C\setminus A)+\mu(A)\tag{finite additivity}
    \end{align*}
    where since $\mu(A)=\mu(B)$ we get
    \[0=\mu(B\setminus C)+\mu(C\setminus A)\quad\Rightarrow\quad-\mu(B\setminus C)=\mu(C\setminus A)\quad\Rightarrow\quad\mu(B\setminus C)=0\quad\text{and}\quad\mu(C\setminus A)=0\]
    with the second implication following from the observation that $-\mu(B\setminus C)\leq 0$ and $0\leq\mu(C\setminus A)$. With this we can finish the proof: 
    \begin{align*}
        \mu(C)&=\mu(C\cup A)\tag{since $A\subseteq C$}\\
        &=\mu((C\setminus A)\cup A)\\
        &=\mu(C\setminus A)+\mu(A)\tag{since $(C\setminus A)\cap A=\emptyset$}\\
        &=\mu(A)\tag{since $\mu(C\setminus A)=0$}
    \end{align*}
    where now we have shown both that $C\in\mc{A}$ and $\mu(C)=\mu(A)$, as desired.\hfill{$\qed$}\\[10pt]
    {\bf Problem 4} A real number is said to be {\it trancendental} if it is not a root of any polynomial with integer coefficients. Show that the set of trancendental numbers
    in the interval $[0,1]$ has Lebesgue measure $1$.\\[10pt]
    {\bf Proof} Let $([0,1],\mc{A},\mu)$ be our measure space and $\mc{T}$ the set of trancendental numbers on $[0,1]$. Then, $\forall x\in[0,1]\setminus\mc{T}$, $x$ is a root of a polynomial with integer coefficients. We will first show that the set of all such points is countable,
    and so is measurable with measure zero.\\[10pt]
    Let $n\in\mbb{N}$ and denote $\mbb{P}_n$ the set of degree $n$ polynomials with integer coefficients. That is, if $p\in\mbb{P}_n$, then $\exists \alpha_0,\alpha_1,\dots,\alpha_n\in\mbb{Z}$ so that $p(x)=\sum_{k=0}^n\alpha_kx^k$. By the following lemma, this set is countable.
    \begin{center}
        \begin{minipage}[c]{0.85\linewidth}
            {\bf Lemma 4.1} $\mbb{P}_n$ is countable for any $n\geq 1$.\\[10pt]
            {\bf Proof} Define $\phi_n:\mbb{P}_n\rightarrow\mbb{Z}^{n+1}$ so that $\phi_n(p)=(\alpha_0,\alpha_1,\dots,\alpha_n)$, the $(n+1)$-tuple containing the coefficients of $p$. $\phi_n$ is injective, since if $p_1,p_2\in\mbb{P}_n$ with $p_1\neq p_2$, then
            $p_1=\sum_{k=0}^n\beta_kx^k$ and $p_2=\sum_{k=0}^n\gamma_kx^k$ with $\beta_i,\gamma_i\in\mbb{Z}$ for $i=0,1,\dots,n$ and $\beta_j\neq\gamma_j$ for some $j=0,1,\dots,n$. But then $\phi_n(p_1)=(\beta_0,\beta_1,\dots,\beta_n)\neq(\gamma_0,\gamma_1,\dots,\gamma_n)=\phi_n(p_2)$,
            and we resolve that the map $\phi_n$ is injective.\\[10pt]
            Now define $\psi_n:\mbb{Z}^{n+1}\rightarrow \mbb{P}_n$ so that $\psi_n((\alpha_0,\alpha_1,\dots,\alpha_n))=p$, with $p(x)=\sum_{k=0}^n\alpha_kx^k$. This function is also injective. To see this, let $\beta=(\beta_0,\beta_1,\dots,\beta_n),\gamma=(\gamma_0,\gamma_1,\dots,\gamma_n)\in\mbb{Z}^{n+1}$ with $(\beta_0,\beta_1,\dots,\beta_n)\neq(\gamma_0,\gamma_1,\dots,\gamma_n)$. That is,
            $\beta_j\neq\gamma_j$ for some $j=0,1,\dots,n$. Let $\psi_n(\beta)=p_1$ and $\psi_n(\gamma)=p_2$. Then, $\exists x\in\mbb{R}:$ $p_1(x)=\sum_{k=0}^n\beta_kx^k\neq\sum_{k=0}^n\gamma_kx^k=p_2(x)$, so $p_1\neq p_2$. We resolve that the map $\psi_n$ is injective.\\[10pt]
            We use the following theorem to establish a bijection between $\mbb{P}_n$ and $\mbb{Z}^{n+1}$.
            \begin{center}
                \begin{minipage}[c]{0.85\linewidth}
                    {\bf Theorem (Schr\"oder-Bernstein)} For any two sets $A,B$, if $\exists f,g$ injective with $f:A\rightarrow B$ and $g:B\rightarrow A$, then $\exists h:A\rightarrow B$, where $h$ is a bijection.
                \end{minipage}
            \end{center}\vspace{10pt}
            So, our work above says that there exists a bijection $\Phi_n:\mbb{P}_n\rightarrow\mbb{Z}^{n+1}$, and we conclude that the sets have the same cardinality. But $\mbb{Z}$ is countable, so $\mbb{Z}^{n+1}$ is countable, and thus $\mbb{P}_n$ is countable.\hfill{$\qed$}
        \end{minipage}
    \end{center}\vspace{10pt}
    Further, by the fundamental theorem of algebra, for each $p\in\mbb{P}_n$, $p$ has at most $n$ real roots. Thus defining $\sigma_n(p)=\{x\in\mbb{R}:p(x)=0\}$ for any $p\in\mbb{P}_n$ and letting $\mbb{P}_n=\{p_k\}_{k=1}^\infty$, the set of all roots of degree $n$ polynomials with integer coefficients is
    \[\bigcup_{k=1}^\infty\sigma_n(p_k)=:\Lambda_n\]
    a countable union of finite sets, which is thus at most countable. But this is true for any $n\geq 1$, so we could repeat this process to obtain $\Lambda_k$ for $k=1,2,\dots$, all of which are at most countable. Thus, the set of all roots of polynomials of any degree with integer coefficients is
    \[\bigcup_{k=1}^\infty\Lambda_k=:\Theta\]
    and $\Theta$ is again at most countable, since it is a countable union of at most countable sets. Further, $[0,1]\cap\Theta$ is at most countable, so we can write $\Theta_{[0,1]}:=[0,1]\cap\Theta=\{\theta_k\}_{k=1}^\infty$. By lemma 2.3, $\{\theta_k\}\in\mc{A}$, so $\Theta_{[0,1]}=\cup_{k=1}^\infty\{\theta_k\}\in\mc{A}$. Furthermore, we have
    \begin{align*}
        \mu(\Theta_{[0,1]})=\mu\left(\bigcup_{k=1}^\infty\{\theta_k\}\right)
        &=\sum_{k=1}^\infty\mu(\{\theta_k\})\tag{countable additivity}\\
        &=\sum_{k=1}^\infty\ell([\theta_k,\theta_k])\tag{definition 2.4}\\
        &=\sum_{k=1}^\infty(\theta_k-\theta_k)\\
        &=0.
    \end{align*}
    Next, we note that the set of trancendental numbers in $[0,1]$ are precisely those not in $\Theta_{[0,1]}$, so $\mc{T}=[0,1]\setminus\Theta_{[0,1]}$. But then $\mc{T}\in\mc{A}$, since $\mc{A}$ is closed under complementation. Finally,
    \[1=\ell([0,1])=\mu([0,1])=\mu(([0,1]\setminus\Theta_{[0,1]})\cup\Theta_{[0,1]})=\mu([0,1]\setminus\Theta_{[0,1]})+\mu(\Theta_{[0,1]})=\mu(\mc{T})\]
    so indeed the trancendentals in $[0,1]$ have Lebesgue measure $1$.\hfill{$\qed$}\\[10pt]
    {\bf Problem 5} Suppose that $f,g$ are two bounded measurable functions on $[0,1]$ endowed with Lebesgue measure satisfying $f(x)\leq g(x)$ almost everywhere.
    Prove that $\int_0^1f(x)dx\leq\int_0^1g(x)dx$.\\[10pt]
    {\bf Proof} Let $([0,1], \mc{A},\mu)$ be a measure space with lebesgue measure $\mu$. Since $f\leq g$ almost everywhere ({\it a.e.}) $\exists N\subset[0,1]$ with $N\in\mc{A}$ such that $\mu(N)=0$ and
    \[\forall x\in [0,1]\setminus N,\quad f(x)\leq g(x)\quad\text{and}\quad \forall x\in N,\quad f(x)> g(x).\]
    Since $f,g$ are not necessarily non-negative, we need to extend the notion of Lebesgue integration to such functions.
    \begin{center}
        \begin{minipage}[c]{0.85\linewidth}
            {\bf Lemma 5.1} Let $(\mbb{X},\mc{B},\lambda)$ be a measure space with $f:\mbb{X}\rightarrow[-\infty,\infty]$ measurable. If $f^+=\max\{f,0\}$ and $f^-=\max\{0,-f\}$, then
            $f=f^+-f^-$ and the functions $f^+$ and $f^-$ are both non-negative and measurable. \\[10pt]
            {\bf Proof} The non-negativity of $f^+$ and $f^-$ are clear. Let $x\in \mbb{X}$. If $f(x)\geq 0$, then $f^+(x)=f(x)$ and $f^-(x)=0$, so $f(x)=f^+(x)-f(x)$. If $f(x)<0$, then $f^+(x)=0$ and $f^-(x)=-f(x)$, so $f(x)=0-(-f(x))=f^+(x)-f^-(x)$. To see measurability, let $\alpha\in[-\infty,\infty]$ and observe that
            \begin{align*}
                \{x\in\mbb{X}:f^+(x)>\alpha\}&=\begin{cases}
                    \{x\in\mbb{X}:f(x)>\alpha\}\quad&\text{if $\alpha\geq 0$}\\
                    \mbb{X}\quad&\text{otherwise}
                \end{cases}
            \end{align*}
            both elements of $\mc{B}$ since $f$ is measurable. Similarly,
            \begin{align*}
                \{x\in\mbb{X}:f^-(x)>\alpha\}&=\begin{cases}
                    \{x\in\mbb{X}:-f(x)>\alpha\}\quad&\text{if $\alpha\geq 0$}\\
                    \mbb{X}\quad&\text{otherwise}
                \end{cases}
            \end{align*}
            where $\{x\in\mbb{X}:-f(x)>\alpha\}=\{x\in\mbb{X}:f(x)<-\alpha\}\in\mc{B}$, once more by measurability of $f$.\hfill{$\qed$}
        \end{minipage}
    \end{center}\vspace{10pt}
    With this we are able to define the Lebesgue integral for signed real functions in terms of the definition for non-negative functions [1].
    \begin{center}
        \begin{minipage}[c]{0.85\linewidth}
            {\bf Definition 5.2} Let $(\mbb{X},\mc{B},\lambda)$ be a measure space, $f:\mbb{X}\rightarrow[-\infty,\infty]$ be a measurable function and $f^+,f^-$ as above. If either
            \[\int_\mbb{X}f^+d\lambda<\infty\quad\text{or}\quad\int_\mbb{X}f^-d\lambda<\infty\]
            then the Lebesgue integral of $f$ is defined as
            \[\int_\mbb{X}fd\lambda=\int_\mbb{X}f^+d\lambda-\int_\mbb{X}f^-d\lambda.\] 
        \end{minipage}
    \end{center}\vspace{10pt}
    Returning to the actual problem at hand, let $\bbm{1}_N$, be the characteristic function of $N$ and $\bbm{1}_{[0,1]\setminus N}$ of $[0,1]\setminus N$. The following lemma establishes the measurability of products of these with $f$ and $g$. 
    \begin{center}
        \begin{minipage}[c]{0.85\linewidth}
            {\bf Lemma 5.3} Let $(\mbb{X},\mc{B},\lambda)$ be a measure space, $h$ be a measurable function and $\bbm{1}_A$ the characteristic function of a set $A\in\mc{B}$. Then $h\bbm{1}_A$ is measurable.\\[10pt]
            {\bf Proof} If $\alpha\geq 0$, we have 
            \begin{align*}
                \{x\in\mbb{X}:(f\bbm{1}_A)(x)>\alpha\}&=
                    \{x\in\mbb{X}:f(x)>\alpha\}\cap A\in\mc{B}
            \end{align*}
            which follows from the measurability of $f$ and $A$. If $\alpha<0$, then instead
            \begin{align*}
                \{x\in\mbb{X}:(f\bbm{1}_A)(x)>\alpha\}&=
                    \{x\in\mbb{X}:f(x)>\alpha\}\cup (\mbb{X}\setminus A)\in\mc{B}
            \end{align*}
            which also follows from the measurability of $f$ and $A$.\hfill{$\qed$}
        \end{minipage}
    \end{center}\vspace{10pt}
    From this, we can proceed by splitting up the integral of $f$ over the sets $N$ and $[0,1]\setminus N$.
    \begin{align*}
        \int_0^1fdx&=\int_0^1f\bbm{1}_N+f\bbm{1}_{[0,1]\setminus N}dx\\
        &=\int_0^1f\bbm{1}_Ndx+\int_0^1 f\bbm{1}_{[0,1]\setminus N}dx.\tag{3}
    \end{align*}
    where the additivity in (3) is well-defined since $f$ is bounded on a domain of finite measure. That is, 
    \[\int_0^1fdx=\int_0^1f^+dx-\int_0^1f^-dx\]
    but $f^+,f^- \leq |f|$ and $\sup_{x\in[0,1]}|f(x)|<\infty$, so
    \[-\infty<0\cdot\mu([0,1])-\sup_{x\in[0,1]}|f(x)|\mu([0,1])=0\cdot\int_0^1dx-\sup_{x\in[0,1]}|f(x)|\int_0^1dx\leq \int_0^1f^+dx-\int_0^1f^-dx\]
    and similarly
    \[\infty>\sup_{x\in[0,1]}|f(x)|\mu([0,1])-0\mu([0,1])=\sup_{x\in[0,1]}|f(x)|\int_0^1dx-0\int_0^1dx\geq \int_0^1f^+dx-\int_0^1f^-dx.\]
    where clearly $f\bbm{1}_N$ and $f\bbm{1}_{[0,1]\setminus N}$ are also bounded functions and so by similar logic have finite integrals over the set $[0,1]$. The following lemma allows us to
    restrict the region of integration for such functions.
    \begin{center}
        \begin{minipage}[c]{0.85\linewidth}
            {\bf Lemma 5.4} Let $(\mbb{X},\mc{B},\lambda)$ be a measure space, and $h:\mbb{X}\rightarrow[0,\infty]$ a non-negative measurable function. Let $A\in\mc{B}$, and $\bbm{1}_A$ its characteristic function.
            Then we have
            \[\int_\mbb{X}h\bbm{1}_Ad\lambda=\int_Ahd\lambda.\]
            {\bf Proof} From the definition of the Lebesgue integral, we have
            \[\int_\mbb{X}h\bbm{1}_Ad\lambda=\sup\int_\mbb{X}\phi d\lambda\]
            where the supremum is taken over the set of simple measurable functions satisfying $0\leq \phi\leq h\bbm{1}_A$. For any such simple function, we have $\phi=\sum_{k=1}^N\alpha_k\bbm{1}_{A_k}$
            with $\{x:\phi(x)=\alpha_i\}=A_i\subseteq A$ for $i=1,2,\dots, N$. Otherwise, for some $j=1,2,\dots,N$ we could find $x\in A_j$ with $x\notin A$ so that $\phi(x)=\alpha_j>0=(h\bbm{1}_A)(x)$.
            But now we have, for any $i=1,2,\dots,N$
            \[A_i\cap\mbb{X}=A_i\cap A\]
            since $A_i\subseteq A$ and $A\subseteq \mbb{X}$. Thus,
            \[\int_{\mbb{X}}h\bbm{1}_Ad\lambda=\sup\sum_{k=1}^N\alpha_k\lambda(\mbb{X}\cap A_k)=\sup\sum_{k=1}^N\alpha_k\lambda(A\cap A_k)=\sup\int_A\phi d\lambda=\int_A hd\lambda\]
            where the last equality holds since the supremum is taken over all measurable simple functions with $0\leq \phi\leq h\bbm{1}_A=h$ on $A$.\hfill{$\qed$}
        \end{minipage}
    \end{center}\vspace{10pt}
    While lemma 5.4 requires non-negativity of the function, note that for $x\in[0,1]$, $A\in\mc{A}$
    \begin{align*}
        (f\bbm{1}_A)^+(x)=\max\{(f\bbm{1}_A)(x),0\}&=\begin{cases}
            f(x)\bbm{1}_A(x)\quad&\text{if $f(x)> 0$, $x\in A$}\\
            0\quad&\text{otherwise}
        \end{cases}
    \end{align*}
    so that $(f\bbm{1}_A)^+=f^+\bbm{1}_A$. Identically, we have $(f^-\bbm{1}_A)=f^-\bbm{1}_A$, so
    \[\int_0^1f\bbm{1}_Adx=\int_0^1f^+\bbm{1}_Adx-\int_0^1f^-\bbm{1}_Adx=\int_Af^+dx-\int_Af^-dx=\int_Afdx\]
    which we can use in our proof. Picking up where we left off at (3):
    \begin{align*}
        \int_0^1fdx&=\int_0^1f\bbm{1}_Ndx+\int_0^1f\bbm{1}_{[0,1]\setminus N}dx\\
        &=\int_Nfdx+\int_{[0,1]\setminus N}fdx.\tag{4}
    \end{align*}
    And we need only one more result to complete the proof.
    \begin{center}
        \begin{minipage}[c]{0.85\linewidth}
            {\bf Lemma 5.5} Let $(\mbb{X},\mc{B},\lambda)$ be a measure space. If $A,B\in\mc{B}$ with $A\subseteq B$, then $\lambda(A)\leq \lambda(B)$.\\[10pt]
            {\bf Proof} Let $C=B\setminus A$. Then $C\cap A=\emptyset$ and $A\cup C=B$, so by finite additivity (lemma 3.1) we get
            \[\lambda(B)=\lambda(A\cup C)=\lambda(A)+\lambda(C).\]
            But $\lambda$ is a positive measure, so this implies that $\lambda(B)\geq \lambda(A)$.
        \end{minipage}
    \end{center}\vspace{10pt}
    \begin{center}
        \begin{minipage}[c]{0.85\linewidth}
            {\bf Lemma 5.6} Let $(\mbb{X},\mc{B},\lambda)$ be a measure space, $h:\mbb{X}\rightarrow[0,\infty]$ be a non-negative, measurable function, and $A\in\mc{B}$ with $\lambda(A)=0$.
            Then
            \[\int_Ahd\lambda=0.\]
            {\bf Proof} Consider the set of measurable simple functions  $\phi$ satisfying $0\leq \phi\leq h$ so that
            \[\int_A hd\lambda=\sup\int_A\phi d\lambda\]
            with the supremum taken over this set. For any particular such simple function $\phi$, $\exists \alpha_i>0$, $A_i\subseteq A$ with $i=1,2\dots, N$ so that $\phi=\sum_{k=1}^N\alpha_k\bbm{1}_{A_k}$. Thus,
            \[\int_A\phi d\lambda=\sum_{k=1}^N\alpha_k\lambda(A\cap A_k)=0\]
            since $A\cap A_i\subseteq A$ implies that $0\leq\lambda(A\cap A_i)\leq \lambda(A)=0$ (lemma 5.5) for $i=1,2\dots, N$. The supremum of a set consisting entirely of zeros is just zero, so we get the result.\hfill{$\qed$}
        \end{minipage}
    \end{center}\vspace{10pt}
    Finally, returning to (4):
    \begin{align*}
        \int_0^1fdx&=\int_Nfdx+\int_{[0,1]\setminus N}fdx\\
        &=\int_{[0,1]\setminus N}fdx\tag{lemma 5.6, since $\mu(N)=0$}\\
        &\leq\int_{[0,1]\setminus N}gdx\tag{since $\forall x\in[0,1]\setminus N$, $f(x)\leq g(x)$}\\
        &=\int_{[0,1]\setminus N}gdx+\int_Ngdx\tag{lemma 5.6, since $\mu(N)=0$}\\
        &=\int_0^1gdx
    \end{align*}
    and we are done.\hfill{$\qed$}\newpage
    \noindent{\bf Problem 6} Let $(X,\mc{A},\mu)$ be a measure space and $f:X\rightarrow[0,\infty]$ be measurable. Suppose that for some $E\in\mc{A}$ we have
    \[\int_Efd\mu=0.\]
    Show that $f(x)=0$ for almost all $x\in E$.\\[10pt]
    {\bf Proof} For this problem we will use the countable subadditivity of positive measures.
    \begin{center}
        \begin{minipage}[c]{0.85\linewidth}
            {\bf Lemma 6.1} Let $(\mbb{X},\mc{B},\lambda)$ be a measure space and $\{A_k\}_{k=1}^\infty\subseteq\mc{B}$. Then
            \[\lambda\left(\bigcup_{k=1}^\infty A_k\right)\leq\sum_{k=1}^\infty\lambda(A_k).\]
            {\bf Proof} Define $B_1=A_1$, and for $n\geq 2$, $B_n=A_n\setminus(\cup_{k=1}^{n-1}A_k)$. If $i,j\geq 1$ and $i\neq j$, then $B_i\cap B_j=\emptyset$. To see this,
            notice that without loss of generality we can assume $j>i$. Then, $x\in B_i\subseteq A_i$ implies $x\in\cup_{k=1}^{j-1}A_k$, so $x\notin B_j$.\\[10pt]
            We also have $\cup_{k=1}^nB_k=\cup_{k=1}^nA_k$.
            The inclusion $\cup_{k=1}^nB_k\subseteq\cup_{k=1}^nA_k$ is clear, so take $x\in\cup_{k=1}^nA_k$. Then $x\in A_j$ for some $j=1,2,\dots,n$. If $x\notin A_j\setminus\cup_{k=1}^{j-1}A_k$, then $x\in A_i$ for some $i=1,2,\dots,j-1$. Proceeding recursively,
            we get $x\in B_\ell$ for some $\ell=1,2,\dots, j$.\\[10pt]
            It is also clear that $B_k\subseteq A_k$ for all $k\geq 1$. With this, the result follows by the countable additivity of $\lambda$ and lemma 5.5:
            \[\lambda\left(\bigcup_{k=1}^\infty A_k\right)=\lambda\left(\bigcup_{k=1}^\infty B_k\right)=\sum_{k=1}^\infty\lambda(B_k)\leq\sum_{k=1}^\infty\lambda(A_k).\tag*{$\qed$}\] 
        \end{minipage}
    \end{center}\vspace{10pt}
    Since $f\geq 0$, we wish to show that the set $N=\{x\in E:f(x)>0\}$ has measure $0$. Firstly, $N\in\mc{A}$ since $f$ is measurable. We next claim that
    \begin{align*}
        N&=\bigcup_{k=1}^\infty\{x\in E:f(x)>1/k\}\tag{5}
    \end{align*}
    and we will denote $\{x\in E:f(x)>1/k\}=:E_k$ for $k\geq 1$. Take $x\in N$. Then $f(x)>0$, so by the Archimedean property of $\mbb{R}$, $\exists M\in\mbb{Z}_{>0}$ so that $f(x)>1/M$, implying $x\in E_M$, and is further in the union.
    Conversely, let $x\in\cup_{k=1}^\infty E_k$. Then $x\in E_k$ for some $k\geq 1$, at which $f(x)>1/k$. Of course, then $f(x)>0$, so $x\in N$, and we have the equality (5). Then, by countable subadditivity (lemma 6.1) we have
    \[\mu(N)=\mu\left(\bigcup_{k=1}^\infty E_k\right)\leq\sum_{k=1}^\infty\mu(E_k).\tag{6}\]
    Assume for the purpose of deriving a contradiction that $\mu(E_k)\neq 0$ for some $k\geq 1$. Then 
    \begin{align*}
        0=\int_Efd\mu&=\int_Ef\bbm{1}_{E\setminus E_k}+f\bbm{1}_{E_k}d\mu\\
        &=\int_{E\setminus E_k}fd\mu+\int_{E_k}fd\mu\tag{lemma 5.4}\\
        &\geq\int_{E_k}fd\mu\tag{since $f\geq 0$}\\
        &\geq\int_{E_k}\frac{1}{k}d\mu\tag{since $f(x)\geq 1/k$ $\forall x\in E_k$}\\
        &=\frac{1}{k}\mu(E_k).\tag{since $(1/k)\bbm{1}_{E_k}$ is simple}\\
    \end{align*}
    However, we assumed that $\mu(E_k)\neq0$, so the above produces the following contradition:
    \[0\geq\frac{1}{k}\mu(E_k)>0\]
    wherefrom we resolve that $\mu(E_k)=0$ for all $k\geq 1$. Returning to (6) with this new fact, we get
    \[0\leq\mu(N)\leq\sum_{k=1}^\infty\mu(E_k)=0\]
    which says that the set $\{x\in E:f(x)>0\}$ has measure zero -- that is, $f$ is zero almost everywhere.\hfill{$\qed$}\\[10pt]
    {\bf Problem 7} Let $[a,b]$ be an interval in $\mbb{R}$ (which may be infinite) and suppose that $f$ is a continuous function on the rectangle $[a,b]\times[c,d]$.
    Let
    \[g(y):=\int_a^bf(x,y)dx.\]
    Suppose that $\partial f/\partial y$ exists and is continuous. Further suppose if $[c,d]$ is infinite, then $\partial f/\partial y$ is bounded. Show $g$ is differentiable, and 
    \[g^\prime(y)=\int_a^b\frac{\partial f}{\partial y}(x,y)dx.\]
    {\bf Proof} To find the derivative of $g$ we evaluate the limit:
    \begin{align*}
        \lim_{h\rightarrow 0}\frac{g(y+h)-g(y)}{h}&=\lim_{h\rightarrow 0}\frac{\int_a^bf(x,y+h)dx-\int_a^bf(x,y)dx}{h}\\
        &=\lim_{h\rightarrow 0}\int_a^b\frac{f(x,y+h)-f(x,y)}{h}dx
    \end{align*} 
    which can be done by way of the dominated convergence theorem, whereby we establish the validity of interchanging the limit with the integral.
    To this end, fix $y\in(c,d)$ and consider the sequence of functions $\{f_n\}_{n\geq 1}$ given by
    \begin{align*}
        f_n(x)=\frac{f(x,y+1/n)-f(x,y)}{1/n}\quad\text{so that}\quad\lim_{n\rightarrow\infty}f_n(x)=f_y(x,y).\tag{7}
    \end{align*}
    with $x\in[a,b]$. In the event that $y+1/n\geq d$ for some $n\geq 1$, $|y-d|>0$ so $\exists N\geq 1$ such that $y+1/n\leq d$ for all $n\geq N$ and the sequence $\{f_{N+k}\}_{k\geq 1}$ 
    has the same limiting behavior. The remainder of the proof still holds in this case, so for the sake of concision we will not explicitly cover it.\\[10pt]
    Now, fixing $x$ for the moment, since $f_y(x,y)$ exists for all $y\in(c,d)$, $f$ is continuous on $[y,y+1/k]$ and differentiable on $(y,y+1/k)$. By the mean value theorem, $\exists\xi\in(y,y+1/k)$ so that
    \begin{align*}
        |f(x,y+1/k)-f(x,y)|\leq\frac{1}{k}|f_y(x,\xi)|
    \end{align*}
    but regardless of the infinitude of $[c,d]$, we assume that $f_y$ is bounded, so let $\sup_{y\in(c,d)}|f_y(x,y)|=M<\infty$. Then we in fact have the bound
    \begin{align*}
        \left|\frac{f(x,y+1/k)-f(x,y)}{1/k}\right|\leq |f_y(x,\xi)|\leq M\tag{8}
    \end{align*}
    which is independent of $x$, and so holds for arbitrary $x\in[a,b]$. Thus we have found $|f_n(x)|\leq M$ for  $n\geq 1$ and any $x\in[a,b]$.
    However, the applicability of Lebesgue's dominated convergence theorem here has not been established, since
    \[M\int_a^bdx=\begin{cases}
        \infty\quad&\text{if $[a,b]$ infinite}\\
        C<\infty\quad&\text{for some $C\geq 0$ if $[a,b]$ finite.}
    \end{cases}\]
    Furthermore, in the case $[a,b]$ is infinite, $M\notin L^1([a,b])$ which precludes application of the dominated convergence theorem. Below we furnish an example to demonstrate the insufficiency of the boundedness of $f_y$. For additional comment
    on why the dominated convergence theorem cannot be applied here, see Appendix A.1. 
    \begin{center}
        \begin{minipage}[c]{0.85\linewidth}
            {\bf Proposition 7.1} There exists a measure space $(\mbb{X},\mc{B},\mu)$, a sequence of measurable functions $\{f_n\}_{n=1}^\infty$ and a bounded function $g\notin L^1(\mbb{X})$ all defined on some infinite domain $\mbb{X}$ with $\lim_{n\rightarrow\infty}f_n=f$ and $|f_n|\leq g$ for $n\geq 1$, but
            \[\lim_{n\rightarrow\infty}\int_\mbb{X}f_nd\mu\neq\int_\mbb{X}fd\mu.\]
            {\bf Proof} Consider the measure space $(\mbb{R}_{\geq 0},\mc{B},\mu)$ where $\mu$ is the Lebesgue measure. Let $f_n=\frac{1}{n}\bbm{1}_{[0,n)}$, $n\geq 1$ define a sequence of measurable functions so that $\lim_{n\rightarrow\infty}f_n(x)=0$ $\forall x\in\mbb{R}_{\geq 0}$. Now, consider the function given by
            \[g=\sum_{k=1}^\infty \frac{1}{k}\bbm{1}_{[k-1,k)}\]
            which is bounded, since $1\geq g(x)$ $\forall x\in\mbb{R}_{\geq 0}$. Observe also that $\bbm{1}_{[k-1,k)}$ is non-negative and measurable (by lemma 2.1) so by the monotone convergence theorem, we have
            \begin{align*}
                \int_{\mbb{R}_{\geq 0}}gd\mu&=\int_{\mbb{R}_{\geq 0}}\sum_{k=1}^\infty\frac{1}{k}\bbm{1}_{[k-1,k)}d\mu\\
                &=\sum_{k=1}^\infty\frac{1}{k}\int_{\mbb{R}_{\geq 0}}\bbm{1}_{[k-1,k)}d\mu\tag{monotone convergence theorem}\\
                &=\sum_{k=1}^\infty\frac{1}{k}\int_{k-1}^kd\mu\\
                &=\sum_{k=1}^\infty\frac{1}{k}\mu([k-1,k))\\
                &=\sum_{k=1}^\infty\frac{1}{k}\\
                &=\infty
            \end{align*}
            where the monotone convergence theorem was applied to the sequence of partial sums $S_N=\sum_{k=1}^{N}\frac{1}{k}\bbm{1}_{[k-1,k)}$. This tells us that $g\notin L^1(\mbb{R}_{\geq 0})$. Further, $g$ bounds the sequence of functions, since taking any $n\geq 1$ and $x\in\mbb{R}_{\geq 0}$ we have
            \[|f_n(x)|=\frac{1}{n}\bbm{1}_{[0,n)}(x)=\begin{cases}
                \frac{1}{n}\leq g(x)\quad&\text{if $x\in[0,n)$}\\
                0\leq g(x)\quad&\text{otherwise.}
            \end{cases}\]
            Despite this, we compute
            \begin{align*}
                \lim_{n\rightarrow\infty}\int_{\mbb{R}_{\geq 0}}f_nd\mu&=\lim_{n\rightarrow\infty}\frac{1}{n}\int_{\mbb{R}_{\geq 0}}\bbm{1}_{[0,n)}d\mu\\
                &=\lim_{n\rightarrow\infty}\frac{1}{n}\int_0^nd\mu\\
                &=1\\
                &\neq 0\\
                &=\int_{\mbb{R}_{\geq 0}}0d\mu\\
                &=\int_{\mbb{R}_{\geq 0}}\lim_{n\rightarrow\infty}f_nd\mu
            \end{align*}
            so the boundedness of $g$ is insufficient to justify the interchange of the limit and integral.\hfill{$\qed$}
        \end{minipage}
    \end{center}\vspace{10pt}
    By proposition 7.1, the boundedness of our sequence in (8) is insufficient to apply the dominated convergence theorem, so we instead prove a weaker result, either:
    \begin{align*}
        \begin{array}{l}
            \text{I) The interval $[a,b]$ is finite, or else}\\
            \text{II) We additionally suppose that there exists $g\in L^1([a,b])$ such that $\forall (x,y)\in[a,b]\times(c,d)$, $|f_y(x,y)|\leq g(x)$}
        \end{array}
    \end{align*} 
    In case (I), we have $|f_n|\leq M\bbm{1}_{[a,b]}\in L^1([a,b])$, and in case (II) we have $|f_n|\leq g\in L^1([a,b])$, which holds in either case for $n\geq 1$ and $\forall x\in[a,b]$.
    Then, by the dominated convergence theorem $f_y\in L^1([a,b])$ and
    \begin{align*}
        \lim_{h\rightarrow 0}\frac{g(y+h)-g(y)}{h}&=\lim_{n\rightarrow\infty}\int_a^b\frac{f(x,y+1/n)-f(x,y)}{1/n}dx\\
        &=\int_a^b\lim_{n\rightarrow\infty}\frac{f(x,y+1/n)-f(x,y)}{1/n}dx\\
        &=\int_a^bf_y(x,y)\,dx\\
        &=g^\prime(y)
    \end{align*}
    and we are done.\hfill{$\qed$}\\[10pt]
    {\bf Problem 8} For $t>0$, define
    \[g(t):=\int_0^\infty e^{-tx}\frac{\sin x}{x}\,dx.\]
    Prove that
    \[g^\prime(t)=-\int_0^\infty e^{-tx}\sin x\,dx.\]
    {\bf Proof} Let $f(x,t)=e^{-tx}\sin(x)/x$ be the integrand. Clearly, the integrand is continuously differentiable with respect to $t$, and its derivative
    is given by
    \[f_t(x,t)=\frac{\partial}{\partial t}e^{-tx}\frac{\sin x}{x}=-xe^{-tx}\frac{\sin x}{x}=-e^{-tx}\sin x.\]
    Additionally, $f_t(x,t)$ is dominated by a Lebesgue integrable function:
    \begin{align*}
        \left|f_t(x,t)\right|&=\left|-e^{-tx}\sin x\right|
        \leq e^{-tx}
    \end{align*}
    with $e^{-tx}\in L^1([0,\infty])$, since
    \[\int_0^\infty e^{-tx}\,dx=-\frac{1}{t}e^{-tx}\bigg|_0^\infty=\frac{1}{t}<\infty.\]
    Having met all the hypotheses of the result in problem 7, we find that
    \[g^\prime(t)=\int_0^\infty f_t(x,t)\,dx=-\int_0^\infty e^{-tx}\sin x\,dx\]
    so we are done.\hfill{$\qed$}\\[10pt]
    {\bf Problem 9} With $g(t)$ as in the previous exercise, show that
    \[g^\prime(t)=-\frac{1}{1+t^2}.\]
    Deduce that
    \[\int_0^\infty\frac{\sin x}{x}\,dx=\frac{\pi}{2}.\]
    {\bf Proof} To begin we evaluate the integral $g^\prime(t)$ using repeated integration by parts.
    \begin{align*}
        g^\prime(t)=\int_0^\infty-e^{-tx}\sin x\,dx&=\frac{1}{t}e^{-tx}\sin x\bigg|_0^\infty-\frac{1}{t}\int_0^\infty e^{-tx}\cos x\,dx\\
        &=\left[\frac{1}{t}e^{-tx}\sin x+\frac{1}{t^2}e^{-tx}\cos x\right]\bigg|_0^\infty+\frac{1}{t^2}\int_0^\infty e^{-tx}\sin x\,dx\\
    \end{align*}
    grouping the integrals on the left hand side, we get
    \begin{align*}
        &\left(\int_0^\infty-e^{-tx}\sin x\,dx\right)\left(\frac{t^2+1}{t^2}\right)=\left[\frac{1}{t}e^{-tx}\sin x+\frac{1}{t^2}e^{-tx}\cos x\right]\bigg|_0^\infty\\ 
        \Rightarrow\qquad&\int_0^\infty-e^{-tx}\sin x\,dx=\frac{te^{-tx}\sin x+e^{-tx}\cos x}{t^2+1}\bigg|_0^\infty\\
        \Rightarrow\qquad&\int_0^\infty-e^{-tx}\sin x\,dx=\lim_{x\rightarrow\infty}\left(\frac{te^{-tx}\sin x+e^{-tx}\cos x}{t^2+1}\right)-\frac{1}{1+t^2}\\
        \Rightarrow\qquad&g^\prime(t)=-\frac{1}{1+t^2}.
    \end{align*}
    where $\lim_{x\rightarrow\infty}e^{-tx}=0$ since $t>0$. Next, we can integrate this to obtain a primitive function $g(t)$. The integral of $1/(1+t^2)$ is well known, so
    \begin{align*}
        g(t)&=\int-\frac{1}{1+t^2}\,dt\\
        &=-\arctan t+C
    \end{align*}
    which consists of a term depending on $t$ and a constant $C$ which, critically, does not. But now we have
    \[\int_0^\infty e^{-tx}\frac{\sin x}{x}\,dx=-\arctan t+C.\tag{9}\]
    Next, observe that our target integral
    \[\int_0^\infty\frac{\sin x}{x}\,dx\tag{10}\]
    will evaluate to an expression without any dependence on $t$, and hence the $t$ dependence in (9) was introduced by the $e^{-tx}$ in the integrand. Thus, to compute (10) we need only identify the identity the constant $C$.\\[10pt]
    To do this, we consider the limit as $t\rightarrow\infty$ to make the integrand vanish in (9):
    \begin{align*}
        &\lim_{t\rightarrow\infty}\int_0^\infty e^{-tx}\frac{\sin x}{x}\,dx=\lim_{t\rightarrow\infty}-\arctan t+C\\
        \Rightarrow\qquad&0=-\frac{\pi}{2}+C\\
        \Rightarrow\qquad&C=\frac{\pi}{2}
    \end{align*}
    and so we deduce
    \[\int_0^\infty\frac{\sin x}{x}\,dx=\frac{\pi}{2}.\]
    How cool!\hfill{$\qed$}\\[10pt]
    {\bf Problem 10} Let $H$ be a Hilbert space and $u_1,u_2,\dots$ an orthonormal set in $H$. If
    \[x=\sum_{j=1}^\infty c_ju_j\]
    show that
    \[\|x\|^2=\sum_{j=1}^\infty|c_j|^2.\]
    {\bf Proof} The norm $\|\cdot\|:H\rightarrow\mbb{R}_{\geq 0}$ on $H$ is induced by the inner product $(\cdot,\cdot):H\times H\rightarrow\mbb{C}$ so that for $z\in H$,
    \[\sqrt{(z,z)}=\|z\|.\]
    Furthermore, the mapping $x\mapsto\|x\|$ is continuous, so $x\mapsto\|x\|^2$ is as well and thus
    \[\|x\|^2=\left\|\sum_{i=1}^\infty c_iu_i\right\|^2=\left\|\lim_{n\rightarrow\infty}\sum_{i=1}^nc_iu_i\right\|^2=\lim_{n\rightarrow\infty}\left\|\sum_{i=1}^nc_iu_i\right\|^2.\]
    With these we compute the squared norm of $x$ directly using properties of the inner product.
    \begin{align*}
        \|x\|^2=\lim_{n\rightarrow\infty}\left\|\sum_{i=1}^nc_iu_i\right\|&=\lim_{n\rightarrow\infty}\left(\sum_{i=1}^nc_iu_i,\sum_{j=1}^nc_ju_j\right)\\
        &=\lim_{n\rightarrow\infty}\sum_{i=1}^n\left(c_iu_i,\sum_{j=1}^n c_ju_j\right)\tag{linearity in the first argument}\\
        &=\lim_{n\rightarrow\infty}\sum_{i=1}^n\sum_{j=1}^n(c_iu_i,c_ju_j)\tag{linearity in the second argument}\\
        &=\lim_{n\rightarrow\infty}\sum_{i=1}^n\sum_{j=1}^n c_i\overline{c_j}(u_i,u_j).\tag{homogeneity/conjugate homogeneity}
    \end{align*}
    Since $u_i\perp u_j$ whenever $i\neq j$, we only need to consider terms with equal indices in the double summation. Further, for $i\geq 1$, $(u_i,u_i)=\|u_i\|^2=1$ by their orthonormality. Thus, the above becomes
    \[\|x\|^2=\lim_{n\rightarrow\infty}\sum_{j=1}^n c_j\overline{c_j}(u_j,u_j)=\lim_{n\rightarrow\infty}\sum_{j=1}^n c_j\overline{c_j}=\sum_{j=1}^\infty c_j\overline{c_j}.\]
    Of course, for any $z\in H$ complex, $z=a+ib$ for some $a,b\in\mbb{R}$ and $z\overline{z}=(a+ib)(a-ib)=a^2+b^2=|z|^2$, the squared modulus of $z$. Continuing the above, we finally have
    \begin{align*}
        \|x\|^2&=\sum_{j=1}^\infty c_j\overline{c_j}
        =\sum_{j=1}^\infty|c_j|^2
    \end{align*}
    as desired.\hfill{$\qed$}


    \newpage
    \noindent{\bf\Large Appendix}\\[10pt]
    {\bf A.1 The problem in problem 7}
    \begin{center}
        \begin{minipage}[c]{0.85\linewidth}
            {\bf Theorem (Dominated Convergence)} Let $(\mbb{X},\mc{B},\mu)$ be a meaure space and $\{f_n\}_{n\geq 1}$ a sequence of measurable functions on $\mbb{X}$ with 
            \[\lim_{n\rightarrow\infty}f_n=f\]
            pointwise on $\mbb{X}$. Further, suppose we have a function $g\in L^1(\mbb{X})$ so that $f_n\leq g$ for $n\in\mbb{N}$. Then $f\in L^1(\mbb{X})$ and
            \[\lim_{n\rightarrow\infty}\int_{\mbb{X}}f_nd\mu=\int_{\mbb{X}}fd\mu.\]
            {\bf Proof} Following the proof in [2], we first observe that $|f_n|\leq g$ for all $n$ implies that $|f|\leq g$, so $f\in L^1(\mbb{X})$. Then for $n\geq 1$
            \[|f_n-f|\leq|f|+|f_n|\leq 2g.\]
            But $\lim_{n\rightarrow\infty}f_n=f$ implies that $\lim_{n\rightarrow\infty}|f_n-f|=0$, so by Fatou's lemma
            \begin{align*}
                \int_\mbb{X}2gd\mu=\int_\mbb{X}2g+\liminf_{n\rightarrow\infty}|f_n-f|d\mu&=\int_\mbb{X}\liminf_{n\rightarrow\infty}2g+|f_n-f|d\mu\\
                &\leq\liminf_{n\rightarrow\infty}\int_{\mbb{X}}2g-|f_n-f|d\mu\tag{Fatou's lemma}\\
                &=\int_{\mbb{X}}2gd\mu+\liminf_{n\rightarrow\infty}\int_\mbb{X}-|f_n-f|d\mu\\
                &=\int_{\mbb{X}}2gd\mu-\limsup_{n\rightarrow\infty}\int_\mbb{X}|f_n-f|d\mu.
            \end{align*}
            Which gives us
            \[0\leq\liminf_{n\rightarrow\infty}\int_\mbb{X}|f_n-f|d\mu\leq\limsup_{n\rightarrow\infty}\int_\mbb{X}|f_n-f|d\mu\leq\int_\mbb{X}2gd\mu-\int_\mbb{X}2gd\mu=0\tag{$\ast$}\]
            so $\liminf_{n\rightarrow\infty}\int_\mbb{X}|f_n-f|d\mu=\limsup_{n\rightarrow\infty}\int_\mbb{X}|f_n-f|d\mu=0$, and thus
            \[\lim_{n\rightarrow\infty}\int_\mbb{X}f_nd\mu=\int_{\mbb{X}}fd\mu\tag*{$\qed$}\]
        \end{minipage}
    \end{center}\vspace{10pt}
    \begin{center}
        \begin{minipage}[c]{0.85\linewidth}
            {\bf Remarks} Observe that if the dominating function $g$ satisfied $\int_\mbb{X}gd\mu=\infty$, then in ($\ast$) we would have had
            \[\limsup_{n\rightarrow\infty}\int_\mbb{X}|f_n-f|d\mu\leq \infty-\infty\]
            an undefined expression which halts resolution of the proof. This on its own does not guarantee the indispensibility of Lebesgue integrable $g$, but proposition 7.1 does.
        \end{minipage}
    \end{center}\vspace{10pt}
    \noindent{\bf\Large References}\\[10pt]
    1. W. Rudin, {\it Principles of Mathematical Analysis}, 3rd edn. (McGraw-Hill, New York, 1953)\\[10pt]
    2. M. Ram Murty, {\it A Second Course in Analysis}, 1st ed. (Hindustan Book Agency, New Delhi, 2020)
\end{document}