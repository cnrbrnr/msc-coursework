\documentclass[11pt, letterpaper]{article}
\usepackage[margin=1.5cm]{geometry}
\pagestyle{plain}

\usepackage{amsmath, amsfonts, amssymb, amsthm}
\usepackage{bbm}
\usepackage[shortlabels]{enumitem}
\usepackage[makeroom]{cancel}
\usepackage{graphicx}
\usepackage{xcolor}
\usepackage{array, booktabs, ragged2e}
\graphicspath{{./Images/}}

\newcommand{\bs}[1]{\boldsymbol{#1}}
\newcommand{\mbb}[1]{\mathbb{#1}}
\newcommand{\mc}[1]{\mathcal{#1}}
\newcommand{\ra}[1]{\renewcommand{\arraystretch}{#1}}

\title{\bf Stochastic Processes: Assignment III}
\author{\bf Connor Braun}
\date{}

\begin{document} 
    \maketitle
    \noindent{\bf Problem 6} Let $\{X_n\}_{n\geq 0}$ be a Markov chain on state space $S=\{0,1,2,\dots\}$. Show that if $\{X_n\}_{n\geq 0}$ is irreducible and recurrent, then
    \[P\left(\liminf_{n\rightarrow\infty}X_n=0,\limsup_{n\rightarrow\infty} X_n=\infty\right)=1.\]
    {\bf Proof} Let $\{X_n\}_{n\geq 0}$ be an irreducible, recurrent markov chain on the nonnegative integers. We will make use of the following lemma:
    \begin{center}
        \begin{minipage}[c]{0.85\linewidth}
            {\bf Lemma} Let $\{Y_n\}_{n\geq 0}$ be a Markov chain that is irreducible and recurrent on some state space $\mc{S}$. Independent of the initial distribution, we have for $j\in\mc{S}$
            \[P(Y_n=j\;\text{for infinitely many $n$})=1.\]  
        \end{minipage}
    \end{center}\vspace{10pt}
    Let $\Omega$ be the sample space for the process $\{X_n\}_{n\geq 0}$. By the lemma, taking $\omega\in\Omega$ and choosing arbitrary $j\in S$ we get
    \[P(X_n(\omega)=j\;\text{for infinitely many $n$})=1\]
    so that for any realization of the Markov chain, $j$ occurs infinitely many times {\it w.p.} 1.
    Let us now fix some $\omega\in\Omega$ giving rise to realization $\{X_n(\omega)\}_{n\geq 0}$. Next, choose $M\in\mbb{R}$, $M\geq 0$. Then $\lceil M\rceil\in S$, so $\lceil M\rceil$ occurs in $\{X_n(\omega)\}_{n\geq 0}$ infinitely many times {\it w.p.} 1.
    That is, for any $N\geq 0$, {\it w.p.} 1 we have $\lceil M\rceil\in\{X_n(\omega)\}_{n\geq N}$ (for otherwise $\lceil M\rceil$ could only occur at most $N$ times), so
    \[\sup_{n\geq N}X_n(\omega)\geq\lceil M\rceil\geq M.\]
    But this was for arbitrary choice of $N\geq 0$, so we get
    \[\limsup_{n\rightarrow\infty}X_n(\omega)\geq M\]
    {\it w.p.} 1. Additionally, $M$ was arbitrary, so the limit supremum is almost surely unbounded and thus
    \[P(\limsup_{n\rightarrow\infty}X_n=\infty)=1.\tag{4}\]
    Additonally, we have $0\in S$, so $0$ occurs in $\{X_n(\omega)\}_{n\geq 0}$ infinitely many times {\it w.p.} 1. But for any $i\in S$, $0\leq i$, so taking $N\geq 0$, $\{X_n(\omega)\}_{n\geq N}$ is bounded below by $0$. It is easy to see that this is a greatest lower bound,
    since $0\in\{X_n(\omega)\}_{n\geq N}$ (for otherwise $0$ could occur at most only $N$ times) so for any $\varepsilon>0$, $\varepsilon$ is not a lower bound of $\{X_n(\omega)\}_{n\geq N}$. In other words, for any $N\geq 0$ we have
    \[\inf\{X_n(\omega)\}_{n\geq N}=0\]
    {\it w.p.} 1. But of course this was for arbitrary choice of $N\geq 0$, so
    \[P(\liminf_{n\rightarrow\infty}X_n=0)=1.\tag{5}\]
    To finish the proof, we make use of the following lemma:
    \begin{center}
        \begin{minipage}[c]{0.85\linewidth}
            {\bf Lemma.} Let $A,B$ be two events with $P(A)=1$ and $P(B)=1$. Then $P(A\cap B)=1$.\\[10pt]
            {\bf Proof.} The result follows directly from, DeMorgan's law, finite subadditivity and elementary properties of probability measures.
            \begin{align*}
                P(A\cap B)&=1-P(A^c\cup B^c)\tag{DeMorgan's law}\\
                &\geq1-P(A^c)-P(B^c)\tag{finite subadditivity}\\
                &=1-(1-1)-(1-1)\\
                &=1
            \end{align*}
            so $1\leq P(A\cap B)\leq 1\Rightarrow P(A\cap B)=1$.\hfill{$\qed$}
        \end{minipage}\vspace{10pt}
    \end{center}
    so that by (4) and (5) and this lemma we determine that the joint probability of interest is
    \[P(\liminf_{n\rightarrow\infty}X_n=0,\;\limsup_{n\rightarrow\infty}X_n=\infty)=1\tag*{$\qed$}\]
\end{document}