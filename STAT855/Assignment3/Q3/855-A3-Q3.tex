\documentclass[11pt, letterpaper]{article}
\usepackage[margin=1.5cm]{geometry}
\pagestyle{plain}

\usepackage{amsmath, amsfonts, amssymb, amsthm}
\usepackage{bbm}
\usepackage[shortlabels]{enumitem}
\usepackage[makeroom]{cancel}
\usepackage{graphicx}
\usepackage{xcolor}
\usepackage{array, booktabs, ragged2e}
\graphicspath{{./Images/}}

\newcommand{\bs}[1]{\boldsymbol{#1}}
\newcommand{\mbb}[1]{\mathbb{#1}}
\newcommand{\mc}[1]{\mathcal{#1}}
\newcommand{\ra}[1]{\renewcommand{\arraystretch}{#1}}

\title{\bf Stochastic Processes: Assignment III}
\author{\bf Connor Braun}
\date{}

\begin{document} 
    \maketitle
    \noindent{\bf Problem 3} Let $\{X_n,n\geq 0\}$ be a Markov chain with the state space $S$, initial distribution $\lambda$ and transition matrix $P$. Let $k\geq 1$ be a fixed integer
    and define $Y_n=X_{kn}$ for $n\geq 0$.\\[10pt]
    {\bf a)} Show $\{Y_n,n\geq 0\}$ is a Markov chain with initial distribution $\lambda$ and transition matrix $P^k$.\\[10pt]
    {\bf Solution} For all $n\geq 0$, $Y_n=X_{nk}\in S$, so denoting the state space of $\{Y_n\}_{n\geq 0}$ $S_Y$, we get $S_Y\subseteq S$, but set $S_Y=S$, accepting the possibility
    than some states may be unreachable from any others. Now take $i_0,i_1,\dots,i_{n-2},i,j\in S_Y$ and observe that
    \begin{align*}
        P(Y_n=j|Y_{n-1}=i,\dots,Y_1=i_1,Y_0=i_0)&=P(X_{kn}=j|X_{k(n-1)}=i,\dots,X_{k}=i_1,X_0=i_0)\\
        &=P(X_{kn}=j|X_{k(n-1)}=i)\\
        &=P(Y_n=j|Y_{n-1}=i)
    \end{align*}
    so, indeed, $\{Y_n\}_{n\geq 0}$ is Markovian and we need only find its initial distribution and transition matrix, which we shall indicate with $\gamma$ and $\mbb{P}$ respectively.
    Let $i,j\in S_Y$. Then
    \begin{align*}
        \mbb{P}_{ij}=P(Y_1=j|Y_0=i)=P(X_k=j|X_0=i)=(P^k)_{ij}
    \end{align*}
    the $i,j$th element of the $k$-step transition probability matrix associated to $\{X_n\}_{n\geq 0}$, so $\mbb{P}=P^k$. Similarly, letting $i\in S_Y$ we have
    \[\gamma_i=P(Y_0=i)=P(X_0=i)=\lambda_i\]
    which holds for any $i\in S_Y$, so $\gamma=\lambda$. Thus we have found that $\{Y_n\}_{n\geq 0}\sim\text{Markov}(\lambda,P^k)$.\hfill{$\qed$}\\[10pt]
    {\bf b)} Show that if $\pi$ is an invariant distribution for $\{X_n\}_{n\geq 0}$, then $\pi$ is also an invariant distribution for $\{Y_n\}_{n\geq 0}$.\\[10pt]
    {\bf Solution} Let $\pi$ be invariant for $\{X_n\}_{n\geq 0}$. Then
    \[\pi=\pi P.\]
    With this property, we find
    \[\pi P^k=(\pi P)P^{k-1}=\pi P^{k-1}=\dots=\pi P=\pi\]
    so $\pi$ is an invariant distribution for $\{Y_n\}_{n\geq 0}$.\hfill{$\qed$}
\end{document}
