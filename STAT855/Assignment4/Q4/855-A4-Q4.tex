\documentclass[11pt, letterpaper]{article}
\usepackage[margin=1.5cm]{geometry}
\pagestyle{plain}

\usepackage{amsmath, amsfonts, amssymb, amsthm}
\usepackage{bbm}
\usepackage[shortlabels]{enumitem}
\usepackage[makeroom]{cancel}
\usepackage{graphicx}
\usepackage{xcolor}
\usepackage{array, booktabs, ragged2e}
\graphicspath{{./Images/}}

\newcommand{\bs}[1]{\boldsymbol{#1}}
\newcommand{\mbb}[1]{\mathbb{#1}}
\newcommand{\mc}[1]{\mathcal{#1}}
\newcommand{\ra}[1]{\renewcommand{\arraystretch}{#1}}

\title{\bf Stochastic Processes: Assignment IV}
\author{\bf Connor Braun}
\date{}

\begin{document}    
    \maketitle
    \noindent{\bf Problem 4} A particle moves on a $d$-dimensional hypercube $\{+1,-1\}^d$, where $d\geq 2$. Define $q=(-1,-1,\dots,-1)$ and $p=-q$, the vertex opposite to $q$.\\[10pt]
    At each step, if the particle is not at $p$, it is equally likely to flip the sign of one of the coordinates in its position vector; if the particle is at $p$,
    it jumps to $q$. Denote by $X_n$ the position vector of the particle after $n$ steps.\\[10pt]
    {\bf a)} Show that $\{X_n,n\geq 0\}$ is irreducible.\\[10pt]
    {\bf Solution} Let the state space $S$ be the set of all $d$-tuples specifying vertices of the hypercube. Then, for $X\in S$, $X=(x_1,x_2,\dots,x_d)$ so that $x_i=\pm 1$, $i=1,2,\dots,d$, define the functions
    \[N_q(X)=\left\|\frac{X-q}{2}\right\|_1=\sum_{n=1}^d\left|\frac{x_n+1}{2}\right|\quad\text{and}\quad N_p(X)=\left\|\frac{X-p}{2}\right\|_1=\sum_{n=1}^d\left|\frac{x_n-1}{2}\right|\]
    where $\|\cdot\|_1$ can be understood as the taxicab norm on $\mbb{R}^d$ restricted to the set of vertices of the $d$-dimensional hypercube, and $N_p(X)$, $N_q(X)$ count the number of coordinates in which $X$ differs from $p$ and $q$ respectively.
    Now, noting that for $1\leq i\leq d$,
    \[\left|\frac{x_i+1}{2}\right|=\begin{cases}
        1\quad&\text{if $x_i=1$}\\
        0\quad&\text{if $x_i=-1$}
    \end{cases}\quad\text{and}\quad \left|\frac{x_i-1}{2}\right|=\begin{cases}
        1\quad&\text{if $x_i=-1$}\\
        0\quad&\text{if $x_i=1$}
    \end{cases}\]
    we get
    \[0=\sum_{n=1}^d0\leq\sum_{n=1}^d\left|\frac{x_n\pm1}{2}\right|\leq\sum_{n=1}^d1=d\]
    so $0\leq N_p(X),N_q(X)\leq d$. Further, $N_q(X)=0$ if and only if $x_i=-1$ $\forall i$. That is, $X=q$. Similarly,
    $N_q(X)=d$ if and only if $X=p$, $N_p(X)=0$ if and only if $X=p$, and $N_p(X)=d$ if and only if $X=q$.\\[10pt]
    Now, fixing some $X\in S\setminus\{p,q\}$, we have $0<N_p(X)<d$, so $X$ differs from $p$ in exactly $N_p(X)$ indices, call them $i_1,i_2,\dots,i_{N_p(X)}$.\\[10pt]
    \noindent We next define a new process, $\{Z_n,n\geq 0\}$ so that $\{Z_n=j\}$ is the event that we flip the $j$-th coordinate of $X_n$, and $\{Z_n,n\geq 0\}$ has state space $D=\{1,2,\dots,d\}$. Thus we get $P(Z_n=j)=p_{j}=1/d$, since the probability of flipping the $j$-th coordinate is uniformly distributed on $D$.
    With all of this, we obtain
    \begin{align*}
        P(X_{N_p(X)}=p|X_0=X)&=\sum_{z_0, z_1,\dots,z_{N_p(X)-1}\in D}P(X_{N_p(X)}=p,Z_{N_p(X)-1}=z_{N_p(X)-1},\dots,Z_1=z_1,Z_0=z_0|X_0=X)\\
        &\geq P(X_{N_p(X)}=p,Z_{N_p(X)-1}=i_{N_p(X)},\dots,Z_1=i_2,Z_0=i_1|X_0=X).
    \end{align*}
    For notational expedience, let $\mc{Z}_{X,p}=\{Z_{N_p(X)-1}=i_{N_p(X)},\dots,Z_1=i_2,Z_0=i_1\}$. Then the above becomes
    \begin{align*}
        P(X_{N_p(X)}=p|X_0=X)&\geq P(X_{N_p(X)}=p,\mc{Z}_{X,p}|X_0=X)\\
        &=P(X_{N_p(X)}=p|\mc{Z}_{X,p},X_0=X)P(\mc{Z}_{X,p}).\tag{7}
    \end{align*}
    Next, note that $\mc{Z}_{X,p}\cup \{X_0=X\}\subseteq \{X_{N_p(X)}=p\}$. To see this, suppose that $\omega\in\mc{Z}_{X,p}\cup \{X_0=X\}$. Then $X$ differs from $p$ in coordinates $i_1,i_2,\dots,i_{N_p(X)}$. But $Z_0=i_1$ yields $X_1$ which differs
    from $p$ only in coordinates $i_2,\dots,i_{N_p(X)}$. Continuing in this way, we see
    \begin{align*}
        N_p(X_1)&=N_p(X)-1\\
        N_p(X_2)&=N_p(X)-2\\
        &\vdots\\
        N_p(X_{N_p(X)})&=N_p(X)-N_p(X)\\
        &=0.
    \end{align*}
    But $N_p(X_{N_p(X)})=0$ if and only if $X_{N_p(X)}=p$, so $\omega\in\{X_{N_p(X)}=p\}$, and $\mc{Z}_{X,p}\cup \{X_0=X\}\subseteq \{X_{N_p(X)}=p\}$ as claimed. But this gives us $P(X_{N_p(X)}=p|\mc{Z}_{X,p},X_0=X)=1$ so, continuing from (7), we thus have
    \begin{align*}
        P(X_{N_p(X)}=p|X_0=X)&\geq P(X_{N_p(X)}=p|\mc{Z}_{X,p},X_0=X)P(\mc{Z}_{X,p})\\
        &=P(Z_{N_p(X)-1}=i_{N_p(X)},\dots,Z_1=i_2,Z_0=i_1)\\
        &=\left(1/d\right)^{N_p(X)}\tag{since $Z_i\perp Z_j$ whenever $i\neq j$}\\
        &>0.
    \end{align*}
    So we have that $X\longrightarrow p$. Of course, $P(X_1=q|X_0=p)=1>0$, so $p\longrightarrow q$ as well. Observe now that $X\neq q$, so $0<N_q(X)$, and $X$ differs from $q$ in exactly $N_q(X)$ coordinates,
    call them $j_1,j_2,\dots j_{N_q(X)}$. Defining $\mc{Z}_{q,X}=\{Z_{N_q(X)-1}=j_{N_q(X)},\dots,Z_1=j_2,Z_0=j_1\}$, we get $\mc{Z}_{q,X}\cup\{X_0=q\}\subseteq \{X_{N_q(X)}=X\}$ just as before. To see this, take $\omega\in\mc{Z}_{q,X}\cup\{X_0=q\}$. Then we get
    $X_0=q$, which differs from $X$ at indices, $j_1,j_2,\dots,j_{N_q(X)}$, but $Z_0=j_1$ means that $X_1$ differs only at indices $j_2,\dots,j_{N_q(X)}$, and $Z_1=j_2$ means that $X_2$ differs from $X$ only in coordinates $j_3,\dots,j_{N_q(X)}$. Continuing in this way, we get
    $X_{N_q(X)}$ differs from $X$ not at all, so $X_{N_q(X)}=X$  and thus $\omega\in \{X_{N_q(X)}=X\}$ implying that $\mc{Z}_{q,X}\cup\{X_0=q\}\subseteq\{X_{N_q(X)}=X\}$, as claimed.\\[10pt]
    Then we can show that $q$ leads to $X$ just as we showed $X$ leads to $p$:
    \begin{align*}
        P(X_{N_q(X)}=X|X_0=q)&=\sum_{z_0,z_1,\dots,z_{N_q(X)-1}\in D}P(X_{N_q(X)}=X,Z_{N_q(X)-1}=z_{N_q(X)-1},\dots,Z_1=z_1,Z_0=z_0|X_0=q)\\
        &\geq P(X_{N_q(X)}=X,\mc{Z}_{q,X}|X_0=q)\\
        &=P(X_{N_q(X)}=X|\mc{Z}_{q,X},X_0=q)P(\mc{Z}_{q,X})\tag{since $P(X_{N_q(X)}=X|\mc{Z}_{q,X},X_0=q)=1$}\\
        &=P(Z_{N_q(X)-1}=j_{N_q(X)},\dots,Z_1=j_2,Z_0=j_1)\\
        &=(1/d)^{N_q(X)}\tag{since $Z_i\perp Z_j$ whenever $i\neq j$}\\
        &>0.
    \end{align*}
    This tells us that $q\longrightarrow X$. So $X\longrightarrow p$ and $q\longrightarrow X$ so by transitivity of the 'leads to' relation, $q\longleftrightarrow p$ and $X\longleftrightarrow p$. But this was for arbitrary $X\in S\setminus\{p,q\}$, so now taking $A,B\in S$, we have that $A\longleftrightarrow p$, and $B\longleftrightarrow p$,
    so by the transitivity of the communication relation, $A\longleftrightarrow B$, and thus the state space consists of a single communicating class and the process is irreducible.\hfill{$\qed$}\\[10pt]
    {\bf b)} Determine the period of $\{X_n,n\geq 0\}$.\\[10pt]
    {\bf Solution} Some preliminaries to get us started. First, observe that in order to begin at one state and return to it, we must have flipped an even number of coordinates. 
    That is, for every coordinate changed between returns to a state $X\in S$, we must revert each of them to be at $X$ again, so 2 divides the number of coordinate changes.\\[10pt]
    Next, we associate a number of coordinate changes to each step in $\{X_n,n\geq 0\}$. If $X_i\in S$, $X_i\neq p$ for some $i\geq 0$, then $X_{i+1}$ will differ by
    exactly one coordinate. If instead $X_i=p$, then $X_{i+1}=q$, and the two differ in $d$ coordinates, the dimension of the hypercube.\\[10pt]
    Let $P$ be the one-step transition matrix for this process so that $\forall i,j\in S$, $P_{i,j}=P(X_1=j|X_0=i)$. Define $\mc{T}_q=\{n>0:(P^n)_{qq}>0\}$, so that the period of $q\in S$ is $\gcd\mc{T}_q$. We proceed by induction to show that $2k\in\mc{T}_q$ $\forall k\in\mbb{N}$.\\[10pt]
    Let $k=1$, and $\ell\in S$ so that $\ell$ and $q$ differ in just one coordinate, so $P(X_1=\ell|X_0=q)=1/d$ and $P(X_1=q|X_0=\ell)=1/d$. Then by the Chapman-Kolmogorov equations
    \[(P^{2k})_{q,q}=(P^2)_{q,q}=\sum_{j\in S}P_{q,j}P_{j,q}\geq P_{q,\ell}P_{\ell,q}=(1/d)^2>0\]
    so $2\cdot1\in\mc{T}_q$. Now, for the inductive hypothesis, suppose that for some $k\geq 1$ we have $(P^{2k})_{q,q}>0$. This gives us
    \[(P^{2(k+1)})_{q,q}=(P^{2k+2})_{q,q}=\sum_{j\in S}(P^{2k})_{q,q}(P^2)_{q,q}\geq(P^{2k})_{q,q}(1/d)^2>0\]
    so $2k\in\mc{T}_q\Rightarrow 2(k+1)\in\mc{T}_q$, so by the principle of mathematical induction, for every $n$ even, $n\in\mc{T}_q$.\\[10pt]
    We now consider the cases when $d$ is even versus odd separately. First, suppose that $d$ is even, so $\exists k\in\mbb{N}:$ $d=2k$. Then $d\in\mc{T}_q$ by the previous result, but also
    \begin{align*}
        (P^{d+1})_{q,q}=\sum_{j\in S}(P^d)_{q,j}(P)_{j,q}&\geq(P^d)_{q,p}(P)_{p,q}\\
        &=(P^d)_{q,p}\tag{since $P_{p,q}=1$}\\
        &=\sum_{z_0,z_1,\dots,z_{d-1}\in D}P(X_d=p,Z_{d-1}=z_{d-1},\dots,Z_1=z_1,Z_0=z_0|X_0=q)\\
        &\geq P(X_d=p,Z_{d-1}=d,\dots,Z_1=2,Z_1=1|X_0=q)\\
        &=P(X_d=p|Z_{d-1}=d,\dots,Z_1=2,Z_0=1,X_0=q)P(Z_{d-1}=d,\dots,Z_1=2,Z_0=1)
    \end{align*} 
    where we have reintroduced the process $\{Z_n,n\geq0\}$ from part a. We can also apply the same reasoning as in part a to show that the conditional probability in the last line is 1. In particular, it is clear that if $X_0=q$, and
    we flip the first coordinate, then the second, and continue until we've flipped the $d$th coordinate, that we will be at $p$. Thus the above becomes
    \begin{align*}
        (P^{d+1})_{q,q}\geq P(Z_{d-1}=d,\dots,Z_1=2,Z_0=1)&=(1/d)^d>0
    \end{align*}
    so $d+1\in\mc{T}_q$. But the only number dividing $d$ and $d+1$ is $1$, so $\gcd\mc{T}_q=1$ and the state $q$ is aperiodic. By part a, this process is irreducible, so in fact every
    element of $S$ shares the same period -- that is, period is a class property (see problem 6 for proof). Thus when $d$ is even, the chain is aperiodic.\\[10pt]
    Now consider the case of $d$ odd. Previously, we commented that this implies that the transition from $p$ to $q$ is associated with an odd number of coordinate changes, along with every other
    possible state transition.
    Assume for the purpose of deriving a contradiction that $(P^{2k+1})_{q,q}>0$ for some $k\in\mbb{N}$. Then there exists a sequence of $2k+1$ transitions from state $q$ back to itself. Let $c_i$ be the number of coordinate changes associated with the
    $i$th of these transitions for $i=1,2,\dots,2k+1$. Then $c_i$ is odd number, so $\exists k_i\in\mbb{N}:$ $c_i=2k_i+1$ for $i=1,2,\dots,2k+1$, and the total number of coordinate changes during the excursion is given by
    \[\sum_{i=1}^{2k+1}c_i=\left(\sum_{i=1}^{2k+1}2k_i+1\right)=\left(\sum_{i=1}^{2k+1}2k_i\right)+2k+1=2\left[\left(\sum_{i=1}^{2k+1}k_i\right)+k \right]+1\]
    so any path of an odd number of state transitions from $q$ to itself must incur an odd number of coordinate changes, and thus cannot terminate at $q$. Thus, $\forall n$ odd, $n\notin\mc{T}_q$, but $n-1\in\mc{T}_q$, so $\gcd\mc{T}_q=2$. Thus, when $d$ is odd, the period of $q\in S$ is two.
    Once more appealing to the irreducibility of the process from a and the forthcoming result in problem 6, we get that $\forall i\in S$, the period of $i$ is also 2, since period is a class property.\hfill{$\qed$}
\end{document}