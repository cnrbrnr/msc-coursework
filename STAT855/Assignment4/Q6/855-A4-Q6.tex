\documentclass[11pt, letterpaper]{article}
\usepackage[margin=1.5cm]{geometry}
\pagestyle{plain}

\usepackage{amsmath, amsfonts, amssymb, amsthm}
\usepackage{bbm}
\usepackage[shortlabels]{enumitem}
\usepackage[makeroom]{cancel}
\usepackage{graphicx}
\usepackage{xcolor}
\usepackage{array, booktabs, ragged2e}
\graphicspath{{./Images/}}

\newcommand{\bs}[1]{\boldsymbol{#1}}
\newcommand{\mbb}[1]{\mathbb{#1}}
\newcommand{\mc}[1]{\mathcal{#1}}
\newcommand{\ra}[1]{\renewcommand{\arraystretch}{#1}}

\title{\bf Stochastic Processes: Assignment IV}
\author{\bf Connor Braun}
\date{}

\begin{document}    
    \maketitle
    \noindent{\bf Problem 6} Let $\{X_n:n\geq 0\}$ be a Markov chain with the state space $S$ and transition matrix $P$. We denote the period of a state $i\in S$ by
    \[d_i=\gcd\{n\geq 1:(P^n)_{i,i}>0\}.\]
    Show that for any $i,j\in S$, if $i$ and $j$ communicate, then $d_i=d_j$.\\[10pt]
    {\bf Proof} Let $i,j\in S$ and suppose $i\neq j$ (since otherwise the proof is trivial). Let $d_i$ be the period of $i$ and $d_j$ the period of $j$. Since $i\longleftrightarrow j$, $\exists a,b\in\mbb{N}$ so that
    $(P^{a})_{i,j}>0$ and $(P^{b})_{j,i}>0$. By the Chapman-Kolmogorov equations,
    \[(P^{a+b})_{i,i}=\sum_{k\in S}(P^{a})_{i,k}(P^{b})_{k,i}\geq (P^{a})_{i,j}(P^{b})_{j,i}>0\]
    So $a+b\in\{n\geq 1:(P^n)_{i,i}>0\}$, and thus $d_i|(a+b)$ (where $\forall \alpha,\beta\in\mbb{N}$, we write $\alpha|\beta$ to mean $\exists k\in\mbb{N}:$ $\beta=k\alpha$ ).
    Now, since $j\longleftrightarrow i$ and $i\longleftrightarrow j$, $\exists {c_j}>0:$ $(P^{c_j})_{j,j}>0$. Then
    \begin{align*}
        (P^{a+c_j+b})_{i,i}=\sum_{k\in S}(P^a)_{i,k}(P^{c_j+b})_{k,i}&=\sum_{k\in S}(P^a)_{i,k}\left(\sum_{\ell\in S}(P^{c_j})_{k,\ell}(P^{b})_{\ell, i}\right)\\
        &\geq\sum_{k\in S}(P^a)_{i,k}(P^{c_j})_{k,j}(P^b)_{j,i}\\
        &\geq (P^a)_{i,j}(P^{c_j})_{j,j}(P^b)_{j,i}\\
        &>0
    \end{align*}
    so that we also get $a+b+{c_j}\in\{n\geq 1:(P^n)_{i,i}>0\}$, thus $d_i|(a+b+{c_j})$. From this, $\exists k,m\in\mbb{N}$ so that $(a+b)=kd_i$, and $(a+b+{c_j})=md_i$. But then
    \[(a+b+{c_j})=md_i\quad\Rightarrow\quad {c_j}+kd_i=md_i\quad\Rightarrow\quad {c_j}=d_i(m-k)\]
    which gives us $d_i|{c_j}$. But $d_j|{c_j}$ by definition, and is the greatest such divisor, so $d_i\leq d_j$. We now reverse this argument. Observe that
    \[(P^{b+a})_{j,j}=\sum_{k\in S}(P^{b})_{j,k}(P^{a})_{k,j}\geq (P^{b})_{j,i}(P^{a})_{i,j}>0\]
    so $b+a\in\{n\geq 1: (P^n)_{j,j}>0\}$, and thus $d_j|(b+a)$. Next, since $i\longleftrightarrow j$ and $j\longleftrightarrow i$, $\exists c_i>0:$ $(P^{c_i})_{i,i}>0$. Then
    \begin{align*}
        (P^{b+c_i+a})_{j,j}=\sum_{k\in S}(P^b)_{j,k}(P^{c_i+a})_{k,j}&=\sum_{k\in S}(P^b)_{j,k}\left(\sum_{\ell\in S}(P^{c_i})_{k,\ell}(P^{a})_{\ell, j}\right)\\
        &\geq\sum_{k\in S}(P^b)_{j,k}(P^{c_i})_{k,i}(P^a)_{i,j}\\
        &\geq (P^b)_{j,i}(P^{c_i})_{i,i}(P^a)_{i,j}\\
        &>0
    \end{align*} 
    so that $a+b+c_i\in\{n\geq 1:(P^n)_{j,j}>0\}$, thus $d_j|(a+b+c_i)$, and we know now that $\exists \ell,n\in\mbb{N}$ so that $(a+b)=\ell d_j$ and $(a+b+c_i)=nd_j$.  
    But this gives us
    \[(a+b+c_i)=nd_j\quad\Rightarrow\quad c_i+\ell d_j=nd_j\quad\Rightarrow\quad c_i=d_j(n-\ell)\]
    which gives us $d_j|c_i$. But $d_i|c_i$, and is the greatest such divisor, so $d_j\leq d_i$. Previously we found that $d_i\leq d_j$, so we are forced to conclude that $d_i=d_j$.
    Thus, for any two states $i,j\in S$ with $i\longleftrightarrow j$, we have $d_i=d_j$, so period is a class property.\hfill{$\qed$}
\end{document}