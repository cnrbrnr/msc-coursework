\documentclass[11pt, letterpaper]{article}
\usepackage[margin=1.5cm]{geometry}
\pagestyle{plain}

\usepackage{amsmath, amsfonts, amssymb, amsthm}
\usepackage{bbm}
\usepackage[shortlabels]{enumitem}
\usepackage[makeroom]{cancel}
\usepackage{graphicx}
\usepackage{xcolor}
\usepackage{array, booktabs, ragged2e}
\graphicspath{{./Images/}}

\newcommand{\bs}[1]{\boldsymbol{#1}}
\newcommand{\mbb}[1]{\mathbb{#1}}
\newcommand{\mc}[1]{\mathcal{#1}}
\newcommand{\ra}[1]{\renewcommand{\arraystretch}{#1}}

\title{\bf Stochastic Processes: Assignment IV}
\author{\bf Connor Braun}
\date{}

\begin{document}    
    \maketitle
    \noindent{\bf Problem 2} Each morning a student takes one of three books he owns from his shelf. The probability that he chooses book 1, 2, 3 is, respectively, $1/4$, $1/2$, $1/4$.
    In the evening, with $90\%$ probability, he replaces the book in the middle of the two books on the shelf, and with $10\%$ probability, he sorts the books in the ordering $1,2,3$ from left to right.
    Assume that the choices on successive days are independent, and denote by $X_n$ the ordering of the books from left to right at the end of day $n\geq 0$.\\[10pt]
    {\bf a)} Identify the state space and transition probabilities for $\{X_n,n\geq 0\}$.\\[10pt]
    {\bf Solution} For $i,j,k\in\{1,2,3\}$ distinct, let $(i,j,k)$ denote an ordering on the shelf, so that $X_n=(i,j,k)$ is the ordering on the shelf at the end of the $n$th day, $n\geq 0$.
    Then the state space $S$ is the set of permutations of $\{1,2,3\}$, so $|S|=3!=6$. We depict the state transition diagram in figure 2 below. 
    \begin{center}
        \makebox[\textwidth]{\includegraphics[width=80mm]{855-A4-Q2-fig.png}}
    \end{center}
    {\bf Figure 2} State transition diagram for the take-a-book-leave-a-book process. Arrows indicate transitions with nonzero probability.
    All states can transition to themselves in one step with nonzero probability (not shown).\\[10pt]
    For the explicit transition probabilities, let $p_i$ be the probability of drawing the $i$th book on a given morning, so that 
    \[p_1=1/4\quad p_2=1/2\quad\text{and}\quad p_3=1/4.\]
    Every state can transition back to $(1,2,3)$ with probability $1/10$, since for $(i,j,k)\in S$,
    \[\sum_{\ell\in S}p_\ell \cdot\frac{1}{10}=\frac{1}{10}\]
    is the probability of choosing any of the three books, and then resorting them at the end of the day. Let $\mc{S}$ be an alternative state space for the process, defined by a map $\mu:S\rightarrow\mc{S}$ such that
    \[\mu((1,2,3))=1,\quad\mu((1,3,2))=2,\quad\mu((3,1,2))=3,\quad\mu((3,2,1,))=4,\quad \mu((2,3,1))=5\quad\text{and}\quad\mu((2,1,3))=6\]
    which allows us to more intuitively express the transition matrix for this system by 
    \begin{align*}
        P&=\begin{pmatrix}
            (9/10)p_2+1/10 & (9/10)p_3 & 0 & 0 & 0 & (9/10)p_1\\
            (9/10)p_2+1/10 & (9/10)p_3 & (9/10)p_1 & 0 & 0 & 0\\
            1/10 & (9/10)p_3 & (9/10)p_1 & (9/10)p_2 & 0 & 0\\
            1/10 & 0 & (9/10)p_1 & (9/10)p_2 & (9/10)p_3 & 0\\
            1/ 10 & 0 & 0 & (9/10)p_2 & (9/10)p_3 & (9/10)p_1\\
            (9/10)p_2+1/10 & 0 & 0 & 0 & (9/10)p_3 & (9/10)p_1
        \end{pmatrix}\\
        &=\begin{pmatrix}
            22/40 & 9/40 & 0 & 0 & 0 & 9/40\\
            22/40 & 9/40 & 9/40 & 0 & 0 & 0\\
            4/40 & 9/40 & 9/40 & 18/40 & 0 & 0\\
            4/40 & 0 & 9/40 & 18/40 & 9/40 & 0\\
            4/40 & 0 & 0 & 18/40 & 9/40 & 9/40\\
            22/40 & 0 & 0 & 0 & 9/40 & 9/40
        \end{pmatrix}.\tag*{$\qed$}
    \end{align*}
    {\bf b)} Prove that $\{X_n,n\geq 0\}$ is irreducible and determine it's period.\\[10pt]
    {\bf Solution} From figure 2 it is clear that the system is irreducible, since there is a path of nonzero probability from any state to any other. Additionally, for $i\in\mc{S}$ we have
    $P_{i,i}>0$ so that $1\in\{n>0:(P^n)_{i,i}>0\}$, so
    \[d_i=\gcd\{n>0:(P^n)_{i,i}>0\}=1.\]
    But this was for arbitrary $i\in\mc{S}$, so we conclude that ever state of this process has period 1 -- that is, every state is aperiodic.\\[10pt]
    {\bf c)} Compute the invariant distribution for $\{X_n,n\geq 0\}$ numerically.\\[10pt]
    {\bf Solution} Please see Appendix A.2 for the code used to do this. Briefly, we seek the right eigenvector of $P^T$ corresponding to eigenvalue $1$. Denoting it $\pi=\{\pi_i,i\in\mc{S}\}$ we find\\[10pt]
    {\bf Table 1} Invariant distribution computed numerically for the take-a-book-leave-a-book process with transition matrix as given above. See Appendix A.2 for the program used
    to generate this result. Values are rounded up to five points of precision.
    \begin{center}
        \begin{tabular}{@{}lc|c|c|c|c|c@{}}\toprule
                & $\pi_1$ & $\pi_2$ & $\pi_3$ & $\pi_4$ & $\pi_5$ & $\pi_6$\\
                \midrule
                &0.41889 & 0.14487 & 0.08013 & 0.13112 & 0.08013 & 0.14487\\
            \bottomrule
        \end{tabular}
    \end{center}
    {\bf d)} Assuming the initial ordering is $(1,2,3)$ from left to right, what is the expected time that the ordering will return to this state?\\[10pt]
    {\bf Solution} Our goal is to compute the expected value of the first passage time corresponding to the state $(1,2,3)\in S$. Let $T=\inf\{n>0:X_n=(1,2,3)\}$ be this passage time. 
    Then we aim to compute $\mbb{E}_{(1,2,3)}[T]$. However, since $P$ is irreducible and has an invariant distribution $\pi$, we have the formula
    \[\pi_i=\frac{1}{\mbb{E}_i[T_i^{(1)}]}\]
    for $i\in S$, where $T_i^{(1)}=\inf\{n>0:X_n=i\}$. But then we can compute the result directly:
    \[E_{(1,2,3)}[T]=\frac{1}{\pi_1}=2.38726,\quad\text{rounded to 5 points of precision.\tag*{$\qed$}}\]
    {\bf\Large Appendix}\\[10pt]
    {\bf A.2 Code used for problem 2c}
    \begin{verbatim}
        # Specify state transition probability matrix
        P = np.array(
            [
                22/40, 9/40, 0, 0, 0, 9/40,
                22/40, 9/40, 9/40, 0, 0, 0,
                4/40, 9/40, 9/40, 18/40, 0, 0,
                4/40, 0, 9/40, 18/40, 9/40, 0,
                4/40, 0, 0, 18/40, 9/40, 9/40,
                22/40, 0, 0, 0, 9/40, 9/40
            ]
        ).reshape((6,6))
        print(P, '\n') # Print it to make sure the entries are correct

        # Get left spectrum and eigenvectors
        eigenvalues, eigenvectors = np.linalg.eig(P.T)

        # Find an invariant eigenvector from the candidates
        tol = 1e-13 # L1 norm tolerance for verifying invariance
        D = [] # list to contain norms
        for i in range(5):
            # Compute distance between v^T and P^Tv^t for each eigenvector v
            D.append(np.sum(np.abs(P.T @ eigenvectors[:, i] - eigenvectors[:, i])))

        # Search for an approximately invariant vector w.r.t. tolerance
        if np.amin(D) > tol:
            raise Exception("No invariant eigenvectors") # Halt if no approximate invariance
        else:
            inv_ind = np.argmin(D) # index of invariant eigenvector

            # Print spectrum and invariant eigenvector
            print('Eigenvalues: {}'.format(eigenvalues))
            print('Invariant Eigenvector: {}\n'.format(eigenvectors[:, inv_ind]))

        # Extract and normalize the invariant measure to identify invariant distribution
        pi = eigenvectors[:, 0] / np.sum(eigenvectors[:, 0])
        print("Invariant distribution: {}\nSums to {}".format(pi, np.sum(pi)))
    \end{verbatim}
\end{document}