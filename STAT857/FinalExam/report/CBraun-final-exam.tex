\documentclass[10pt]{article}
\usepackage[margin=1.3cm]{geometry}

% Packages
\usepackage{amsmath, amsfonts, amssymb, amsthm}
\usepackage{bbm} 
\usepackage{dutchcal} % [dutchcal, calrsfs, pzzcal] calligraphic fonts
\usepackage{graphicx}
\usepackage[T1]{fontenc}
\usepackage[tracking]{microtype}

% Palatino for text goes well with Euler
\usepackage[sc,osf]{mathpazo}   % With old-style figures and real smallcaps.
\linespread{1.025}              % Palatino leads a little more leading

% Euler for math and numbers
\usepackage[euler-digits,small]{eulervm}

% Command initialization
\DeclareMathAlphabet{\pazocal}{OMS}{zplm}{m}{n}
\graphicspath{{./images/}}

% Custom Commands
\newcommand{\bs}[1]{\boldsymbol{#1}}
\newcommand{\E}{\mathbb{E}}
\newcommand{\var}[1]{\text{Var}\left(#1\right)}
\newcommand{\bp}[1]{\left({#1}\right)}
\newcommand{\mbb}[1]{\mathbb{#1}}
\newcommand{\1}[1]{\mathbbm{1}_{#1}}
\newcommand{\mc}[1]{\mathcal{#1}}
\newcommand{\nck}[2]{{#1\choose#2}}
\newcommand{\pc}[1]{\pazocal{#1}}
\newcommand{\ra}[1]{\renewcommand{\arraystretch}{#1}}
\newcommand*{\floor}[1]{\left\lfloor#1\right\rfloor}
\newcommand*{\ceil}[1]{\left\lceil#1\right\rceil}

\DeclareMathOperator{\Var}{Var}
\DeclareMathOperator{\Cov}{Cov}
\DeclareMathOperator{\diag}{diag}
\DeclareMathOperator{\argmin}{arg\,min}
\DeclareMathOperator{\sgm}{sgm}

\newtheorem{theorem}{Theorem}
\newtheorem{lemma}{Lemma}

\begin{document}
    \begin{center}
        {\bf\large{MATH 857: STATISTICAL LEARNING II}}
        \smallskip
        \hrule
        \smallskip
        {\bf FINAL EXAM} \hfill {\bf Connor Braun} \hfill {\bf 2024-04-16}
    \end{center}
{\bf Problem}\hspace{5pt} Consider the following model:
\begin{align*}
    &\Delta\sim\text{Bernoulli}(p)\\
    &Y|\Delta=1\sim\mc{N}(\mathbbm{1}_{(x>c_1)},\sigma^2)\\
    &Y|\Delta=0\sim\mc{N}(2\mathbbm{1}_{(x>c_0)}, \sigma^2).
\end{align*}
In the model, $Y$ is the response, $x$ is a nonrandom feature and $\Delta$ is a latent variable. The parameters are
$c_1,c_0\in\mbb{R}$, $\sigma^2>0$ and $p\in(0,1)$.
Let $\mc{D}=\{(x_i,Y_i)\}_{i=1}^n$ be the observed data and $\mc{D}_c=\{(x_i,Y_i,\Delta_i)\}_{i=1}^n$ the complete data.\\[5pt]
{\bf a)}\hspace{5pt} Write down the complete data likelihood and complete data log-likelihood.\\[5pt]
{\bf Solution}\hspace{5pt}Let $\theta=(c_1,c_0,\sigma^2,p)$. Since $x$ is deterministic, for $i=1,\dots, n$ we may write the density
\begin{align*}
    f_{Y,x}(Y_i,x)=f_Y(Y_i|\theta, x)\mathbbm{1}_{(x=x_i)}=\begin{cases}
        0,\quad&\text{if $x\neq x_i$}\\
        f_Y(Y_i|\theta,x_i)\quad&\text{o.w.}
    \end{cases}
\end{align*}
and further
\begin{align*}
    f_Y(Y_i|\theta,x_i)=\sum_{k=0,1}f_Y(Y_i,\Delta=k|\theta,x_i)&=f_Y(Y_i|\Delta=k,\theta,x_i)f_\Delta(k)\\
    &=\frac{(1-p)}{\sqrt{2\pi\sigma^2}}\exp\bp{-\tfrac{1}{2\sigma^2}(Y_i-2\mathbbm{1}_{(x_i>c_0)})^2}+\frac{p}{\sqrt{2\pi\sigma^2}}\exp\bp{-\tfrac{1}{2\sigma^2}(Y_i-\mathbbm{1}_{(x_i>c_1)})^2}
\end{align*}
which reveals the form of the joint density $f_{Y,x,\Delta}$:
\begin{align*}
    f_{Y,x,\Delta}(Y_i,x,0)&=\begin{cases}
        0,\quad&\text{if $x\neq x_i$}\\
        \frac{(1-p)}{\sqrt{2\pi\sigma^2}}\exp\bp{-\tfrac{1}{2\sigma^2}(Y_i-2\mathbbm{1}_{(x_i>c_0)})^2}\quad&\text{o.w.}
    \end{cases}\\
    f_{Y,x,\Delta}(Y_i,x,1)&=\begin{cases}
        0,\quad&\text{if $x\neq x_i$}\\
        \frac{p}{\sqrt{2\pi\sigma^2}}\exp\bp{-\tfrac{1}{2\sigma^2}(Y_i-\mathbbm{1}_{(x_i>c_1)})^2}\quad&\text{o.w.}
    \end{cases}
\end{align*}
and thus the complete data likelihood function is given by
\begin{align*}
    \mc{L}_c(\theta|\mc{D}_c)=\prod_{i=1}^nf_{Y,x,\Delta}(Y_i,x_i,\Delta_i)=\prod_{i=1}^n\bp{\frac{(1-p)}{\sqrt{2\pi\sigma^2}}\exp\bp{-\tfrac{1}{2\sigma^2}(Y_i-2\mathbbm{1}_{(x_i>c_0)})^2}}^{1-\Delta_i}\bp{\frac{p}{\sqrt{2\pi\sigma^2}}\exp\bp{-\tfrac{1}{2\sigma^2}(Y_i-\mathbbm{1}_{(x_i>c_1)})^2}}^{\Delta_i}
\end{align*}
with corresponding complete data log-likelihood
\begin{align*}
    \ell_c(\theta|\mc{D}_c)=\sum_{i=1}^n(1-\Delta_i)\bp{\log(\phi(Y_i,2\mathbbm{1}_{(x_i>c_0)},\sigma^2))+\log(1-p)}+\Delta_i\bp{\log(\phi(Y_i,\mathbbm{1}_{(x_i>c_1)},\sigma^2))+\log(p)}
\end{align*}
where we have taken $\phi(y,\mu,\sigma^2)$ to be the Gaussian density evaluated at $y$ with mean $\mu$ and variance $\sigma^2$ as shorthand.\hfill{$\qed$}\\[5pt]
{\bf b)} Write down the observed data log-likelihood.\\[5pt]
{\bf Solution}\hspace{5pt} Using our work from the previous part, the observed data log-likelihood function $\ell$ is given by
\begin{align*}
    \ell(\theta|\mc{D})=\log\left(\prod_{i=1}^nf_Y(Y_i|\theta,x_i)\right)=\sum_{i=1}^n\log\left(\frac{(1-p)}{\sqrt{2\pi\sigma^2}}\exp\bp{-\tfrac{1}{2\sigma^2}(Y_i-2\mathbbm{1}_{(x_i>c_0)})^2}+\frac{p}{\sqrt{2\pi\sigma^2}}\exp\bp{-\tfrac{1}{2\sigma^2}(Y_i-\mathbbm{1}_{(x_i>c_1)})^2}\right)\tag*{$\qed$}
\end{align*}
{\bf c)} Write down the E-step and M-step of the EM algorithm.\\[5pt]
{\bf Solution}\hspace{5pt} We shall begin with the E-step. Let $\theta^\prime=(c_1^\prime,c_0^\prime,\sigma^{2\prime},p^\prime)$ denote the current parameter estimate.  Proceding by direct computation, we have
\begin{align*}
    Q(\theta|\theta^\prime)&=\E(\ell_c(\theta|\mc{D}_c)|\mc{D},\theta^\prime)\\
    &=\sum_{i=1}^n(1-\E(\Delta_i|\mc{D},\theta^\prime))\bp{\log(\phi(Y_i,2\mathbbm{1}_{(x_i>c_0)},\sigma^2))+\log(1-p)}+\E(\Delta_i|\mc{D},\theta^\prime)\bp{\log(\phi(Y_i,\mathbbm{1}_{(x_i>c_1)},\sigma^2))+\log(p)}.
\end{align*}
and we remind the reader that $\phi(y,\mu,\sigma^2)$ denotes the Gaussian density evaluated at $y$ with mean $\mu$ and variance $\sigma^2$. Now, since $\Delta_i\overset{i.i.d.}{\sim}\text{Bernoulli}(p)$, we get
\begin{align*}
    \E(\Delta_i|\mc{D},\theta^\prime)&=P(\Delta_i=1|\mc{D},\theta^\prime)=\frac{\phi(Y_i,\mathbbm{1}_{(x_i>c_1^\prime)},\sigma^{2\prime})p^\prime}{\phi(Y_i,\mathbbm{1}_{(x_i>c_1^\prime)},\sigma^{2\prime})p^\prime + \phi(Y_i,2\mathbbm{1}_{(x_i>c_0^\prime)},\sigma^{2\prime})(1-p^\prime)}.\tag{Baye's theorem}
\end{align*}
Now, denoting $\hat{\xi}_i:=\E(\Delta_i|\mc{D},\theta^\prime)$ as given above, we may return to the expression for $Q$ and write
\begin{align*}
    Q(\theta|\theta^\prime)=\sum_{i=1}^n(1-\hat{\xi}_i)\bp{\log(\phi(Y_i,2\mathbbm{1}_{(x_i>c_0)},\sigma^2))+\log(1-p)}+\hat{\xi}_i\bp{\log(\phi(Y_i,\mathbbm{1}_{(x_i>c_1)},\sigma^2))+\log(p)}
\end{align*}
which defines the E-step. Algorithmically, we update $\hat{\xi}_i$ for $i=1,\dots,n$ at each step using the latest parameter estimates, and these will then be used to further update the parameter estimates. For the M-step, we begin by finding an expression for $\hat{p}$ which maximizes $Q$ in $p$. By differentiating,
\begin{align*}
    \frac{\partial}{\partial p}Q(\theta|\theta^\prime)=\sum_{i=1}^n-\frac{1-\hat{\xi}_i}{1-p}+\frac{\hat{\xi}_i}{p}\quad\Rightarrow\quad\frac{\sum_{i=1}^n\hat{\xi}_i}{\hat{p}}=\frac{n-\sum_{i=1}^n\hat{\xi}_i}{1-\hat{p}}\quad\Rightarrow\quad n\hat{p}=\sum_{i=1}^n\hat{\xi}_i\quad\Rightarrow\quad \hat{p}=\frac{1}{n}\sum_{i=1}^n\hat{\xi}_i.
\end{align*} 
In a similar fashion we may determine a maximizer $\hat{\sigma}^2$ for $Q$ in $\sigma^2$. Before that though, we have 
\[\log(\phi(y,\mu,\sigma^2))=-\log(\sqrt{2\pi})-\frac{1}{2}\log(\sigma^2)-\frac{1}{2\sigma^2}(y-\mu)^2\]
which depends on $\sigma^2$ only through the second and third terms. Thus,
\begin{align*}
    \frac{\partial}{\partial\sigma^2}Q(\theta|\theta^\prime)&=\sum_{i=1}^n-(1-\hat{\xi}_i)\bp{\frac{1}{2\sigma^2}-\frac{(Y_i-2\mathbbm{1}_{(x_i>c_0)})^2}{2\sigma^4}}-\hat{\xi}\bp{\frac{1}{2\sigma^2}-\frac{(Y_i-\mathbbm{1}_{(x_i>c_1)})^2}{2\sigma^4}}\\
    &=\sum_{i=1}^n\frac{1}{2\sigma^4}\left[(\hat{\xi}_i-1)\bp{\sigma^2-(Y_i-2\mathbbm{1}_{(x_i>c_0)})^2}-\hat{\xi}\bp{\sigma^2-(Y_i-\mathbbm{1}_{(x_i>c_1)})^2}\right]\\
    &=\sum_{i=1}^n\frac{1}{2\sigma^4}\left[(1-\hat{\xi}_i)(Y_i-2\mathbbm{1}_{(x_i>c_0)})^2-\sigma^2+\hat{\xi}_i(Y_i-\mathbbm{1}_{(x_i>c_1)})^2\right]
\end{align*}
so that
\begin{align*}
    n\hat{\sigma}^2=\sum_{i=1}^n(1-\hat{\xi}_i)(Y_i-2\mathbbm{1}_{(x_i>c_0)})^2+\hat{\xi}_i(Y_i-\mathbbm{1}_{(x_i>c_1)})^2\quad\Rightarrow\quad\hat{\sigma}^2=\frac{1}{n}\sum_{i=1}^n(1-\hat{\xi}_i)(Y_i-2\mathbbm{1}_{(x_i>c_0)})^2+\hat{\xi}_i(Y_i-\mathbbm{1}_{(x_i>c_1)})^2
\end{align*}
which leaves only the maximizers $\hat{c}_0$, $\hat{c}_1$ of $Q$ in $c_0$, $c_1$ respectively. The issue here is that $Q$ is not differentiable in $c_0$, $c_1$. Instead, we have
\begin{align*}
    \max_{c_0,c_1}\left\{Q(\theta|\theta^\prime)\right\}=\max_{c_0,c_1}\Bigg\{\sum_{i=1}^n&(1-\hat{\xi}_i)\bp{-\log(\sqrt{2\pi})-\frac{1}{2}\log(\sigma^2)-\frac{1}{2\sigma^2}(Y_i-2\mathbbm{1}_{(x_i>c_0)})^2+\log(1-p)}\\
    &+\hat{\xi}_i\bp{-\log(\sqrt{2\pi})-\frac{1}{2}\log(\sigma^2)-\frac{1}{2\sigma^2}(Y_i-\mathbbm{1}_{(x_i>c_1)})^2+\log(p)}\Bigg\}.
\end{align*}
It is sufficient to minimize this expression in only those terms depending on $c_0$ or $c_1$, so we compute
\begin{align*}
    \max_{c_0,c_1}\left\{\sum_{i=1}^n-\frac{1-\hat{\xi}_i}{2\sigma^2}(Y_i-2\mathbbm{1}_{(x_i>c_0)})^2-\frac{\hat{\xi}_i}{2\sigma^2}(Y_i-\mathbbm{1}_{(x_i>c_1)})^2\right\}&=\min_{c_0,c_1}\left\{\sum_{i=1}^n\frac{1-\hat{\xi}_i}{2\sigma^2}(Y_i-2\mathbbm{1}_{(x_i>c_0)})^2+\frac{\hat{\xi}_i}{2\sigma^2}(Y_i-\mathbbm{1}_{(x_i>c_1)})^2\right\}\\
    &=\min_{c_0}\left\{\sum_{i=1}^n(1-\hat{\xi}_i)(Y_i-2\mathbbm{1}_{(x_i>c_0)})^2\right\}+\min_{c_1}\left\{\sum_{i=1}^n\hat{\xi}_i(Y_i-\mathbbm{1}_{(x_i>c_1)})^2\right\}.
\end{align*} 
Now, let $\mc{D}:=\{x_{\alpha(j)}\}_{j=1}^n$ be a permutation of the feature variables so that $x_{\alpha(i)}\leq x_{\alpha(i+1)}$ for $i=1,\dots,n-1$. For any $1\leq i\leq n-1$, provided $x_{\alpha(i)}<x_{\alpha(i+1)}$, varying $x_{\alpha(i)}<c_1,c_0<x_{\alpha(i+1)}$ does not change the value of either term in the above objective function. It suffices then to minimize the objective over the finite set
\[\mc{U}_x:=\{x_{\alpha(j)}\in\mc{D}:x_{\alpha(j)}\neq x_{\alpha(k)}\;\forall k<j\}\]
which are the pairwise distinct values of the features in $\mc{D}_x$. Thus, to complete the M-step, we may compute
\begin{align*}
    \hat{c}_1=\argmin_{c_1\in\mc{U}_x}\left\{\sum_{i=1}^n\hat{\xi}_i(Y_i-\mathbbm{1}_{(x_i>c_1)})^2\right\}\quad\text{and}\quad\hat{c}_0=\argmin_{c_0\in\mc{U}_x}\left\{\sum_{i=1}^n(1-\hat{\xi}_i)(Y_i-2\mathbbm{1}_{(x_i>c_0)})^2\right\}.
\end{align*}
To summarize, the maximizers of each of the parameters are given by
\begin{align*}
    &\hat{p}=\frac{1}{n}\sum_{i=1}^n\hat{\xi}_i\\
    &\hat{\sigma}^2=\frac{1}{n}\sum_{i=1}^n(1-\hat{\xi}_i)(Y_i-2\mathbbm{1}_{(x_i>c_0)})^2+\hat{\xi}_i(Y_i-\mathbbm{1}_{(x_i>c_1)})^2\\
    &\hat{c}_1=\argmin_{c_1\in\mc{U}_x}\left\{\sum_{i=1}^n\hat{\xi}_i(Y_i-\mathbbm{1}_{(x_i>c_1)})^2\right\}\\
    &\hat{c}_0=\argmin_{c_0\in\mc{U}_x}\left\{\sum_{i=1}^n(1-\hat{\xi}_i)(Y_i-2\mathbbm{1}_{(x_i>c_0)})^2\right\}
\end{align*}
where $\hat{\xi}_i$, $i=1,\dots,n$ are as defined in the E-step.\hfill{$\qed$}
\end{document}