\documentclass[10pt]{article}
\usepackage[margin=1.3cm]{geometry}

% Packages
\usepackage{amsmath, amsfonts, amssymb, amsthm}
\usepackage{bbm} 
\usepackage{dutchcal} % [dutchcal, calrsfs, pzzcal] calligraphic fonts
\usepackage{graphicx}
\usepackage[T1]{fontenc}
\usepackage[tracking]{microtype}

% Palatino for text goes well with Euler
\usepackage[sc,osf]{mathpazo}   % With old-style figures and real smallcaps.
\linespread{1.025}              % Palatino leads a little more leading

% Euler for math and numbers
\usepackage[euler-digits,small]{eulervm}

% Command initialization
\DeclareMathAlphabet{\pazocal}{OMS}{zplm}{m}{n}
\graphicspath{{./images/}}

% Custom Commands
\newcommand{\bs}[1]{\boldsymbol{#1}}
\newcommand{\E}{\mathbb{E}}
\newcommand{\var}[1]{\text{Var}\left(#1\right)}
\newcommand{\bp}[1]{\left({#1}\right)}
\newcommand{\mbb}[1]{\mathbb{#1}}
\newcommand{\1}[1]{\mathbbm{1}_{#1}}
\newcommand{\mc}[1]{\mathcal{#1}}
\newcommand{\nck}[2]{{#1\choose#2}}
\newcommand{\pc}[1]{\pazocal{#1}}
\newcommand{\ra}[1]{\renewcommand{\arraystretch}{#1}}
\newcommand*{\floor}[1]{\left\lfloor#1\right\rfloor}
\newcommand*{\ceil}[1]{\left\lceil#1\right\rceil}

\DeclareMathOperator{\Var}{Var}
\DeclareMathOperator{\Cov}{Cov}
\DeclareMathOperator{\diag}{diag}
\DeclareMathOperator{\argmin}{arg\,min}
\DeclareMathOperator{\sgm}{sgm}

\newtheorem{theorem}{Theorem}
\newtheorem{lemma}{Lemma}

\begin{document}
    \begin{center}
        {\bf\large{MATH 857: STATISTICAL LEARNING II}}
        \smallskip
        \hrule
        \smallskip
        {\bf Assignment 3} \hfill {\bf Connor Braun} \hfill {\bf 2024-03-03}
    \end{center}
    \noindent{\bf Problem 2}\\[5pt]
    {\bf a)}\hspace{5pt}Let $\phi:\mathbb{R}\rightarrow\mbb{R}$ be a differentiable function. Show that if the first-order
    derivative $\phi^\prime$ is non-decreasing, then for any $x,y\in\mbb{R}$, we have
    \[\phi(y)\geq \phi(x)+\phi^\prime(x)(y-x).\]
    {\bf Proof}\hspace{5pt} With $\phi$ as described, take some $x,y\in\mathbb{R}$ and without loss of generality let us have that $x<y$ (since if $x=y$ then the proof is trivial -- that is, $\phi(y)=\phi(x)+0$ implies $\phi(y)=\phi(x)+\phi^\prime(x)(y-x)$). 
    Then $\phi$ is differentiable on $[x,y]$, so by the mean value theorem $\exists c\in[x,y]$ satisfying
    \[\phi(y)-\phi(x)=\phi^\prime(c)(y-x)\quad\Rightarrow\quad \phi(y)=\phi(x)+\phi^\prime(c)(y-x)\]
    but $\phi^\prime$ is non-decreasing, so $x\leq c$ implies that $\phi^\prime(x)\leq \phi^\prime(c)$ and so from the above we obtain
    \[\phi(y)\geq \phi(x)+\phi^\prime(x)(y-x)\]
    which holds since $y-x>0$. Thus, any such $\phi$ is convex. \hfill{$\qed$}\\[5pt]
\end{document}