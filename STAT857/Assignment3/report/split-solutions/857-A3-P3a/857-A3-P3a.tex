\documentclass[10pt]{article}
\usepackage[margin=1.3cm]{geometry}

% Packages
\usepackage{amsmath, amsfonts, amssymb, amsthm}
\usepackage{bbm} 
\usepackage{dutchcal} % [dutchcal, calrsfs, pzzcal] calligraphic fonts
\usepackage{graphicx}
\usepackage[T1]{fontenc}
\usepackage[tracking]{microtype}

% Palatino for text goes well with Euler
\usepackage[sc,osf]{mathpazo}   % With old-style figures and real smallcaps.
\linespread{1.025}              % Palatino leads a little more leading

% Euler for math and numbers
\usepackage[euler-digits,small]{eulervm}

% Command initialization
\DeclareMathAlphabet{\pazocal}{OMS}{zplm}{m}{n}
\graphicspath{{./images/}}

% Custom Commands
\newcommand{\bs}[1]{\boldsymbol{#1}}
\newcommand{\E}{\mathbb{E}}
\newcommand{\var}[1]{\text{Var}\left(#1\right)}
\newcommand{\bp}[1]{\left({#1}\right)}
\newcommand{\mbb}[1]{\mathbb{#1}}
\newcommand{\1}[1]{\mathbbm{1}_{#1}}
\newcommand{\mc}[1]{\mathcal{#1}}
\newcommand{\nck}[2]{{#1\choose#2}}
\newcommand{\pc}[1]{\pazocal{#1}}
\newcommand{\ra}[1]{\renewcommand{\arraystretch}{#1}}
\newcommand*{\floor}[1]{\left\lfloor#1\right\rfloor}
\newcommand*{\ceil}[1]{\left\lceil#1\right\rceil}

\DeclareMathOperator{\Var}{Var}
\DeclareMathOperator{\Cov}{Cov}
\DeclareMathOperator{\diag}{diag}
\DeclareMathOperator{\argmin}{arg\,min}
\DeclareMathOperator{\sgm}{sgm}

\newtheorem{theorem}{Theorem}
\newtheorem{lemma}{Lemma}

\begin{document}
    \begin{center}
        {\bf\large{MATH 857: STATISTICAL LEARNING II}}
        \smallskip
        \hrule
        \smallskip
        {\bf Assignment 3} \hfill {\bf Connor Braun} \hfill {\bf 2024-03-03}
    \end{center}
    \noindent{\bf Problem 3}\\[5pt]
    Consider a function $f:\mbb{R}\rightarrow\mbb{R}$ given by
    \[f(x)=(x-1)^6+(x-3)^4+(x-5)^2.\]
    Show that $f$ is convex.\\[5pt]
    {\bf Solution}\hspace{5pt} Since $\text{dom}(f)=\mbb{R}$ is convex, we may simply differentiate $f$ twice to find
    \begin{align*}
        f^\prime(x)=6(x-1)^5+4(x-3)^3+2(x-5),\qquad f^{\prime\prime}(x)=30(x-1)^4+12(x-3)^2+2>0\quad\forall x\in\mbb{R}
    \end{align*}
    so in fact $f$ is strictly convex on $\mbb{R}$.\hfill{$\qed$}
\end{document}