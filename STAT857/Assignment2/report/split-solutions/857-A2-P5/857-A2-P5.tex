\documentclass[10pt]{article}
\usepackage[margin=1.3cm]{geometry}

% Packages
\usepackage{amsmath, amsfonts, amssymb, amsthm}
\usepackage{bbm} 
\usepackage{dutchcal} % [dutchcal, calrsfs, pzzcal] calligraphic fonts
\usepackage{graphicx}
\usepackage[T1]{fontenc}
\usepackage[tracking]{microtype}

% Palatino for text goes well with Euler
\usepackage[sc,osf]{mathpazo}   % With old-style figures and real smallcaps.
\linespread{1.025}              % Palatino leads a little more leading

% Euler for math and numbers
\usepackage[euler-digits,small]{eulervm}

% Command initialization
\DeclareMathAlphabet{\pazocal}{OMS}{zplm}{m}{n}
\graphicspath{{./images/}}

% Custom Commands
\newcommand{\bs}[1]{\boldsymbol{#1}}
\newcommand{\E}{\mathbb{E}}
\newcommand{\var}[1]{\text{Var}\left(#1\right)}
\newcommand{\bp}[1]{\left({#1}\right)}
\newcommand{\mbb}[1]{\mathbb{#1}}
\newcommand{\1}[1]{\mathbbm{1}_{#1}}
\newcommand{\mc}[1]{\mathcal{#1}}
\newcommand{\nck}[2]{{#1\choose#2}}
\newcommand{\pc}[1]{\pazocal{#1}}
\newcommand{\ra}[1]{\renewcommand{\arraystretch}{#1}}
\newcommand*{\floor}[1]{\left\lfloor#1\right\rfloor}
\newcommand*{\ceil}[1]{\left\lceil#1\right\rceil}

\DeclareMathOperator{\Var}{Var}
\DeclareMathOperator{\Cov}{Cov}
\DeclareMathOperator{\diag}{diag}
\DeclareMathOperator{\argmin}{arg\,min}
\DeclareMathOperator{\sgm}{sgm}

\newtheorem{theorem}{Theorem}
\newtheorem{lemma}{Lemma}

\begin{document}
    \begin{center}
        {\bf\large{MATH 857: STATISTICAL LEARNING II}}
        \smallskip
        \hrule
        \smallskip
        {\bf Assignment 2} \hfill {\bf Connor Braun} \hfill {\bf 2024-02-08}
    \end{center}
    \noindent{\bf Problem 5}\\[5pt]
    Show that the following form a basis for the space of cubic splines with one interior knot at $\xi$:
    \begin{align*}
        h_1(x)=1,\quad h_2(x)=x,\quad h_3(x)=x^2,\quad h_4(x)=x^3,\quad h_5(x)=(x-\xi)^3_+.
    \end{align*}
    {\bf Proof}\hspace{5pt} First, let $(\gamma_i)_{i=1}^5$ with $\gamma_i\in\mbb{R}$ for $i=1,\dots,5$ specify a linear combination of our basis functions such that
    \begin{align*}
        g(x)=\gamma_1+\gamma_2x+\gamma_3x^2+\gamma^4x^3+\gamma^5(x-\xi)_+^3=\begin{cases}
            \gamma_1+\gamma_2x+\gamma_3x^2+\gamma_4x^3,\quad&\text{if $x<\xi$}\\
            \gamma_1+\gamma_2x+\gamma_3x^2+\gamma_4x^3+\gamma_5(x-\xi)^3,\quad&\text{if $\xi\leq x$}
        \end{cases}
    \end{align*}
    with $g$ clearly a piecewise-defined cubic polynomial on either side of $\xi$. We can also show this is indeed a spline by verifying the necessary continuity conditions
    \begin{align*}
        g(\xi^-)&=\lim_{x\rightarrow\xi^-}(\gamma_1+\gamma_2x+\gamma_3x^2+\gamma_4x^3)=\gamma_1+\gamma_2x+\gamma_3x^2+\gamma_4x^3=\lim_{x\rightarrow\xi^+}(\gamma_1+\gamma_2x+\gamma_3x^2+\gamma_4x^3+\gamma_5(x-\xi)^3_+)=g(\xi^+)\\
        g^{\prime}(\xi^-)&=\lim_{x\rightarrow\xi^-}(\gamma_2+2\gamma_3x+3\gamma_4x^2)=\gamma_2+2\gamma_3x+3\gamma_4x^2=\lim_{x\rightarrow\xi^+}(\gamma_2+2\gamma_3x+3\gamma_4x^2+3\gamma_5(x-\xi)^2)=g^\prime(\xi^+)\\
        g^{\prime\prime}(\xi^-)&=\lim_{x\rightarrow\xi^-}(2\gamma_3+6\gamma_4x)=2\gamma_3+6\gamma_4x=\lim_{x\rightarrow\xi^+}(2\gamma_3+6\gamma_4x+6\gamma_5(x-\xi))=g(\xi^+)
    \end{align*} 
    so any real linear combination of the basis functions produces a cubic spline with a single knot at $\xi$. Conversely, consider a cubic spline defined by
    \begin{align*}
        f(x)=\begin{cases}
            \alpha_0+\alpha_1x+\alpha_2x^2+\alpha_3x^3,\quad&\text{if $x<\xi$}\\
            \beta_0+\beta_1x+\beta_2x^2+\beta_3x^3,\quad&\text{if $x\geq \xi$}
        \end{cases}\quad\text{satisfying}\quad\begin{cases}
            \alpha_0+\alpha_1\xi+\alpha_2\xi^2\alpha_3\xi^3=\beta_3\xi^3+\beta_2\xi^2+\beta_1\xi+\beta_0\quad&\text{(1)}\\
            \alpha_1+2\alpha_2\xi+3\alpha_3\xi^2=3\beta_3\xi^2+2\beta_2\xi+\beta_1\quad&\text{(2)}\\
            2\alpha_2+6\alpha_3\xi=6\beta_3\xi+2\beta_2\quad&\text{(3)}
        \end{cases}
    \end{align*}
    now we set $g(x)=\gamma_0+\gamma_1x+\gamma_2x^2+\gamma_3x^3+\gamma_4(x-\xi)^3_+$, and solve for the coefficients $\gamma_i$, $i=0,1,\dots,4$ so that $g\equiv f$. First, observe that when $x<\xi$ we have
    $g(x)=\gamma_0+\gamma_1x+\gamma_2x^2+\gamma_3x^3$, so we require that $\gamma_i=\alpha_i$ for $i=0,1,2,3$. When instead we have $x\geq \xi$, then
    \begin{align*}
        g(x)&=\gamma_0+\gamma_1x+\gamma_2x^2+\gamma_3x^3+\gamma_4(x-\xi)^3\\
        &=\gamma_0+\gamma_1x+\gamma_2x^2+\gamma_3x^3+\gamma_4(x^3-3\xi x^2+3\xi^2x-\xi^3)\\
        &=(\gamma_3+\gamma_4)x^3+(\gamma_2-3\gamma_4\xi)x^2+(\gamma_1+3\gamma_4\xi^2)x+(\gamma_0-\gamma_4\xi^3)\\
        &=(\alpha_3+\gamma_4)x^3+(\alpha_2-3\gamma_4\xi)x^2+(\alpha_1+3\gamma_4\xi^2)x+(\alpha_0-\gamma_4\xi^3).
    \end{align*}
    This produces the overspecified system of equations
    \begin{align*}
        \begin{cases}
            \beta_3=\alpha_3+\gamma_4\\
            \beta_2=\alpha_2-3\gamma_4\xi\\
            \beta_1=\alpha_1+3\gamma_4\xi^2\\
            \beta_0=\alpha_0-\gamma_4\xi^3
        \end{cases}
    \end{align*}
    the solution to which is obtained by setting $\gamma_4=\beta_3-\alpha_3$. To see that this is in fact the solution, observe that it satisfies the first equation trivially, and for the second we have
    \begin{align*}
        \beta_2=\alpha_2-3\gamma_4\xi^2\quad\Rightarrow\quad 3\gamma_4\xi=\alpha_2-\beta_2=3\beta_3\xi-3\alpha_3\xi=3(\beta_3-\alpha_3)\xi.\tag{4}
    \end{align*}
    Where we used constraint (3) for the second last equality in (4). Similarly, the third equation implies that $3\gamma_3\xi^2=\beta_1-\alpha_1$, and so
    \begin{align*}
        3\gamma_4\xi^2=\beta_1-\alpha_1&=2\alpha_2\xi+3\alpha_3\xi^2-3\beta_3\xi^2-2\beta_2\xi\tag{by (2)}\\
        &=2(\alpha_2-\beta_2)\xi+3(\alpha_3-\beta_3)\xi^2\\
        &=6(\beta_3-\alpha_3)\xi^2-3(\beta_3-\alpha_3)\xi^2\tag{by (4)}\\
        &=3(\beta_3-\alpha_3)\xi^2.\tag{5}
    \end{align*}
    Lastly, from the fourth equation we get that $\gamma_4\xi^3=\alpha_0-\beta_0$, and so
    \begin{align*}
        \gamma_4\xi^3=\alpha_0-\beta_0&=\beta_3\xi^3+\beta_2\xi^2+\beta_1\xi-\alpha_3\xi^3-\alpha_2\xi^2-\alpha_1\xi\\
        &=(\beta_3-\alpha_3)\xi^3+(\beta_2-\alpha_2)\xi^2+(\beta_1-\alpha_1)\xi\\
        &=(\beta_3-\alpha_3)\xi^3-3(\beta_3-\alpha_3)\xi^3+3(\beta_3-\alpha_3)\xi^3\tag{by (4) and (5)}\\
        &=(\beta_3-\alpha_3)\xi^3.\tag{6}
    \end{align*}
    Examining (4), (5), and (6) we see that $\gamma_4=\beta_3-\alpha_3$ satisfies all members of the system of equations. Thus, setting
    \begin{align*}
        g(x)=\alpha_0+\alpha_1x+\alpha_2x^2\alpha_3x^3+(\beta_3-\alpha_3)(x-\xi)^3_+\quad\Rightarrow\quad g\equiv f.
    \end{align*}
    The coefficients $\beta_i$, $i=0,1,2$ do not appear in our truncated-power basis representation of $f$ since $f$ is overspecified. That is, we only need five parameters to
    specify a cubic spline with a single interior knot in the first place.\hfill{$\qed$}\\[5pt]
\end{document}