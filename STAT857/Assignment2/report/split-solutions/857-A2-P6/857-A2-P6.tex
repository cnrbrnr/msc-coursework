\documentclass[10pt]{article}
\usepackage[margin=1.3cm]{geometry}

% Packages
\usepackage{amsmath, amsfonts, amssymb, amsthm}
\usepackage{bbm} 
\usepackage{dutchcal} % [dutchcal, calrsfs, pzzcal] calligraphic fonts
\usepackage{graphicx}
\usepackage[T1]{fontenc}
\usepackage[tracking]{microtype}

% Palatino for text goes well with Euler
\usepackage[sc,osf]{mathpazo}   % With old-style figures and real smallcaps.
\linespread{1.025}              % Palatino leads a little more leading

% Euler for math and numbers
\usepackage[euler-digits,small]{eulervm}

% Command initialization
\DeclareMathAlphabet{\pazocal}{OMS}{zplm}{m}{n}
\graphicspath{{./images/}}

% Custom Commands
\newcommand{\bs}[1]{\boldsymbol{#1}}
\newcommand{\E}{\mathbb{E}}
\newcommand{\var}[1]{\text{Var}\left(#1\right)}
\newcommand{\bp}[1]{\left({#1}\right)}
\newcommand{\mbb}[1]{\mathbb{#1}}
\newcommand{\1}[1]{\mathbbm{1}_{#1}}
\newcommand{\mc}[1]{\mathcal{#1}}
\newcommand{\nck}[2]{{#1\choose#2}}
\newcommand{\pc}[1]{\pazocal{#1}}
\newcommand{\ra}[1]{\renewcommand{\arraystretch}{#1}}
\newcommand*{\floor}[1]{\left\lfloor#1\right\rfloor}
\newcommand*{\ceil}[1]{\left\lceil#1\right\rceil}

\DeclareMathOperator{\Var}{Var}
\DeclareMathOperator{\Cov}{Cov}
\DeclareMathOperator{\diag}{diag}
\DeclareMathOperator{\argmin}{arg\,min}
\DeclareMathOperator{\sgm}{sgm}

\newtheorem{theorem}{Theorem}
\newtheorem{lemma}{Lemma}

\begin{document}
    \begin{center}
        {\bf\large{MATH 857: STATISTICAL LEARNING II}}
        \smallskip
        \hrule
        \smallskip
        {\bf Assignment 2} \hfill {\bf Connor Braun} \hfill {\bf 2024-02-08}
    \end{center}
    \noindent{\bf Problem 6}\\[5pt]
    Consider the truncated-power series representation for cubic splines with $K$ knotes $\xi_1<\dots<\xi_K$. Let
    \[f(x)=\sum_{j=0}^3\beta_jx^j+\sum_{k=1}^K\theta_k(x-\xi_k)^3_+\tag{1}\]
    {\bf a)}\hspace{5pt} Prove that the natural boundary conditions for natural cubic splines imply the following linear constraints on the coefficients:
    \begin{align*}
        \beta_2=0,\quad\beta_3=0,\tag{2}\\
        \sum_{i=1}^K\theta_i=0,\quad\sum_{i=1}^K\xi_i\theta_i=0\tag{3}.
    \end{align*}
    {\bf Proof}\hspace{5pt} First, notice that when $x<\xi_1$ we have $f(x)=\sum_{j=1}^3\beta_jx^j$. However, the natural boundary conditions require that $f^{\prime\prime}(x)=0$ for all $x<\xi_1$ too, so now
    \begin{align*}
        f^{\prime\prime}(x)=2\beta_2+6\beta_3x=0\quad\Leftrightarrow\quad 2\beta_2=-6\beta_3x
    \end{align*}
    which holds $\forall x<\xi_1$, so it must be the case that $\beta_2=\beta_3=0$. Similarly, when $x>\xi_K$, $f$ is linear so $f^{\prime\prime}(x)=0$. This, along with the finding that $\beta_2=\beta_3=0$ gives us
    \begin{align*}
        f^{\prime\prime}(x)=6\sum_{i=1}^K\theta_i(x-\xi_i)=0\quad\Leftrightarrow\quad 6x\sum_{i=1}^K\theta_i=6\sum_{i=1}^K\theta_i\xi_i
    \end{align*}
    where, again, since this must hold $\forall x>\xi_K$, it must be the case that $\sum_{i=1}^K\theta_i=\sum_{i=1}^K\theta_i\xi_i=0$.\hfill{$\qed$}\\[5pt]
    {\bf b)}\hspace{5pt} Define 
    \[N_1(x)=1,\quad N_2(x)=x,\quad N_{i+2}(x)=d_i(x)-d_{K-1}(x),\quad i=1,\dots,K-2,\]
    where 
    \[d_i(x)=\frac{(x-\xi_i)^3_+-(x-\xi_K)^3_+}{\xi_K-\xi_i}\quad i=1,2,\dots, K-2.\]
    First, show that $g(x)=\sum_{i=1}^K\theta_iN_i(x)$ is a natural cubic spline. Then, show that if $f$ has form (1) and satisfies (2) and (3) then it can be written as
    \[\beta_0N_1(x)+\beta_1N_2(x)+\sum_{i=1}^{K-2}(\xi_K-\xi_i)\theta_iN_{i+2}(x).\tag{4}\]
    Conclude that $N_1,\dots N_K$ is a set of basis functions for the natural cubic spline with respect to knots $\xi_1,\dots,\xi_K$. \\[5pt]
    {\bf Proof}\hspace{5pt} First we shall show that $g$ satisfies the natural boundary conditions. When $x<\xi_1$, we have $d_i(x)=0$ for $i=1,2,\dots,K-1$, so $N_i(x)=0$ for $i=3,\dots,K$ and thus
    \begin{align*} 
        g(x)=\theta_1+\theta_2x
    \end{align*}
    which is linear. Now, letting $x>\xi_K$, we have that for $m=3,\dots,K$
    \begin{align*}
        N_m^{\prime\prime}(x)&=\frac{1}{\xi_K-\xi_m}\bp{6(x-\xi_m)-6(x-\xi_K)}-\frac{1}{\xi_K-\xi_{K-1}}\bp{6(x-\xi_{K-1})-6(x-\xi_K)}\\
        &=6\bp{\frac{\xi_K-\xi_m}{\xi_K-\xi_m}}-6\bp{\frac{\xi_K-\xi_{K-1}}{\xi_K-\xi_{K-1}}}\\
        &=0
    \end{align*}
    and
    \begin{align*}
        N^{\prime\prime\prime}(x)&=\frac{1}{\xi_{K}-\xi_m}(6-6)-\frac{1}{\xi_K-\xi_{K-1}}(6-6)=0
    \end{align*}
    so it must be that $g(x)$ is linear when $x>\xi_K$. It is easy to see that $g$ is a cubic polynomial piecewise defined between each of the knots, since for $x\in[\xi_m,\xi_{m+1}]$, with $m=1,\dots,K-2$,
    \begin{align*}
        g(x)=\sum_{i=1}^K\theta_iN_i(x)=\theta_1+\theta_2x+\sum_{i=3}^{m+2}\theta_i\bp{\frac{(x-\xi_{i-2})^3}{\xi_K-\xi_{i-2}}}
    \end{align*}
    similarly, for $x\in[\xi_{K-1},\xi_K]$,
    \begin{align*}
        g(x)=\theta_1+\theta_2x+\sum_{i=3}^{K}\theta_i\bp{\frac{(x-\xi_{i-2})^3}{\xi_K-\xi_{i-2}}-\frac{(x-\xi_{K-1})^3}{\xi_K-\xi_{K-1}}}
    \end{align*}
    and finally for $x\in [\xi_{K},\infty)$
    \begin{align*}
        g(x)&=\theta_1+\theta_2x+\sum_{i=3}^K\theta_i\bp{\frac{(x-\xi_{i-2})^3-(x-\xi_K)^3}{\xi_K-\xi_{i-2}}-\frac{(x-\xi_{K-1})^3-(x-\xi_K)^3}{\xi_K-\xi_{K-1}}}
    \end{align*}
    all three of which are linear combinations of cubic polynomials and so are themselves cubic polynomials, as claimed. It only remains to show that $g$ satisfies the continuity conditions at each of the knots. 
    Showing this is straightforward but exceptionally tedious, given how the functional form of $g$ varies over its domain. We compute
    \begin{align*}
        \lim_{x\rightarrow\xi_1^-}g(x)&=\lim_{x\rightarrow\xi_1^-}\theta_1+\theta_2x=\theta_1+\theta_2\xi_1\\
        \lim_{x\rightarrow\xi_1^-}g^\prime(x)&=\lim_{x\rightarrow\xi_1^-}\theta_2=\theta_2
    \end{align*}
    whereby the natural boundary condition we have $\lim_{x\rightarrow\xi_1^-}g^{\prime\prime}(x)=0$. For notational expediency, in what follows we adopt the convention that sums with final indices smaller than initial indices are equal to $0$. Letting $i=1,2\dots,K-2$:
    \begin{align*}
        \lim_{x\rightarrow\xi_i^+}g(x)&=\lim_{x\rightarrow\xi_i^+}\theta_1+\theta_2x+\sum_{j=3}^{i+2}\theta_j\bp{\frac{(x-\xi_{j-2})^3}{\xi_K-\xi_{j-2}}}=\theta_1+\theta_2\xi_i+\sum_{j=3}^{i+1}\theta_j\bp{\frac{(\xi_i-\xi_{j-2})^3}{\xi_K-\xi_{j-2}}}\\
        \lim_{x\rightarrow\xi_i^+}g^\prime(x)&=\lim_{x\rightarrow\xi_i^+}\theta_2+\sum_{j=3}^{i+2}\theta_j\bp{\frac{3(x-\xi_{j-2})^2}{\xi_K-\xi_{j-2}}}=\theta_2+\sum_{j=3}^{i+1}\theta_j\bp{\frac{3(\xi_i-\xi_{j-2})^2}{\xi_K-\xi_{j-2}}}\\
        \lim_{x\rightarrow\xi_i^+}g^{\prime\prime}(x)&=\lim_{x\rightarrow\xi_i^+}\sum_{j=3}^{i+2}\theta_j\bp{\frac{6(x-\xi_{j-2})}{\xi_K-\xi_{j-2}}}=\sum_{j=3}^{i+1}\theta_j\bp{\frac{6(\xi_i-\xi_{j-2})}{\xi_K-\xi_{j-2}}}\\
    \end{align*}
    where in the limit, the final term of the sums vanishes since $\xi_{i}-\xi_{i+2-2}=0$. In particular (and recalling our convention for sums with inadmissable indices) we can now see that when $i=1$, $\lim_{x\rightarrow\xi_i^-}g^{(m)}(x)=\lim_{x\rightarrow\xi_i^+}g^{(m)}(x)$ for $m=0,1,2$. Next, for $i=2,3,\dots,K-1$ we have
    \begin{align*}
        \lim_{x\rightarrow\xi_i^-}g(x)&=\lim_{x\rightarrow\xi_i^-}\theta_1+\theta_2x+\sum_{j=3}^{i+1}\theta_j\bp{\frac{(x-\xi_{j-2})^3}{\xi_K-\xi_{j-2}}}=\theta_1+\theta_2\xi_i+\sum_{j=3}^{i+1}\theta_j\bp{\frac{(\xi_i-\xi_{j-2})^3}{\xi_K-\xi_{j-2}}}\\
        \lim_{x\rightarrow\xi_i^-}g^\prime(x)&=\lim_{x\rightarrow\xi_i^-}\theta_2+\sum_{j=3}^{i+1}\theta_j\bp{\frac{3(x-\xi_{j-2})^2}{\xi_K-\xi_{j-2}}}=\theta_2+\sum_{j=3}^{i+1}\theta_j\bp{\frac{3(\xi_i-\xi_{j-2})^2}{\xi_K-\xi_{j-2}}}\\
        \lim_{x\rightarrow\xi_i^-}g^{\prime\prime}(x)&=\lim_{x\rightarrow\xi_i^-}\sum_{j=3}^{i+1}\theta_j\bp{\frac{6(x-\xi_{j-2})}{\xi_K-\xi_{j-2}}}=\sum_{j=3}^{i+1}\theta_j\bp{\frac{6(\xi_i-\xi_{j-2})}{\xi_K-\xi_{j-2}}}\\
    \end{align*}
    so now the necessary continuity conditions $\lim_{x\rightarrow\xi_i^-}g^{(m)}(x)=\lim_{x\rightarrow\xi_i^+}g^{(m)}(x)$ are verified on knots $i=1,2,\dots,K-2$ and for $m=0,1,2$. Now, on the interval $[\xi_{K-1},\xi_K]$:
    \begin{align*}
        \lim_{x\rightarrow\xi_{K-1}^+}g(x)&=\lim_{x\rightarrow\xi_{K-1}^+}\theta_1+\theta_2x+\sum_{j=3}^K\theta_j\bp{\frac{(x-\xi_{j-2})^3}{\xi_K-\xi_{j-2}}-\frac{(x-\xi_{K-1})^3}{\xi_K-\xi_{K-1}}}=\theta_1+\theta_2\xi_{K-1}+\sum_{j=3}^K\theta_j\bp{\frac{(\xi_{K-1}-\xi_{j-2})^3}{\xi_K-\xi_{j-2}}}\\
        \lim_{x\rightarrow\xi_{K-1}^+}g^\prime(x)&=\lim_{x\rightarrow\xi_{K-1}^+}\theta_2+\sum_{j=3}^K\theta_j\bp{\frac{3(x-\xi_{j-2})^2}{\xi_K-\xi_{j-2}}-\frac{3(x-\xi_{K-1})^2}{\xi_K-\xi_{K-1}}}=\theta_2+\sum_{j=3}^K\theta_j\bp{\frac{3(\xi_{K-1}-\xi_{j-2})^2}{\xi_K-\xi_{j-2}}}\\
        \lim_{x\rightarrow\xi_{K-1}^+}g^{\prime\prime}(x)&=\lim_{x\rightarrow\xi_{K-1}^+}\sum_{j=3}^K\theta_j\bp{\frac{6(x-\xi_{j-2})}{\xi_K-\xi_{j-2}}-\frac{6(x-\xi_{K-1})}{\xi_K-\xi_{K-1}}}=\sum_{j=3}^K\theta_j\bp{\frac{6(\xi_{K-1}-\xi_{j-2})}{\xi_K-\xi_{j-2}}}\\
    \end{align*}
    where now with the previous set of limits we can observe that $\lim_{x\rightarrow\xi_{K-1}^-}g^{(m)}(x)=\lim_{x\rightarrow\xi_{K-1}^+}g^{(m)}(x)$ for $m=0,1,2$. Now, for the final knot: 
    \begin{align*}
        \lim_{x\rightarrow\xi_{K}^-}g(x)&=\lim_{x\rightarrow\xi_{K}^-}\theta_1+\theta_2x+\sum_{j=3}^K\theta_j\bp{\frac{(x-\xi_{j-2})^3}{\xi_K-\xi_{j-2}}-\frac{(x-\xi_{K-1})^3}{\xi_K-\xi_{K-1}}}=\theta_1+\theta_2\xi_{K}+\sum_{j=3}^K\theta_j\bp{(\xi_{K}-\xi_{j-2})^2-(\xi_K-\xi_{K-1})^2}\\
        \lim_{x\rightarrow\xi_{K}^-}g^\prime(x)&=\lim_{x\rightarrow\xi_{K}^-}\theta_2+\sum_{j=3}^K\theta_j\bp{\frac{3(x-\xi_{j-2})^2}{\xi_K-\xi_{j-2}}-\frac{3(x-\xi_{K-1})^2}{\xi_K-\xi_{K-1}}}=\theta_2+\sum_{j=3}^K\theta_j\bp{3(\xi_{K}-\xi_{j-2})-3(\xi_K-\xi_{K-1})}\\
        \lim_{x\rightarrow\xi_{K}^-}g^{\prime\prime}(x)&=\lim_{x\rightarrow\xi_{K}^-}\sum_{j=3}^K\theta_j\bp{\frac{6(x-\xi_{j-2})}{\xi_K-\xi_{j-2}}-\frac{6(x-\xi_{K-1})}{\xi_K-\xi_{K-1}}}=\sum_{j=3}^K\theta_j\bp{6-6}=0\\
    \end{align*}
    checking the right-continuity of the functions at $\xi_K$, we once again need only check up to the first derivative since we know $g$ satisfies the natural boundary conditions. That is, 
    \begin{align*}
        \lim_{x\rightarrow\xi_K^+}g(x)&=\lim_{x\rightarrow\xi_K^+}\theta_1+\theta_2x+\sum_{j=3}^K\theta_j\bp{\frac{(x-\xi_{j-2})^3-(x-\xi_K)^3}{\xi_K-\xi_{j-2}}-\frac{(x-\xi_{K-1})^3-(x-\xi_K)^3}{\xi_K-\xi_{K-1}}}\\
        &=\theta_1+\theta_2\xi_{K}+\sum_{j=3}^K\theta_j\bp{(\xi_{K}-\xi_{j-2})^2-(\xi_K-\xi_{K-1})^2}\\
        \lim_{x\rightarrow\xi_K^+}g^\prime(x)&=\lim_{x\rightarrow\xi_K^+}\theta_2+\sum_{j=3}^K\theta_j\bp{\frac{3(x-\xi_{j-2})^2-3(x-\xi_K)^2}{\xi_K-\xi_{j-2}}-\frac{3(x-\xi_{K-1})^2-3(x-\xi_K)^2}{\xi_K-\xi_{K-1}}}\\
        &=\theta_2+\sum_{j=3}^K\theta_j\bp{3(\xi_{K}-\xi_{j-2})-3(\xi_K-\xi_{K-1})}
    \end{align*}
    so, at long last, we have shown finally that $\lim_{x\rightarrow\xi_K^-}g^{(m)}(x)=\lim_{x\rightarrow\xi_K^+}g^{(m)}(x)$ for $m=0,1,2$. Since $g$ is a piecewise defined cubic polynomial satisfying the natural boundary conditions and has continuous derivatives up to second order, $g$
    is a natural cubic spline.\\[5pt]
    Next, we show that a function $f$ of the form specified in (1) satisfying (2) and (3) can be written as (4). Specifically, we have $\beta_2=\beta_3=0$ and $\sum_{j=1}^K\theta_j=\sum_{j=1}^K\theta_j\xi_j=0$, and so find
    \begin{align*}
        f(x)=\sum_{j=1}^3\beta_jx^j+\sum_{i=1}^K\theta_i(x-\xi_i)^3_+=\beta_0+\beta_1x+\sum_{i=1}^K\theta_i(x-\xi_i)^3_+=\beta_0N_1(x)+\beta_1N_2(x)+\sum_{i=1}^K\theta_i(x-\xi_i)^3_+
    \end{align*}
    so that we need only analyze the sum $\sum_{i=1}^K\theta_i(x-\xi_i)^3_+$. Fixing some $1\leq i\leq K$, we can write
    \begin{align*}
        \theta_i(x-\xi_i)^3_+&=\theta_i(\xi_K-\xi_i)\bp{\frac{(x-\xi_i)^3_+}{\xi_K-\xi_i}-\frac{(x-\xi_K)^3_+}{\xi_K-\xi_{i}}+\frac{(x-\xi_K)^3_+}{\xi_K-\xi_i}}\\
        &=\theta_i(\xi_K-\xi_i)\bp{\frac{(x-\xi_i)^3_+-(x-\xi_K)^3_+}{\xi_K-\xi_i}-\frac{(x-\xi_{K-1})^3_+-(x-\xi_K)^3_+}{\xi_K-\xi_{K-1}}+\frac{(x-\xi_K)^3_+}{\xi_K-\xi_i}+\frac{(x-\xi_{K-1})^3_+-(x-\xi_K)^3_+}{\xi_K-\xi_{K-1}}}\\
        &=\theta_i(\xi_K-\xi_i)N_{i+2}(x)+\theta_i(x-\xi_K)^3_++\theta_i(\xi_K-\xi_i)d_{K-1}(x)
    \end{align*}
    and so
    \begin{align*}
        f(x)&=\beta_0N_1(x)+\beta_1N_2(x)+\sum_{i=1}^K\theta_i(x-\xi_i)^3_+\\
        &=\beta_0N_1(x)+\beta_1N_2(x)+\bp{\sum_{i=1}^K\theta_i(\xi_K-\xi_i)N_{i+2}(x)}+\bp{(x-\xi_K)^3_+\sum_{i=1}^K\theta_i}+d_{K-1}\bp{\xi_K\sum_{i=1}^K\theta_i-\sum_{i=1}^K\theta_i\xi_i}\\
        &=\beta_0N_1(x)+\beta_1N_2(x)+\sum_{i=1}^K\theta_i(\xi_K-\xi_i)N_{i+2}(x)
    \end{align*}
    with the last equality holding since $f$ satisfies (3). Finally, note that when $i=K$, $(\xi_K-\xi_i)=0$, and when $i=K-1$, $N_{K-1}(x)=0$, so these terms of the sum are equal to zero and we arrive at
    \[f(x)=\beta_0N_1(x)+\beta_1N_2(x)+\sum_{i=1}^{K-2}\theta_i(\xi_K-\xi_i)N_{i+2}(x)\]
    which was the desired expression for $f$. Moreover, we have shown that the functions $N_1,N_2,\dots N_K$ form a basis in the space of natural cubic splines with respect to knots $\{\xi_i\}_{i=1}^K$.\hfill{$\qed$}
\end{document}
\end{document}